%Template
%Copyright (C) 2019  Patrick Diehl
%
%This program is free software: you can redistribute it and/or modify
%it under the terms of the GNU General Public License as published by
%the Free Software Foundation, either version 3 of the License, or
%(at your option) any later version.

%This program is distributed in the hope that it will be useful,
%but WITHOUT ANY WARRANTY; without even the implied warranty of
%MERCHANTABILITY or FITNESS FOR A PARTICULAR PURPOSE.  See the
%GNU General Public License for more details.

%You should have received a copy of the GNU General Public License
%along with this program.  If not, see <http://www.gnu.org/licenses/>.
\documentclass[12pt,t]{beamer}

\beamertemplatenavigationsymbolsempty

% usepackage
%\usepackage{template/dbt}
\usepackage{listings}
\usepackage{tikz}
\usetikzlibrary{shapes.geometric, arrows,matrix}
\usetikzlibrary{calc}
\usepackage{xfrac}

\definecolor{comments}{RGB}{81,81,81}
\definecolor{keywords}{RGB}{255,0,90}

% lstlisting
\lstset{
    language=C,
    basicstyle=\footnotesize\ttfamily,
    keywordstyle=\color{keywords},
    showspaces=false,
    showstringspaces=false,
    commentstyle=\color{blue}\emph
    %frame=single,
    %rulecolor=\color{comments},
    %rulesepcolor=\color{comments},
    %backgroundcolor = \color{lightgray}
}

\usetheme{default}

\usepackage[
    type={CC},
    modifier={by-nc-nd},
    version={4.0},
]{doclicense} 

\newcommand{\courseurl}[0]{https://www.cct.lsu.edu/\string~pdiehl/teaching/2021/4997/}
\newcommand{\coursetimeline}[0]{https://www.cct.lsu.edu/~pdiehl/teaching/2021/4997/timeline.pdf}
\newcommand{\coursesyllabus}[0]{https://www.cct.lsu.edu/~pdiehl/teaching/2021/4997/syllabus.pdf}
\newcommand{\coursename}[0]{Math 4997-3}
\newcommand{\coursemailinglist}[0]{https://mail.cct.lsu.edu/mailman/listinfo/par4997}
\newcommand{\coursesemester}{Fall 2021}








% frame slide
\title{\coursename}
\subtitle{Lecture 7: Asynchronous programming }

%\author{\href{}{}}
%\institute {
%    \href{}{\tt \scriptsize \today}
%}
\date {
 \tiny \url{\courseurl}
\vspace{2cm}
\doclicenseThis  
  
}



\usepackage{ifxetex}

\ifxetex
\usepackage{fontspec}
\setmainfont{Raleway}
\fi

\begin{document} {
    \setbeamertemplate{footline}{}
    \frame {
        \titlepage
    }
}

\frame{

\tableofcontents

}

\AtBeginSection[]{
  \begin{frame}
  \vfill
  \centering
  \begin{beamercolorbox}[sep=8pt,center,shadow=true,rounded=true]{title}
    \usebeamerfont{title}\insertsectionhead\par%
  \end{beamercolorbox}
  \vfill
  \end{frame}
}

%%%%%%%%%%%%%%%%%%%%%%%%%%%%%%%%%%%%%%%%%%%%%%%%%%%%%%%%%%%%%%%%%%%%%%%%%%%%%%
\section{Reminder}
%%%%%%%%%%%%%%%%%%%%%%%%%%%%%%%%%%%%%%%%%%%%%%%%%%%%%%%%%%%%%%%%%%%%%%%%%%%%%%
\begin{frame}{Lecture 6}
\begin{block}{What you should know from last lecture}
\begin{itemize}
\item Shared memory parallelism
\item Parallel algorithms and execution policies
\item Data races and dead locks
\end{itemize}
\end{block}
\end{frame}


%%%%%%%%%%%%%%%%%%%%%%%%%%%%%%%%%%%%%%%%%%%%%%%%%%%%%%%%%%%%%%%%%%%%%%%%%%%%%%
\section{Asynchronous programming}
%%%%%%%%%%%%%%%%%%%%%%%%%%%%%%%%%%%%%%%%%%%%%%%%%%%%%%%%%%%%%%%%%%%%%%%%%%%%%%

\begin{frame}[fragile]{Synchronous programming}


\begin{columns}[T]
\begin{column}{0.5\textwidth}
   \begin{block}{Dependency graph}
\begin{center}
\begin{tikzpicture}
\node (a) [draw,circle] at (0,2) {$H$};
\node (b) [draw,circle] at (0,0) {$P$};
\node (c) [draw,circle] at (1,0) {$X$};
\draw [->] (a) -- (b);
\draw [->] (a) -- (c);
\end{tikzpicture}
\end{center}
\end{block}
\end{column}
\begin{column}{0.5\textwidth}  %%<--- here
\begin{block}{Code}
\begin{lstlisting}
auto P = compute();
auto X = compute();
auto H = compute(P,X);
\end{lstlisting}
\end{block}
\end{column}
\end{columns}

\begin{itemize}
\item The program is executed line by line
\item Each time a function is called the code waits until the functions finishes
\item We can not compute P and X at the same time, since the data is independent
\end{itemize}
\end{frame}


\begin{frame}[fragile]{Asynchronous programming~\cite{williams2012c++} }
\begin{block}{Code}
\begin{lstlisting}
int P,X = 1;

std::future<int> f1 = std::async(compute,P);
auto f2 = std::async(compute,X);

std::cout << compute(f1.get() + f2.get()) << std::endl;
\end{lstlisting}
\end{block}

\begin{itemize}
\item The program is some times executed line by line
\item Calling \lstinline|std::async| the next line is executed, even if the function has not finished yet
\item We have to use the \lstinline|std::future| to synchronize the asynchronous function calls
\end{itemize}
\tiny More details: CppCon 2017: H. Kaiser “The Asynchronous C++ Parallel Programming Model”\footnote{\tiny\url{https://www.youtube.com/watch?v=js-e8xAMd1s}}
\end{frame}


\begin{frame}[fragile]{Asynchronous execution of functions\footnote{\tiny\url{http://www.cplusplus.com/reference/future/async/}}}

\begin{lstlisting}
bool is_prime (int x) {
  std::cout << "Calculating. Please, wait...\n";
  for (int i=2; i<x; ++i) if (x%i==0) return false;
  return true;
}

std::future<bool> f = std::async (is_prime,313222313);

\end{lstlisting}

\begin{itemize}
\item The first argument \lstinline|fn| is a function pointer
\item The second argument is the first argument of the function, and so on
\item The return value is a \lstinline|std::future<T>| where \lstinline|T| is the return type of the function
\end{itemize}
For each call of \lstinline|std::async| launches a new thread to execute the function the function pointer \lstinline|fn| points to. 
\end{frame}


\begin{frame}{Futurization\footnote{\tiny\url{https://en.cppreference.com/w/cpp/thread/future}}}
A \lstinline|std::future| provides a mechanism to access the result of asynchronous operations, like \lstinline|std::async| and provides methods for synchronization.

\begin{block}{Synchronization}

\begin{itemize}
\item \lstinline|.get()| returns the result of the functions and wait until the computation finished
\item \lstinline|.wait()| waits until the computation finished
\item \lstinline|.wait_for(std::chrono::seconds(1))| returns if it is not available for the specified timeout duration 
\item \lstinline|.wait_until(std::chrono::seconds(1))| waits for a result to become available. It blocks until specified timeout time has been reached or the result becomes available, whichever comes first. 
\end{itemize}

\end{block}
\end{frame}

\begin{frame}{Parallelism using asynchronous programming}

\begin{block}{Example: Taylor series}
$$
sin(x) = \sum\limits_{n=0}^n (-1)^{n-1} \frac{x^{2n}}{(2n)!} 
$$
\end{block}

\begin{block}{Approach}
\begin{enumerate}
\item Split $n$ into slices, e.g. 2 times $\sfrac{n}{2}$ for two threads
\item Start two times \lstinline|std::async| where each thread computes $\sfrac{n}{2}$
\item Use the two futures to synchronize the results
\item Combine the two futures to obtain the result
\end{enumerate}
\end{block}

\end{frame}

\begin{frame}[fragile]{Implementation I}

\begin{block}{Function}
\begin{lstlisting}
double taylor(size_t begin, size_t end, 
double x,size_t n){
double res = 0;

for( size_t i = begin ; i < end ; i++)
{
res += pow(-1,i-1) * pow(x,2*n) / factorial(2*n);
} 
return res;
}

\end{lstlisting}
\end{block}

\begin{itemize}
\item With \lstinline|begin| and \lstinline|end|, the range is defined
\item The range needs to be adapted to the amount of threads you want to launch
\end{itemize}
\end{frame}

\begin{frame}[fragile]{Implementation II}

\begin{block}{Launching}
\begin{lstlisting}
auto f1 = std::async(taylor,0,49,2,100); 
auto f2 = std::async(taylor,50,99,2,100); 
\end{lstlisting}
\end{block}

\begin{block}{Launching}
\begin{lstlisting}
double result = f1.get() + f2.get();
\end{lstlisting}
\end{block}
\end{frame}

%%%%%%%%%%%%%%%%%%%%%%%%%%%%%%%%%%%%%%%%%%%%%%%%%%%%%%%%%%%%%%%%%%%%%%%%%%%%%%
\section{Lambda functions}
%%%%%%%%%%%%%%%%%%%%%%%%%%%%%%%%%%%%%%%%%%%%%%%%%%%%%%%%%%%%%%%%%%%%%%%%%%%%%%

\begin{frame}[fragile]{Lambda expression\footnote{\tiny{\url{https://en.cppreference.com/w/cpp/language/lambda}}} }

\begin{block}{Structure}
\begin{lstlisting}
[ capture clause ] (parameters) -> return-type  
{   
   definition of method   
} 
\end{lstlisting}
\end{block}

\begin{block}{Notes}
\begin{itemize}
\item Generally return-type in lambda expression are evaluated by compiler
\item Capture clause:
\begin{itemize}
\item  \lstinline|[&]| : capture all external variable by reference
\item  \lstinline|[=]| : capture all external variable by value
\item  \lstinline|[a, &b]| : capture a by value and b by reference
\end{itemize}
\end{itemize}
\end{block}
\vspace{-0.2cm}
\begin{center}
More about the capture clauses in lecture 11/12.
\end{center}
\end{frame}

\begin{frame}[fragile]{Practical example}
\begin{lstlisting}
std::vector<int> v {4, 1, 3, 5, 2, 3, 1, 7}; 

\end{lstlisting}


\begin{block}{Classical function}
\begin{lstlisting}
void print(int i){
std::cout << i << std::endl;
}
std::for_each(v.begin(), v.end(), print); 
\end{lstlisting}
\end{block}

\begin{block}{Lambda expression}
\begin{lstlisting}
std::for_each(v.begin(),v.end(),
	[](int i){std::cout<< i << std::endl;})
\end{lstlisting}
\end{block}


\end{frame}
     
\begin{frame}[fragile]{More examples}

\begin{block}{\lstinline|find_if|\footnote{\tiny\url{https://en.cppreference.com/w/cpp/algorithm/find}}}
\begin{lstlisting}
    std::vector<int>:: iterator p = find_if(
    v.begin(), 
    	v.end(), 
    	[](int i) 
    { 
        return i > 4; 
    }); 
    std::cout << "First number greater than 4 is : 
    " << *p 
    << endl;
\end{lstlisting}
\end{block}
Many more algorithms are available in the \lstinline|#include <algorithm>|\footnote{\tiny\url{https://en.cppreference.com/w/cpp/algorithm}}
\end{frame}
     


%%%%%%%%%%%%%%%%%%%%%%%%%%%%%%%%%%%%%%%%%%%%%%%%%%%%%%%%%%%%%%%%%%%%%%%%%%%%%%
\section{Summary}
%%%%%%%%%%%%%%%%%%%%%%%%%%%%%%%%%%%%%%%%%%%%%%%%%%%%%%%%%%%%%%%%%%%%%%%%%%%%%%
\begin{frame}{Summary}
\begin{block}{After this lecture, you should know}
\begin{itemize}
\item Asynchronous programming \lstinline|std::async| and \lstinline|std::future|
\item Lambda functions
\end{itemize}
\end{block}
\end{frame}


%%%%%%%%%%%%%%%%%%%%%%%%%%%%%%%%%%%%%%%%%%%%%%%%%%%%%%%%%%%%%%%%%%%%%%%%%%%%%%
\section{References}
%%%%%%%%%%%%%%%%%%%%%%%%%%%%%%%%%%%%%%%%%%%%%%%%%%%%%%%%%%%%%%%%%%%%%%%%%%%%%%

\begin{frame}[t, allowframebreaks]
\frametitle{References}
\bibliographystyle{plain}
\bibliography{bib}
\end{frame}
\end{document}