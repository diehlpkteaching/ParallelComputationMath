%Template
%Copyright (C) 2019  Patrick Diehl
%
%This program is free software: you can redistribute it and/or modify
%it under the terms of the GNU General Public License as published by
%the Free Software Foundation, either version 3 of the License, or
%(at your option) any later version.

%This program is distributed in the hope that it will be useful,
%but WITHOUT ANY WARRANTY; without even the implied warranty of
%MERCHANTABILITY or FITNESS FOR A PARTICULAR PURPOSE.  See the
%GNU General Public License for more details.

%You should have received a copy of the GNU General Public License
%along with this program.  If not, see <http://www.gnu.org/licenses/>.
\documentclass[12pt,t]{beamer}

\beamertemplatenavigationsymbolsempty

% usepackage
%\usepackage{template/dbt}
\usepackage{listings}

\definecolor{comments}{RGB}{81,81,81}
\definecolor{keywords}{RGB}{255,0,90}

% lstlisting
\lstset{
    language=C,
    basicstyle=\footnotesize\ttfamily,
    keywordstyle=\color{keywords},
    showspaces=false,
    showstringspaces=false,
    commentstyle=\color{blue}\emph
    %frame=single,
    %rulecolor=\color{comments},
    %rulesepcolor=\color{comments},
    %backgroundcolor = \color{lightgray}
}

\usetheme{default}

\usepackage[
    type={CC},
    modifier={by-nc-nd},
    version={4.0},
]{doclicense} 

\usepackage{tikz}

\newcommand{\courseurl}[0]{https://www.cct.lsu.edu/\string~pdiehl/teaching/2021/4997/}
\newcommand{\coursetimeline}[0]{https://www.cct.lsu.edu/~pdiehl/teaching/2021/4997/timeline.pdf}
\newcommand{\coursesyllabus}[0]{https://www.cct.lsu.edu/~pdiehl/teaching/2021/4997/syllabus.pdf}
\newcommand{\coursename}[0]{}
\newcommand{\coursemailinglist}[0]{https://mail.cct.lsu.edu/mailman/listinfo/par4997}
\newcommand{\coursesemester}{Fall 2023}
\newcommand{\courselocation}{TBA}
\newcommand{\coursedate}{Tuesday and Thursday, 09:00 to 10:20}
\newcommand{\courseserver}[0]{https://hpx-tutorial.cct.lsu.edu}
\newcommand{\coursetitle}[0]{Applications of high performance computing with C++ in biomedical science and engineering}




% frame slide
\title{\coursename}
\subtitle{Lecture 7: Shared memory  parallelism}

%\author{\href{}{}}
%\institute {
%    \href{}{\tt \scriptsize \today}
%}
\date {
 \tiny \url{\courseurl}
\vspace{2cm}
\doclicenseThis  
  
}



\usepackage{ifxetex}

\ifxetex
\usepackage{fontspec}
\setmainfont{Raleway}
\fi

\begin{document} {
    \setbeamertemplate{footline}{}
    \frame {
        \titlepage
    }
}

\frame{

\tableofcontents

}

\AtBeginSection[]{
  \begin{frame}
  \vfill
  \centering
  \begin{beamercolorbox}[sep=8pt,center,shadow=true,rounded=true]{title}
    \usebeamerfont{title}\insertsectionhead\par%
  \end{beamercolorbox}
  \vfill
  \end{frame}
}

%%%%%%%%%%%%%%%%%%%%%%%%%%%%%%%%%%%%%%%%%%%%%%%%%%%%%%%%%%%%%%%%%%%%%%%%%%%%%%
\section{Reminder}
%%%%%%%%%%%%%%%%%%%%%%%%%%%%%%%%%%%%%%%%%%%%%%%%%%%%%%%%%%%%%%%%%%%%%%%%%%%%%%
\begin{frame}{Lecture X}
\begin{block}{What you should know from last lecture}
\begin{itemize}
\item Lamba functions
\item Advanced loops
\end{itemize}
\end{block}
\end{frame}

%%%%%%%%%%%%%%%%%%%%%%%%%%%%%%%%%%%%%%%%%%%%%%%%%%%%%%%%%%%%%%%%%%%%%%%%%%%%%%
\section{Shared memory parallelism}
%%%%%%%%%%%%%%%%%%%%%%%%%%%%%%%%%%%%%%%%%%%%%%%%%%%%%%%%%%%%%%%%%%%%%%%%%%%%%%

\begin{frame}{Concept of shared memory papallelism~\cite{el2005advanced,hager2010introduction}}


\begin{center}
\begin{tikzpicture}
%Threads
\draw (0,3) rectangle (.5,3.5) node[pos=.5] {1};
\draw (0.5,3) rectangle (1.0,3.5) node[pos=.5] {..};
\draw (1.,3) rectangle (1.5,3.5) node[pos=.5] {n};

\draw (2.5,3) rectangle (3.,3.5) node[pos=.5] {1};
\draw (3.,3) rectangle (3.5,3.5) node[pos=.5] {..};
\draw (3.5,3) rectangle (4,3.5) node[pos=.5] {n};

\draw (0.75,2.5) -- (0.75,3.);
\draw (3.25,2.5) -- (3.25,3.);

%BUS
\draw (0,1) rectangle (4,1.5) node[pos=.5] {System Bus};

%CPU
\draw (0,2) rectangle (1.5,2.5) node[pos=.5] {CPU 1};
\draw (2.5,2) rectangle (4,2.5) node[pos=.5] {CPU 2};
\draw (0.75,2) -- (0.75,1.5);
\draw (3.25,2) -- (3.25,1.5);

%Memory
\draw (0,0) rectangle (4,0.5) node[pos=.5] {Memory};
\draw (0.75,.5) -- (0.75,1.);
\draw (3.25,.5) -- (3.25,1.);
\end{tikzpicture}
\end{center}

\begin{block}{System design}
\begin{itemize}
\item uniform memory access (UMA)
\item non-uniform memory access (NUMA)
\end{itemize}
\end{block}

\end{frame}




%%%%%%%%%%%%%%%%%%%%%%%%%%%%%%%%%%%%%%%%%%%%%%%%%%%%%%%%%%%%%%%%%%%%%%%%%%%%%%
\section{Parallelism}
%%%%%%%%%%%%%%%%%%%%%%%%%%%%%%%%%%%%%%%%%%%%%%%%%%%%%%%%%%%%%%%%%%%%%%%%%%%%%%

\begin{frame}[fragile]{Parallel algorithms in C++ 17\footnote{\tiny\url{https://en.cppreference.com/w/cpp/experimental/parallelism}}}
\begin{itemize}
\item C++17 added support for parallel algorithms to the standard library, to help programs take advantage of parallel execution for improved performance.
\item Parallelized versions of 69 algorithms from \lstinline|<algorithm>|, \lstinline|<numeric>| and \lstinline|<memory>| are available. 
\end{itemize}

\begin{block}{Recently new feature!}
Only recently released compilers (gcc 9 and MSVC 19.14) implement these new features and some of them are still experimental.\\
\vspace{0.5cm}
Some special compiler flags are needed to use these features:

\begin{lstlisting}[language=bash]
g++ -std=c++1z -ltbb lecture7-loops.cpp 
\end{lstlisting}
\end{block}
\end{frame}

\begin{frame}[fragile]{Example: Accumulate}

\begin{lstlisting}
std::vector<int> nums(1000000,0);
\end{lstlisting}

\begin{block}{Sequential\footnote{\tiny\url{https://en.cppreference.com/w/cpp/algorithm/accumulate}}}
\begin{lstlisting}
auto result = std::accumulate(nums.begin(), 
                              nums.end(),
                              0.0);
\end{lstlisting}
\end{block}

\begin{block}{Parallel\footnote{\tiny\url{https://en.cppreference.com/w/cpp/experimental/reduce}}}
\begin{lstlisting}
auto result = std::reduce(
              std::execution::par,
              nums.begin(), nums.end());
\end{lstlisting}
\end{block}

Important: \lstinline|std::execution::par| from \lstinline|#include<execution>|\footnote{\tiny\url{https://en.cppreference.com/w/cpp/experimental/execution_policy_tag}}
\end{frame}


\begin{frame}[fragile]{Execution time}

\begin{block}{Time measurements}
\begin{lstlisting}[language=bash]
g++ -std=c++1z -ltbb lecture7-loops.cpp 
./a.out
std::accumulate result 0.000000 took 8164.458818 ms
std::reduce result 0.000000 took 584.451218 ms
\end{lstlisting}


\end{block}


\end{frame}

%%%%%%%%%%%%%%%%%%%%%%%%%%%%%%%%%%%%%%%%%%%%%%%%%%%%%%%%%%%%%%%%%%%%%%%%%%%%%%
\section{Execution policies}
%%%%%%%%%%%%%%%%%%%%%%%%%%%%%%%%%%%%%%%%%%%%%%%%%%%%%%%%%%%%%%%%%%%%%%%%%%%%%%


\begin{frame}{Execution policies}

\begin{itemize}
\item \lstinline|std::execution::seq| \\
The algorithm is executed sequential, like \lstinline|std::accumulate| in the previous example and using only once thread.
\item \lstinline|std::execution::par| \\
The algorithm is executed in parallel and used multiple threads.
\item \lstinline| std::execution::par_unseq| \\
The algorithm is executed in parallel and vectorization is used.
\end{itemize}
Note we will not cover vectorization in this course. \\
\vspace{0.5cm}
Fore more details: CppCon 2016: Bryce Adelstein Lelbach “The C++17 Parallel Algorithms Library and Beyond"\footnote{\tiny{\url{https://www.youtube.com/watch?v=Vck6kzWjY88}}}
\end{frame}


%%%%%%%%%%%%%%%%%%%%%%%%%%%%%%%%%%%%%%%%%%%%%%%%%%%%%%%%%%%%%%%%%%%%%%%%%%%%%%
\section{Synchronization}
%%%%%%%%%%%%%%%%%%%%%%%%%%%%%%%%%%%%%%%%%%%%%%%%%%%%%%%%%%%%%%%%%%%%%%%%%%%%%%

\begin{frame}{}

\end{frame}



%%%%%%%%%%%%%%%%%%%%%%%%%%%%%%%%%%%%%%%%%%%%%%%%%%%%%%%%%%%%%%%%%%%%%%%%%%%%%%
\section{Summary}
%%%%%%%%%%%%%%%%%%%%%%%%%%%%%%%%%%%%%%%%%%%%%%%%%%%%%%%%%%%%%%%%%%%%%%%%%%%%%%
\begin{frame}{Summary}
\begin{block}{After this lecture, you should know}
\begin{itemize}
\item Shared memory parallelism
\item Parallel algorithms
\item Execution policies
\item Mutex
\end{itemize}
\end{block}
\end{frame}

%%%%%%%%%%%%%%%%%%%%%%%%%%%%%%%%%%%%%%%%%%%%%%%%%%%%%%%%%%%%%%%%%%%%%%%%%%%%%%
\section{References}
%%%%%%%%%%%%%%%%%%%%%%%%%%%%%%%%%%%%%%%%%%%%%%%%%%%%%%%%%%%%%%%%%%%%%%%%%%%%%%

\begin{frame}[t, allowframebreaks]
\frametitle{References}
\bibliographystyle{plain}
\bibliography{bib}
\end{frame}


\end{document}