%Template
%Copyright (C) 2019  Patrick Diehl
%
%This program is free software: you can redistribute it and/or modify
%it under the terms of the GNU General Public License as published by
%the Free Software Foundation, either version 3 of the License, or
%(at your option) any later version.

%This program is distributed in the hope that it will be useful,
%but WITHOUT ANY WARRANTY; without even the implied warranty of
%MERCHANTABILITY or FITNESS FOR A PARTICULAR PURPOSE.  See the
%GNU General Public License for more details.

%You should have received a copy of the GNU General Public License
%along with this program.  If not, see <http://www.gnu.org/licenses/>.
\documentclass[12pt,t]{beamer}

\beamertemplatenavigationsymbolsempty

% usepackage
%\usepackage{template/dbt}
\usepackage{listings}

\definecolor{comments}{RGB}{81,81,81}
\definecolor{keywords}{RGB}{255,0,90}

% lstlisting
\lstset{
    language=C,
    basicstyle=\footnotesize\ttfamily,
    keywordstyle=\color{keywords},
    showspaces=false,
    showstringspaces=false,
    commentstyle=\color{blue}\emph
    %frame=single,
    %rulecolor=\color{comments},
    %rulesepcolor=\color{comments},
    %backgroundcolor = \color{lightgray}
}

\usetheme{default}

\usepackage[
    type={CC},
    modifier={by-nc-nd},
    version={4.0},
]{doclicense} 

\newcommand{\courseurl}[0]{https://www.cct.lsu.edu/\string~pdiehl/teaching/2021/4997/}
\newcommand{\coursetimeline}[0]{https://www.cct.lsu.edu/~pdiehl/teaching/2021/4997/timeline.pdf}
\newcommand{\coursesyllabus}[0]{https://www.cct.lsu.edu/~pdiehl/teaching/2021/4997/syllabus.pdf}
\newcommand{\coursename}[0]{Math 4997-3}
\newcommand{\coursemailinglist}[0]{https://mail.cct.lsu.edu/mailman/listinfo/par4997}
\newcommand{\coursesemester}{Fall 2021}








% frame slide
\title{\coursename}
\subtitle{Lecture 7: Asynchronous programming }

%\author{\href{}{}}
%\institute {
%    \href{}{\tt \scriptsize \today}
%}
\date {
 \tiny \url{\courseurl}
\vspace{2cm}
\doclicenseThis  
  
}



\usepackage{ifxetex}

\ifxetex
\usepackage{fontspec}
\setmainfont{Raleway}
\fi

\begin{document} {
    \setbeamertemplate{footline}{}
    \frame {
        \titlepage
    }
}

\frame{

\tableofcontents

}

\AtBeginSection[]{
  \begin{frame}
  \vfill
  \centering
  \begin{beamercolorbox}[sep=8pt,center,shadow=true,rounded=true]{title}
    \usebeamerfont{title}\insertsectionhead\par%
  \end{beamercolorbox}
  \vfill
  \end{frame}
}

%%%%%%%%%%%%%%%%%%%%%%%%%%%%%%%%%%%%%%%%%%%%%%%%%%%%%%%%%%%%%%%%%%%%%%%%%%%%%%
\section{Reminder}
%%%%%%%%%%%%%%%%%%%%%%%%%%%%%%%%%%%%%%%%%%%%%%%%%%%%%%%%%%%%%%%%%%%%%%%%%%%%%%
\begin{frame}{Lecture 6}
\begin{block}{What you should know from last lecture}
\begin{itemize}
\item Shared memory parallelism
\item Parallel algorithms and execution policies
\item Data races and dead locks
\end{itemize}
\end{block}
\end{frame}


%%%%%%%%%%%%%%%%%%%%%%%%%%%%%%%%%%%%%%%%%%%%%%%%%%%%%%%%%%%%%%%%%%%%%%%%%%%%%%
\section{Asynchronous programming}
%%%%%%%%%%%%%%%%%%%%%%%%%%%%%%%%%%%%%%%%%%%%%%%%%%%%%%%%%%%%%%%%%%%%%%%%%%%%%%


%%%%%%%%%%%%%%%%%%%%%%%%%%%%%%%%%%%%%%%%%%%%%%%%%%%%%%%%%%%%%%%%%%%%%%%%%%%%%%
\section{Lambda functions}
%%%%%%%%%%%%%%%%%%%%%%%%%%%%%%%%%%%%%%%%%%%%%%%%%%%%%%%%%%%%%%%%%%%%%%%%%%%%%%

\begin{frame}[fragile]{Lambda expression\footnote{\tiny{\url{https://en.cppreference.com/w/cpp/language/lambda}}} }

\begin{block}{Structure}
\begin{lstlisting}
[ capture clause ] (parameters) -> return-type  
{   
   definition of method   
} 
\end{lstlisting}
\end{block}

\begin{block}{Notes}
\begin{itemize}
\item Generally return-type in lambda expression are evaluated by compiler
\item Capture clause:
\begin{itemize}
\item  \lstinline|[&]| : capture all external variable by reference
\item  \lstinline|[=]| : capture all external variable by value
\item  \lstinline|[a, &b]| : capture a by value and b by reference
\end{itemize}
\end{itemize}
\end{block}
\vspace{-0.2cm}
\begin{center}
More about the capture clauses in lecture 11/12.
\end{center}
\end{frame}

\begin{frame}[fragile]{Practical example}
\begin{lstlisting}
std::vector<int> v {4, 1, 3, 5, 2, 3, 1, 7}; 

\end{lstlisting}


\begin{block}{Classical function}
\begin{lstlisting}
void print(int i){
std::cout << i << std::endl;
}
std::for_each(v.begin(), v.end(), print); 
\end{lstlisting}
\end{block}

\begin{block}{Lambda expression}
\begin{lstlisting}
std::for_each(v.begin(),v.end(),
	[](int i){std::cout<< i << std::endl;})
\end{lstlisting}
\end{block}


\end{frame}
     
\begin{frame}[fragile]{More examples}

\begin{block}{\lstinline|find_if|\footnote{\tiny\url{https://en.cppreference.com/w/cpp/algorithm/find}}}
\begin{lstlisting}
    vector<int>:: iterator p = find_if(
    v.begin(), 
    	v.end(), 
    	[](int i) 
    { 
        return i > 4; 
    }); 
    cout << "First number greater than 4 is : 
    " << *p 
    << endl;
\end{lstlisting}
\end{block}
Many more algorithms are available in the \lstinline|#include <algorithm>|\footnote{\tiny\url{https://en.cppreference.com/w/cpp/algorithm}}
\end{frame}
     


%%%%%%%%%%%%%%%%%%%%%%%%%%%%%%%%%%%%%%%%%%%%%%%%%%%%%%%%%%%%%%%%%%%%%%%%%%%%%%
\section{Summary}
%%%%%%%%%%%%%%%%%%%%%%%%%%%%%%%%%%%%%%%%%%%%%%%%%%%%%%%%%%%%%%%%%%%%%%%%%%%%%%
\begin{frame}{Summary}
\begin{block}{After this lecture, you should know}
\begin{itemize}
\item Asynchronous programming \lstinline|std::async| and \lstinline|std::future|
\item Lambda functions
\end{itemize}
\end{block}
\end{frame}


\end{document}