%Template
%Copyright (C) 2019  Patrick Diehl
%
%This program is free software: you can redistribute it and/or modify
%it under the terms of the GNU General Public License as published by
%the Free Software Foundation, either version 3 of the License, or
%(at your option) any later version.

%This program is distributed in the hope that it will be useful,
%but WITHOUT ANY WARRANTY; without even the implied warranty of
%MERCHANTABILITY or FITNESS FOR A PARTICULAR PURPOSE.  See the
%GNU General Public License for more details.

%You should have received a copy of the GNU General Public License
%along with this program.  If not, see <http://www.gnu.org/licenses/>.
\providecommand\classoption{12pt}
\documentclass[\classoption]{beamer}

\newcommand{\orcid}[1]{\href{https://orcid.org/#1}{\includegraphics[height=10pt]{images/orcid}}}

\mode<beamer>{
\usepackage{template/dbt}
\setbeamercolor{mycolor}{fg=background,bg=background}
}

\mode<handout>{%
	\usepackage{handoutWithNotes} 
    \pgfpagesuselayout{4 on 1 with notes}[letterpaper] 
    \setbeameroption{show notes}
    \usetheme{default}
}

\beamertemplatenavigationsymbolsempty

\setbeamertemplate{bibliography item}{\insertbiblabel}

% usepackage
\usepackage{subcaption}

\definecolor{azure}{rgb}{0.0, 0.5, 1.0}
\definecolor{awesome}{rgb}{1.0, 0.13, 0.32}
\definecolor{cadetgrey}{rgb}{0.57, 0.64, 0.69}
\definecolor{asparagus}{rgb}{0.53, 0.66, 0.42}
\definecolor{atomictangerine}{rgb}{1.0, 0.6, 0.4}

\usepackage{lstautodedent}
% lstlisting
\lstset{
    language=C,
    basicstyle=\footnotesize\ttfamily,
    keywordstyle=\color{awesome},
    showspaces=false,
    showstringspaces=false,
    commentstyle=\color{azure}\emph,
	autodedent
    %frame=single,
    %rulecolor=\color{comments},
    %rulesepcolor=\color{comments},
    %backgroundcolor = \color{lightgray}
}

\lstset{inputpath=./ParallelComputationMathExamples/}

\usepackage[
    type={CC},
    modifier={by-nc-nd},
    version={4.0},
]{doclicense} 

\usepackage{ifxetex}

\ifxetex
\usepackage{fontspec}
\setmainfont{Raleway}
\fi

\ifluatex
\usepackage{fontspec}
\setmainfont{Raleway}
\fi

\usepackage{xfrac}

\usepackage{tikz}
\usepackage{pgfplots}


\newcommand{\courseurl}[0]{https://www.cct.lsu.edu/\string~pdiehl/teaching/2021/4997/}
\newcommand{\coursetimeline}[0]{https://www.cct.lsu.edu/~pdiehl/teaching/2021/4997/timeline.pdf}
\newcommand{\coursesyllabus}[0]{https://www.cct.lsu.edu/~pdiehl/teaching/2021/4997/syllabus.pdf}
\newcommand{\coursename}[0]{}
\newcommand{\coursemailinglist}[0]{https://mail.cct.lsu.edu/mailman/listinfo/par4997}
\newcommand{\coursesemester}{Fall 2023}
\newcommand{\courselocation}{TBA}
\newcommand{\coursedate}{Tuesday and Thursday, 09:00 to 10:20}
\newcommand{\courseserver}[0]{https://hpx-tutorial.cct.lsu.edu}
\newcommand{\coursetitle}[0]{Applications of high performance computing with C++ in biomedical science and engineering}




% frame slide
\title{\coursename}
\subtitle{Lecture 20: Managing memory and low-level data structures}

\author{\tiny Patrick Diehl \orcid{0000-0003-3922-8419}}
%\institute {
%    \href{}{\tt \scriptsize \today}
%}
\date {
 \tiny \url{\courseurl}
\vspace{2cm}
\doclicenseThis  
  
}



\usepackage{ifxetex}

\ifxetex
\usepackage{fontspec}
\setmainfont{Raleway}
\fi

\ifluatex
\usepackage{fontspec}
\setmainfont{Raleway}
\fi

\usepackage{tikz}

\begin{document} {
    \setbeamertemplate{footline}{}
    \frame {
        \titlepage
    }
}

\frame{

\tableofcontents

}

\AtBeginSection[]{
  \begin{frame}
  \vfill
  \centering
  \begin{beamercolorbox}[sep=8pt,center,shadow=true,rounded=true]{title}
    \usebeamerfont{title}\insertsectionhead\par%
  \end{beamercolorbox}
  \vfill
  \end{frame}
}

%%%%%%%%%%%%%%%%%%%%%%%%%%%%%%%%%%%%%%%%%%%%%%%%%%%%%%%%%%%%%%%%%%%%%%%%%%%%%%
\section{Reminder}
%%%%%%%%%%%%%%%%%%%%%%%%%%%%%%%%%%%%%%%%%%%%%%%%%%%%%%%%%%%%%%%%%%%%%%%%%%%%%%
\begin{frame}{Lecture 19}
\begin{block}{What you should know from last lecture}
\begin{itemize}
\item Running distributed HPX applications
\item Using the slurm environment and modules on clusters
\end{itemize}
\end{block}
\end{frame}


\begin{frame}{Why do we talk about pointers so late?}

\textbf{Person A}: Would you teach a toddler how to eat with a butcher's knife? \\
\vspace{1cm}
\textbf{Person B}: No? \\
\vspace{1cm}
\textbf{Person A}: So stop mentioning pointers to people barely starting with C++
\end{frame}


%%%%%%%%%%%%%%%%%%%%%%%%%%%%%%%%%%%%%%%%%%%%%%%%%%%%%%%%%%%%%%%%%%%%%%%%%%%%%%
\section{Pointer}
%%%%%%%%%%%%%%%%%%%%%%%%%%%%%%%%%%%%%%%%%%%%%%%%%%%%%%%%%%%%%%%%%%%%%%%%%%%%%%

\begin{frame}{Pointer}

\begin{center}
\begin{tikzpicture}
\draw (0,0) rectangle ++(0.75,0.75) node[pos=.5] {$p$};
\draw (1.5,0) rectangle ++(0.75,0.75) node[pos=.5] {$x$};
\draw[->] (0.75,0.365) -- (1.5,0.365);
\end{tikzpicture}

\end{center}
\begin{itemize}
\item A pointer $p$ is a value that represents the address of an object
\item Every object $x$ has a distinct unique address to a part of the computer's memory.
\end{itemize}

\begin{block}{Operators}
\begin{itemize}
\item The address of object $x$ can be accessed using the $\&$ address operator 
\item The deference operator * provides the object the pointer is pointing to
\end{itemize}
\end{block}

\end{frame}

\begin{frame}[fragile]{Example}

\begin{lstlisting}
// Initialize
int x = 42;

// Get the pointer to the object x
int* p = &x;

// Get the object the pointer is pointing to
int tmp = *p;

// Using pointers to manipulate objects
std::cout << x << std::endl;
*p = 42;
std::cout << x << std::endl;

\end{lstlisting}

\end{frame}

\begin{frame}[fragile]{Pointer arithmetic}

\begin{lstlisting}
int* array = new int[3];
*array = 1;
*(array + 1) = 2;
*(array + 2) = 3;

// Accessing the first element
int first = *array;

// Accessing the second element
int second = *(array + 1);

// Getting the distance between two pointers
ptrdiff_t dist = array+2 - array;
\end{lstlisting}

Note that \lstinline|ptrdiff_t| is a signed type because the distance can be negative
\end{frame}


\begin{frame}[fragile]{Pointers to functions}

\begin{lstlisting}
int square(int a)
{
return a * a;
}

// Generating a function pointer
int (*fp)(int) = square; //We need the (int) for
int (*fp2)(int) = &square; // the return type

// Calling the function using its pointer
std::cout << (*fp)(5);
std::cout << fp2(5);

\end{lstlisting}
Note that each of two lines to get the pointer or call the function are equivalent. 

\end{frame}

%%%%%%%%%%%%%%%%%%%%%%%%%%%%%%%%%%%%%%%%%%%%%%%%%%%%%%%%%%%%%%%%%%%%%%%%%%%%%%
\section{Arrays}
%%%%%%%%%%%%%%%%%%%%%%%%%%%%%%%%%%%%%%%%%%%%%%%%%%%%%%%%%%%%%%%%%%%%%%%%%%%%%%

\begin{frame}{Array vs Vector}


\begin{block}{Array\footnote{\tiny\url{https://en.cppreference.com/w/cpp/container/array}}\footnote{\tiny\url{https://en.cppreference.com/w/cpp/language/array}}}
\begin{itemize}
\item Language feature
\item Number of elements must be known at compile time
\item Can not grow or shrink dynamically
\end{itemize}
\end{block}

\begin{block}{Vector}
\begin{itemize}
\item Part of the standard library
\item Can grow or shrink dynamically
\end{itemize}
\end{block}
An array is a kind of container, similar to a vector but less powerful.
\end{frame}

\begin{frame}[fragile]{Working with arrays}

\begin{lstlisting}
//Define the length
size_t size = 6;

//Generate a double array of size 6
double array[size];

//Access all elements
for(size_t i = 0; i < size ; i++){
        array[i] = i;
        std::cout << array[i] << std::endl;
    }

//Access the first element
*array = 42;
std::cout << array[0] << std::endl;

//Initializing 
double array = {1,2,3.5,5};
\end{lstlisting}

\end{frame}



%%%%%%%%%%%%%%%%%%%%%%%%%%%%%%%%%%%%%%%%%%%%%%%%%%%%%%%%%%%%%%%%%%%%%%%%%%%%%%
\section{Command line arguments}
%%%%%%%%%%%%%%%%%%%%%%%%%%%%%%%%%%%%%%%%%%%%%%%%%%%%%%%%%%%%%%%%%%%%%%%%%%%%%%

\begin{frame}[fragile]{Arguments to \lstinline|main|}

\begin{lstlisting}
int main(int argc, char** argv)
{

	return EXIT_SUCCESS:
}
\end{lstlisting}

\begin{block}{Parameters}
\begin{itemize}
\item \lstinline|int argc| -- Number of pointers in the \lstinline|char** argv|
\item \lstinline|char** argv| -- Initial pointer to an array of pointers for each command line option. Note that the first entry is the name of the executable.
\end{itemize}
\end{block}
Note that these parameters can have any name, but the two presented are very common.
\end{frame}


\begin{frame}[fragile]{Example}

\lstinputlisting{chapter2/lecture20-argument.cpp}

\end{frame}

%%%%%%%%%%%%%%%%%%%%%%%%%%%%%%%%%%%%%%%%%%%%%%%%%%%%%%%%%%%%%%%%%%%%%%%%%%%%%%
\section{Memory management}
%%%%%%%%%%%%%%%%%%%%%%%%%%%%%%%%%%%%%%%%%%%%%%%%%%%%%%%%%%%%%%%%%%%%%%%%%%%%%%


\begin{frame}{Three kind of memory management}

\begin{block}{Automatic memory management}
\begin{itemize}
\item What we have done so far
\item The system is allocating the memory for a local variable
\item The system is deallocating the memory if the variable goes out of scope
\end{itemize}
\end{block}

\begin{block}{Dynamic memory management}
\begin{itemize}
\item The programmer allocates the memory with the the \lstinline|new|\footnote{\tiny\url{https://en.cppreference.com/w/cpp/language/new}} keyword
\item The programmer deallocates the memory with the the \lstinline|delete|\footnote{\tiny\url{https://en.cppreference.com/w/cpp/language/delete}} keyword
\end{itemize}
\end{block}

\end{frame}

\begin{frame}[fragile]{Memory management of an object}

\begin{block}{Allocation}
\begin{lstlisting}
int* p = new int(42);
\end{lstlisting}
\end{block}

\begin{block}{Deallocation}
\begin{lstlisting}
delete p;
\end{lstlisting}
\end{block}


\end{frame}

\begin{frame}[fragile]{Memory management of an array}

\begin{block}{Allocation}
\begin{lstlisting}
int* p = new int[5];
\end{lstlisting}
\end{block}

\begin{block}{Deallocation}
\begin{lstlisting}
delete[] p;
\end{lstlisting}
\end{block}


\end{frame}

%%%%%%%%%%%%%%%%%%%%%%%%%%%%%%%%%%%%%%%%%%%%%%%%%%%%%%%%%%%%%%%%%%%%%%%%%%%%%%
\section{Summary}
%%%%%%%%%%%%%%%%%%%%%%%%%%%%%%%%%%%%%%%%%%%%%%%%%%%%%%%%%%%%%%%%%%%%%%%%%%%%%%
\begin{frame}{Summary}
\begin{block}{After this lecture, you should know}
\begin{itemize}
\item Pointer and Arrays
\item How to read command line arguments
\item Allocating and deallocating memory
\end{itemize}
\end{block}
\end{frame}


\end{document}