%Template
%Copyright (C) 2019  Patrick Diehl
%
%This program is free software: you can redistribute it and/or modify
%it under the terms of the GNU General Public License as published by
%the Free Software Foundation, either version 3 of the License, or
%(at your option) any later version.

%This program is distributed in the hope that it will be useful,
%but WITHOUT ANY WARRANTY; without even the implied warranty of
%MERCHANTABILITY or FITNESS FOR A PARTICULAR PURPOSE.  See the
%GNU General Public License for more details.

%You should have received a copy of the GNU General Public License
%along with this program.  If not, see <http://www.gnu.org/licenses/>.
\documentclass[12pt,t]{beamer}

\beamertemplatenavigationsymbolsempty

\usepackage{pgfplots}
\pgfplotsset{compat=1.8}

% usepackage
%\usepackage{template/dbt}
\usepackage{listings}

\definecolor{comments}{RGB}{81,81,81}
\definecolor{keywords}{RGB}{255,0,90}

% lstlisting
\lstset{
    language=C,
    basicstyle=\footnotesize\ttfamily,
    keywordstyle=\color{keywords},
    showspaces=false,
    showstringspaces=false,
    commentstyle=\color{blue}\emph
    %frame=single,
    %rulecolor=\color{comments},
    %rulesepcolor=\color{comments},
    %backgroundcolor = \color{lightgray}
}

\usetheme{default}

\usepackage[
    type={CC},
    modifier={by-nc-nd},
    version={4.0},
]{doclicense} 

\newcommand{\courseurl}[0]{https://www.cct.lsu.edu/\string~pdiehl/teaching/2021/4997/}
\newcommand{\coursetimeline}[0]{https://www.cct.lsu.edu/~pdiehl/teaching/2021/4997/timeline.pdf}
\newcommand{\coursesyllabus}[0]{https://www.cct.lsu.edu/~pdiehl/teaching/2021/4997/syllabus.pdf}
\newcommand{\coursename}[0]{Math 4997-3}
\newcommand{\coursemailinglist}[0]{https://mail.cct.lsu.edu/mailman/listinfo/par4997}
\newcommand{\coursesemester}{Fall 2021}








% frame slide
\title{\coursename}
\subtitle{Lecture 13: Futurization of the 1D heat equation}

%\author{\href{}{}}
%\institute {
%    \href{}{\tt \scriptsize \today}
%}
\date {
 \tiny \url{\courseurl}
\vspace{2cm}
\doclicenseThis  
  
}



\usepackage{ifxetex}

\ifxetex
\usepackage{fontspec}
\setmainfont{Raleway}
\fi

\ifluatex
\usepackage{fontspec}
\setmainfont{Raleway}
\fi


\begin{document} {
    \setbeamertemplate{footline}{}
    \frame {
        \titlepage
    }
}

\frame{

\tableofcontents

}

\AtBeginSection[]{
  \begin{frame}
  \vfill
  \centering
  \begin{beamercolorbox}[sep=8pt,center,shadow=true,rounded=true]{title}
    \usebeamerfont{title}\insertsectionhead\par%
  \end{beamercolorbox}
  \vfill
  \end{frame}
}

%%%%%%%%%%%%%%%%%%%%%%%%%%%%%%%%%%%%%%%%%%%%%%%%%%%%%%%%%%%%%%%%%%%%%%%%%%%%%%
\section{Reminder}
%%%%%%%%%%%%%%%%%%%%%%%%%%%%%%%%%%%%%%%%%%%%%%%%%%%%%%%%%%%%%%%%%%%%%%%%%%%%%%
\begin{frame}{Lecture X}
\begin{block}{What you should know from last lecture}
\begin{itemize}
\item One-dimensional heat equation 
\item Serial implementation of the one-dimensional heat equation
\end{itemize}
\end{block}
\end{frame}

%%%%%%%%%%%%%%%%%%%%%%%%%%%%%%%%%%%%%%%%%%%%%%%%%%%%%%%%%%%%%%%%%%%%%%%%%%%%%%
\section{HPX features}
%%%%%%%%%%%%%%%%%%%%%%%%%%%%%%%%%%%%%%%%%%%%%%%%%%%%%%%%%%%%%%%%%%%%%%%%%%%%%%

\begin{frame}[fragile]{A ready future}
Some times, we need a future, which is already ready, since there is no computation needed.


\begin{lstlisting}
hpx::lcos::future<double> f 
   = hpx::make_ready_future(1);
\end{lstlisting}

\begin{block}{Example}
\begin{lstlisting}
auto f = hpx::make_ready_future(1);
/* 
 * Since the future is ready the output will happen
 * and there will be no barrier.
 */
std::cout << f.get() << std::endl;
\end{lstlisting}
\end{block}

\end{frame}


\begin{frame}[fragile]{Data flow I}
\begin{lstlisting}
std::vector<hpx::lcos::future<int>> futures;
futures.push_back(hpx::async(square,10));
futures.push_back(hpx::async(square,100));

// When all returns a future containing the vector 
// of futures
hpx::when_all(futures).then([](auto&& f){
    // We need to unwrap this future to get 
    // the content of it
    auto futures = f.get();
    int result = 0;
    for(size_t i = 0; i < futures.size();i++)
        result += futures[i].get();
    std::cout << result << std::endl;
});
\end{lstlisting}
\end{frame}


\begin{frame}[fragile]{Data flow II}

\begin{lstlisting}
hpx::dataflow(hpx::launch::sync,[](auto f){
    int result = 0;
    for(size_t i = 0; i < f.size();i++)
        result += f[i].get();
    std::cout << result << std::endl;
},futures);
\end{lstlisting}

\begin{block}{Parameters}
\begin{enumerate}
\item \lstinline|hpx::launch::async| or \lstinline|hpx::launch::sync|
\item The function to call
\item Futures to the arguments to the arguments of the function
\end{enumerate}
\end{block}


\end{frame}


\begin{frame}[fragile]{Passing futures}

\begin{lstlisting}
void sum(int first, int second){

std:: cout << first + second << std::endl;

}

auto f1 = hpx::async(square,10);
auto f2 = hpx::async(square,100);

// We have to call .get() to pass 
// the values of the future
sum(f1.get(),f2.get());
\end{lstlisting}

\end{frame}

\begin{frame}[fragile]{Unwrapping futures}

\begin{lstlisting}
void sum(int first, int second){

std:: cout << first + second << std::endl;

}

// We can unwrapp the function
auto fp = hpx::util::unwrapping(sum);

// After unwrapping, we can pass the future
// directly to the function
hpx::dataflow(hpx::launch::sync,fp,f1,f2);

\end{lstlisting}

\end{frame}


\begin{frame}[fragile]{Shared future}

\begin{columns}
\begin{column}{0.5\textwidth}
\begin{lstlisting}
 hpx::lcos::future
\end{lstlisting}
\begin{itemize}
\item Exclusive ownership model
\item If the future is out of the scope, it will be not available anymore.
\end{itemize}
\end{column}
\begin{column}{0.5\textwidth}  
    \begin{lstlisting}
 hpx::shared_future
\end{lstlisting}
\begin{itemize}
\item Reference counting ownership model
\item All references to the object are counted and the object is only destroyed if there are zero references.
\end{itemize}
\end{column}
\end{columns}
\vspace{.5cm}
\begin{center}
Can be seen to be equal to \lstinline|std::unique_ptr| and \lstinline|std::shared_ptr|.
\end{center}
\end{frame}

%%%%%%%%%%%%%%%%%%%%%%%%%%%%%%%%%%%%%%%%%%%%%%%%%%%%%%%%%%%%%%%%%%%%%%%%%%%%%%
\section{Scaling results}
%%%%%%%%%%%%%%%%%%%%%%%%%%%%%%%%%%%%%%%%%%%%%%%%%%%%%%%%%%%%%%%%%%%%%%%%%%%%%%

\begin{frame}{Overhead}

\begin{center}
\begin{tikzpicture}
\selectcolormodel{gray}
\begin{axis}[xlabel=Grid points,ylabel=Execution time, grid=both,title=1 CPU]
\addplot table [x=Grid_Points, y=Execution_Time_sec, col sep=comma] {./data/stencil_serial.dat};
\addlegendentry{Serial}
\addplot table [x=Grid_Points, y=Execution_Time_sec, col sep=comma] {./data/stencil_2_1.dat};
\addlegendentry{Futurization}
\end{axis}
\end{tikzpicture}
\end{center}
\end{frame}


\begin{frame}{Scaling}

\begin{center}
\begin{tikzpicture}
\selectcolormodel{gray}
\begin{axis}[xlabel=Grid points,ylabel=Execution time, grid=both,title= Stencil 2,legend pos=north west]
\addplot table [x=Grid_Points, y=Execution_Time_sec, col sep=comma] {./data/stencil_2_1.dat};
\addlegendentry{1 CPU}
\addplot table [x=Grid_Points, y=Execution_Time_sec, col sep=comma] {./data/stencil_2_2.dat};
\addlegendentry{2 CPU}
\addplot table [x=Grid_Points, y=Execution_Time_sec, col sep=comma] {./data/stencil_2_4.dat};
\addlegendentry{4 CPU}
\addplot table [x=Grid_Points, y=Execution_Time_sec, col sep=comma] {./data/stencil_2_6.dat};
\addlegendentry{6 CPU}
\end{axis}
\end{tikzpicture}
\end{center}
\end{frame}






%%%%%%%%%%%%%%%%%%%%%%%%%%%%%%%%%%%%%%%%%%%%%%%%%%%%%%%%%%%%%%%%%%%%%%%%%%%%%%
\section{Summary}
%%%%%%%%%%%%%%%%%%%%%%%%%%%%%%%%%%%%%%%%%%%%%%%%%%%%%%%%%%%%%%%%%%%%%%%%%%%%%%
\begin{frame}{Summary}
\begin{block}{After this lecture, you should know}
\begin{itemize}
\item \lstinline|hpx::make_ready_future|
\item \lstinline|hpx::dataflow|
\item \lstinline|hpx::util::unwrapping|
\item \lstinline|hpx::shared_future|
\end{itemize}
\end{block}
\end{frame}


\end{document}