%Template
%Copyright (C) 2019  Patrick Diehl
%
%This program is free software: you can redistribute it and/or modify
%it under the terms of the GNU General Public License as published by
%the Free Software Foundation, either version 3 of the License, or
%(at your option) any later version.

%This program is distributed in the hope that it will be useful,
%but WITHOUT ANY WARRANTY; without even the implied warranty of
%MERCHANTABILITY or FITNESS FOR A PARTICULAR PURPOSE.  See the
%GNU General Public License for more details.

%You should have received a copy of the GNU General Public License
%along with this program.  If not, see <http://www.gnu.org/licenses/>.
\documentclass[12pt,t]{beamer}

\beamertemplatenavigationsymbolsempty

% usepackage
%\usepackage{template/dbt}
\usepackage{listings}

\definecolor{comments}{RGB}{81,81,81}
\definecolor{keywords}{RGB}{255,0,90}

% lstlisting
\lstset{
    language=C,
    basicstyle=\footnotesize\ttfamily,
    keywordstyle=\color{keywords},
    showspaces=false,
    showstringspaces=false,
    commentstyle=\color{blue}\emph
    %frame=single,
    %rulecolor=\color{comments},
    %rulesepcolor=\color{comments},
    %backgroundcolor = \color{lightgray}
}

\usetheme{default}

\usepackage[
    type={CC},
    modifier={by-nc-nd},
    version={4.0},
]{doclicense} 

\usepackage{tikz}

\newcommand{\courseurl}[0]{https://www.cct.lsu.edu/\string~pdiehl/teaching/2021/4997/}
\newcommand{\coursetimeline}[0]{https://www.cct.lsu.edu/~pdiehl/teaching/2021/4997/timeline.pdf}
\newcommand{\coursesyllabus}[0]{https://www.cct.lsu.edu/~pdiehl/teaching/2021/4997/syllabus.pdf}
\newcommand{\coursename}[0]{Math 4997-3}
\newcommand{\coursemailinglist}[0]{https://mail.cct.lsu.edu/mailman/listinfo/par4997}
\newcommand{\coursesemester}{Fall 2021}








% frame slide
\title{\coursename}
\subtitle{Lecture 16: Preparation for distributed computing }

%\author{\href{}{}}
%\institute {
%    \href{}{\tt \scriptsize \today}
%}
\date {
 \tiny \url{\courseurl}
\vspace{2cm}
\doclicenseThis  
  
}



\usepackage{ifxetex}

\ifxetex
\usepackage{fontspec}
\setmainfont{Raleway}
\fi

\ifluatex
\usepackage{fontspec}
\setmainfont{Raleway}
\fi


\begin{document} {
    \setbeamertemplate{footline}{}
    \frame {
        \titlepage
    }
}

\frame{

\tableofcontents

}

\AtBeginSection[]{
  \begin{frame}
  \vfill
  \centering
  \begin{beamercolorbox}[sep=8pt,center,shadow=true,rounded=true]{title}
    \usebeamerfont{title}\insertsectionhead\par%
  \end{beamercolorbox}
  \vfill
  \end{frame}
}

%%%%%%%%%%%%%%%%%%%%%%%%%%%%%%%%%%%%%%%%%%%%%%%%%%%%%%%%%%%%%%%%%%%%%%%%%%%%%%
\section{Reminder}
%%%%%%%%%%%%%%%%%%%%%%%%%%%%%%%%%%%%%%%%%%%%%%%%%%%%%%%%%%%%%%%%%%%%%%%%%%%%%%
\begin{frame}{Lecture 15}
\begin{block}{What you should know from last lecture}
\begin{itemize}
\item Parallel partition-based implementation
\item Allocating and deallocating memory with the \lstinline|new| and \lstinline|delete| expression
\end{itemize}
\end{block}
\end{frame}

%%%%%%%%%%%%%%%%%%%%%%%%%%%%%%%%%%%%%%%%%%%%%%%%%%%%%%%%%%%%%%%%%%%%%%%%%%%%%%
\section{Serialization}
%%%%%%%%%%%%%%%%%%%%%%%%%%%%%%%%%%%%%%%%%%%%%%%%%%%%%%%%%%%%%%%%%%%%%%%%%%%%%%

\begin{frame}{What is this serialization thingy}

If we want to send ab object or a bunch of objects over the network to another node of the cluster:

\begin{itemize}
\item The sender needs to flatten the object(s) to a one-dimensional stream of bits.
\item The receiver needs to unflatten the one-dimensional stream of bits and convert it back to the objects(s)
\end{itemize}

You have to decide whether to serialize
\begin{itemize}
\item human-readable (text)
\item non-human-readable (binary)
\end{itemize}
\end{frame}

\begin{frame}[fragile]{Serialization in HPX~\cite{ac:2017}}

\begin{block}{Initialize data}
\begin{lstlisting}
size_t size = 5;
double* data = new double[size];
\end{lstlisting}
\end{block}

\begin{block}{Serialization}
\begin{lstlisting}
using hpx::serialization::serialize_buffer;

serialize_buffer<double> serializable_data(
     data, size,
     serialize_buffer<double>::init_mode::reference);
\end{lstlisting}
\end{block}

\begin{block}{Deserialization}
\begin{lstlisting}
double* copied_data = serializable_data.data();
\end{lstlisting}
\end{block}

\end{frame}

\begin{frame}{Sending data over the network}

\begin{center}
\begin{tikzpicture}
\draw (0,0) rectangle ++(3,-0.75) node[pos=.5] {Allocate data};

\draw (0,-1.25) rectangle ++(3,-.75) node[pos=.5] {Serialize data};

\draw (0,-2.5) rectangle ++(3,-.75) node[pos=.5] {Send parcel};

\draw (-0.1,0.1) rectangle ++(3.2,-3.75);

\node[above] at (1.5,0) {Locality 1};

\draw (6,0) rectangle ++(3,-0.75) node[pos=.5] {Deserialize data};

\draw (6,-1.25) rectangle ++(3,-.75) node[pos=.5] {Allocate data};

\draw (6,-2.5) rectangle ++(3,-.75) node[pos=.5] {Receive parcel};

\draw (5.9,0.1) rectangle ++(3.2,-3.75);

\node[above] at (7.5,0) {Locality 2};

\draw (1.5,-0.75) -- (1.5,-1.23) ;
\draw (1.5,-2.) -- (1.5,-2.5) ;

\draw (3,-2.875) -- (6,-2.875) ;

\draw (7.5,-0.75) -- (7.5,-1.23) ;
\draw (7.5,-2.) -- (7.5,-2.5) ;

\end{tikzpicture}
\end{center}
The communication between the localities (nodes) is handled by the so-called parcel port~\cite{kaiser2009parallex}. HPX uses MPI or libfrabric for communication between nodes. 
\end{frame}


%%%%%%%%%%%%%%%%%%%%%%%%%%%%%%%%%%%%%%%%%%%%%%%%%%%%%%%%%%%%%%%%%%%%%%%%%%%%%%
\section{Components}
%%%%%%%%%%%%%%%%%%%%%%%%%%%%%%%%%%%%%%%%%%%%%%%%%%%%%%%%%%%%%%%%%%%%%%%%%%%%%%

\begin{frame}{Components and Actions}
For distributed computations within HPX, we need to look into following 

\begin{enumerate}
\item Server (Component) \\
The server represents the global data and is a so-called HPX component which allows to create and access the data remotely through a global address space (AGAS~\cite{kaiser2014hpx}) using \lstinline|hpx::id_type|.
\item Component action \\
Each function of the component needs to be wrapped into a component action to be remotely available. 
\item Plain actions \\
Allows to wrap global (\lstinline|static|) functions in an action. So this function can be called remotely on a given locality.
\end{enumerate} 

\end{frame}


\begin{frame}[fragile]{Server (Component)}
\begin{lstlisting}
struct data_server
  : hpx::components::component_base<data_server>{

data_server(size_t size)
{
data = std::shared_ptr<double[]>(new double[size]);
}

private:

// This data will be in the global address space 
std::shared_ptr<double[]> data;

};
\end{lstlisting}
Note that we need component actions to call the public functions of the server.
\end{frame}


\begin{frame}[fragile]{Component actions}
\begin{lstlisting}
struct data_server
  : hpx::components::component_base<data_server>{

// This function needs to be wrapped in an action
double* getData(){
    return data.get();
}
// Define the action for remote access
HPX_DEFINE_COMPONENT_DIRECT_ACTION(data_server,
                     getData, getData_action);

private:

// This data will be in the global address space 
std::shared_ptr<double[]> data;

};
\end{lstlisting}
\end{frame}

\begin{frame}[fragile]{Component actions}

\begin{lstlisting}
// Define the action for remote access
HPX_DEFINE_COMPONENT_DIRECT_ACTION(
                     data_server,
                     getData,
                     getData_action);
\end{lstlisting}
where
\begin{itemize}
\item The first argument is the name of the component
\item The second argument is the name of the function
\item The third argument is the name of the action
\end{itemize}
\end{frame}

\begin{frame}[fragile]{Registering components and actions}

\begin{lstlisting}
// Generation of the code, which is needed to call 
// the component action remotely

// Registering the component
typedef hpx::components::component<data_server> 
    data_server_type;
HPX_REGISTER_COMPONENT(data_server_type, 
    data_server);
    
// Registering the component actions
typedef data_server::getData_action getData_action;
HPX_REGISTER_ACTION(getData_action);
\end{lstlisting}

\end{frame}



\begin{frame}[fragile]{Global functions}

\begin{block}{Function definition}
\begin{lstlisting}
static void square(double a ){

std::cout << a * a << std::endl;

}
\end{lstlisting}
\end{block}

\begin{block}{Define the action}
\begin{lstlisting}
HPX_PLAIN_ACTION(square, square_action);
\end{lstlisting}
where
\begin{itemize}
\item The first argument is the name of the function
\item The second argument is the name of the action
\end{itemize}

\end{block}
\end{frame}

\begin{frame}[fragile]{Knowing where you are}

\begin{block}{Getting the ID of your locality}
\begin{lstlisting}
hpx::find_here();
\end{lstlisting}
\end{block}

\begin{block}{Getting the ID of the component's locality}
\begin{lstlisting}
hpx::get_colocation_id(hpx::launch::sync, get_id();
\end{lstlisting}
\end{block}

\begin{block}{Getting the ID of the component's locality}
\begin{lstlisting}
bool is_local 
    = (hpx::get_colocation_id(hpx::launch::sync, 
          get_id()) == hpx::find_here());
\end{lstlisting}
\end{block}

\end{frame}

%%%%%%%%%%%%%%%%%%%%%%%%%%%%%%%%%%%%%%%%%%%%%%%%%%%%%%%%%%%%%%%%%%%%%%%%%%%%%%
\section{Summary}
%%%%%%%%%%%%%%%%%%%%%%%%%%%%%%%%%%%%%%%%%%%%%%%%%%%%%%%%%%%%%%%%%%%%%%%%%%%%%%
\begin{frame}{Summary}
\begin{block}{After this lecture, you should know}
\begin{itemize}
\item Serialization
\item Distributed computing
\begin{itemize}
\item Plain actions
\item Components
\item Components actions
\end{itemize}
\end{itemize}
\end{block}
\end{frame}

%%%%%%%%%%%%%%%%%%%%%%%%%%%%%%%%%%%%%%%%%%%%%%%%%%%%%%%%%%%%%%%%%%%%%%%%%%%%%%
\section{References}
%%%%%%%%%%%%%%%%%%%%%%%%%%%%%%%%%%%%%%%%%%%%%%%%%%%%%%%%%%%%%%%%%%%%%%%%%%%%%%

\begin{frame}[t, allowframebreaks]
\frametitle{References}
\bibliographystyle{plain}
\bibliography{bib}
\end{frame}



\end{document}