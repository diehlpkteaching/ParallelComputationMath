%Poster
%Copyright (C) 2019  Patrick Diehl
%
%This program is free software: you can redistribute it and/or modify
%it under the terms of the GNU General Public License as published by
%the Free Software Foundation, either version 3 of the License, or
%(at your option) any later version.

%This program is distributed in the hope that it will be useful,
%but WITHOUT ANY WARRANTY; without even the implied warranty of
%MERCHANTABILITY or FITNESS FOR A PARTICULAR PURPOSE.  See the
%GNU General Public License for more details.

%You should have received a copy of the GNU General Public License
%along with this program.  If not, see <http://www.gnu.org/licenses/>.
\documentclass[25pt, letter, portrait]{tikzposter}
\usepackage[utf8]{inputenc}
\usepackage{qrcode}
 
\title{\coursename~Parallel computational mathematics}
\author{Fall 2020}
%\date{\today}
%\institute{ShareLaTeX Institute}

\newcommand{\courseurl}[0]{https://www.cct.lsu.edu/\string~pdiehl/teaching/2021/4997/}
\newcommand{\coursetimeline}[0]{https://www.cct.lsu.edu/~pdiehl/teaching/2021/4997/timeline.pdf}
\newcommand{\coursesyllabus}[0]{https://www.cct.lsu.edu/~pdiehl/teaching/2021/4997/syllabus.pdf}
\newcommand{\coursename}[0]{Math 4997-3}
\newcommand{\coursemailinglist}[0]{https://mail.cct.lsu.edu/mailman/listinfo/par4997}
\newcommand{\coursesemester}{Fall 2021}









\usetheme{simple}

\usepackage{ifxetex}

\ifxetex
\usepackage{fontspec}
\setmainfont{Raleway}
\fi

\usepackage[
    type={CC},
    modifier={by-nc-nd},
    version={4.0},
]{doclicense}

\colorlet{backgroundcolor}{white}
\colorlet{framecolor}{black}
\colorlet{blocktitlebgcolor}{white}
\colorlet{blocktitlefgcolor}{black}
\colorlet{blockbodybgcolor}{white}
\colorlet{blockbodyfgcolor}{black}
\begin{document}
 
\maketitle


\block{General Information}{
 \centering
	\begin{tabular}{ll}
	\textbf{Time}: & Tuesday and Thursday, 09:00 to 10:20 \\   
	\textbf{Location}: & 130 LCKT  \\
    \textbf{Instructor}: & Dr. Patrick Diehl \\
    \textbf{Contact}: & patrickdiehl@lsu.edu	
	\end{tabular}	
}

\begin{columns}
\column{0.1}
\column{0.8}
\block{Course description}{
This course will focus on the parallel implementation of computational mathematics problems using modern accelerated C++. The aim of this course is to learn how to quickly write useful efficient C++ programs. The students will not learn low-level C/C++ instead they will learn how to use high-level data structures, iterators, generic strings, and streams (including interactive and file I/O) of the C++ ISO Standard library. In addition, highly-optimized linear algebra libraries are introduced since the course teaches to solve problems, instead of explaining low-level C++and computer science algorithms, like sorting algorithms, which are provided in the C++ standard library.\\


The first part, provides a brief overview of the containers, strings, streams, input/output, and the numeric library of the C++ standard library. For linear algebra, we will look into Blaze which is an open-source, high-performance C++ math library for dense and sparse arithmetic.The second part will solve computational mathematics problems based-on the the previous introduced features of the C++ standard library.The third part will focus on the parallel features provided by the C++ standard library. Here,the implemented computational problems in the second part of the course will be parallized using the C++ standard library for parallelism and concurrency.Since programming skills can only be improved by doing, there will be weekly programming exercises and a small project. After this course students have a basic overview of the C++ standard library to solve efficiently computational mathematics problems without using low-level C/C++.
}
\column{0.1}
\end{columns}

\block{Textbooks}{
\centering
\begin{tabular}{l}
 Andrew, Koenig. Accelerated C++: practical programming by example. Pearson Education India, 2000.  \\
 Stroustrup, Bjarne. Programming: principles and practice using C++. Pearson Education, 2014.
\end{tabular}
}


\block{Content}{
\centering
\begin{tabular}{l}
Accelerated/Modern C++ \\ 
Modern C++ implementation of computational mathematics problems \\
Parallel computation using the C++ standard library for parallelism and concurrency 
\end{tabular}
}

\block{Grading}{

\centering
\begin{tabular}{l}
Quizzes and homework 30\% \\ 
Project 20\% \\
Midterm exam 20\% \\
Final exam 30\%
\end{tabular}
}

\block{}
{
\centering
\qrcode[hyperlink,height=8cm]{\courseurl}
    
}

\block{}
{
\centering
\doclicenseImage[imagewidth=8cm] 
    
}

\end{document}

