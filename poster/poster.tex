\documentclass[25pt, letter, portrait]{tikzposter}
\usepackage[utf8]{inputenc}
 
\title{Math 4977-3 Parallel computational mathematics}
\author{Fall 2019}
%\date{\today}
%\institute{ShareLaTeX Institute}
 
\usetheme{simple}

 
\begin{document}
 
\maketitle

\begin{columns}
    \column{0.5}
    \block{}{Text and more text}
 
    \column{0.5}
    \block{General information}{
    \centering
	\begin{tabular}{ll}
	\textbf{Time}: & Tuesday and Thursday, 09:00 to 10:20 \\   
	\textbf{Location}: & 130 LCKT  \\
    \textbf{Instructor}: & Dr. Patrick Diehl \\
    \textbf{Contact}: & patrickdiehl@lsu.edu	
	\end{tabular}	    
    }
 
\end{columns}


\block{Course description}{
This course will focus on the parallel implementation of computational mathematics problems using modern accelerated C++. The aim of this course is to learn how to quickly write useful efficient C++ programs. The students will not learn low-level C/C++ instead they will learn how to use high-level data structures, iterators, generic strings, and streams (including interactive and file I/O) of the C++ ISO Standard library. In addition, highly-optimized linear algebra libraries are introduced since the course teaches to solve problems, instead of explaining low-level C++and computer science algorithms, like sorting algorithms, which are provided in the C++ standard library.\\


The first part, provides a brief overview of the containers, strings, streams, input/output, and the numeric library of the C++ standard library. For linear algebra, we will look into Blaze which is an open-source, high-performance C++ math library for dense and sparse arithmetic.The second part will solve computational mathematics problems based-on the the previous introduced features of the C++ standard library.The third part will focus on the parallel features provided by the C++ standard library. Here,the implemented computational problems in the second part of the course will be parallized using the C++ standard library for parallelism and concurrency.Since programming skills can only be improved by doing, there will be weekly programming exercises and a small project. After this course students have a basic overview of the C++ standard library to solve efficiently computational mathematics problems without using low-level C/C++.
}

\block{Textbooks}{
\centering
\begin{tabular}{l}
 Andrew, Koenig. Accelerated C++: practical programming by example. Pearson Education India, 2000.  \\
 Stroustrup, Bjarne. Programming: principles and practice using C++. Pearson Education, 2014.
\end{tabular}
}

\block{Grading}{

\begin{minipage}{0.35\textwidth}
\hspace{2cm}
\begin{itemize}
\item Quizzes and homework 30\%
\item Project 20\%
\item Midterm exam 20\%
\item Final exam 30\%
\end{itemize}
\end{minipage}
\begin{minipage}{0.5\textwidth}
Overall, in the end of the semester 90\% of all points or more will give you an ‘A’, 80\% or more a ‘B’, 70\% or more a ‘C’, and 60\% or more results in a ‘D’. Below that you’ll fail the course, but I’m sure that will not happen to anyone
\end{minipage}
}

\block{}
{
        \begin{tikzfigure}
            \includegraphics[width=0.2\textwidth]{qrcode2019.png}
        \end{tikzfigure}
}


\end{document}

