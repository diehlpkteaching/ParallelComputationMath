%Template
%Copyright (C) 2019  Patrick Diehl
%
%This program is free software: you can redistribute it and/or modify
%it under the terms of the GNU General Public License as published by
%the Free Software Foundation, either version 3 of the License, or
%(at your option) any later version.

%This program is distributed in the hope that it will be useful,
%but WITHOUT ANY WARRANTY; without even the implied warranty of
%MERCHANTABILITY or FITNESS FOR A PARTICULAR PURPOSE.  See the
%GNU General Public License for more details.

%You should have received a copy of the GNU General Public License
%along with this program.  If not, see <http://www.gnu.org/licenses/>.
\providecommand\classoption{12pt}
\documentclass[\classoption]{beamer}

\newcommand{\orcid}[1]{\href{https://orcid.org/#1}{\includegraphics[height=10pt]{images/orcid}}}

\mode<beamer>{
\usepackage{template/dbt}
\setbeamercolor{mycolor}{fg=background,bg=background}
}

\mode<handout>{%
	\usepackage{handoutWithNotes} 
    \pgfpagesuselayout{4 on 1 with notes}[letterpaper] 
    \setbeameroption{show notes}
    \usetheme{default}
}

\beamertemplatenavigationsymbolsempty

\setbeamertemplate{bibliography item}{\insertbiblabel}

% usepackage
\usepackage{subcaption}

\definecolor{azure}{rgb}{0.0, 0.5, 1.0}
\definecolor{awesome}{rgb}{1.0, 0.13, 0.32}
\definecolor{cadetgrey}{rgb}{0.57, 0.64, 0.69}
\definecolor{asparagus}{rgb}{0.53, 0.66, 0.42}
\definecolor{atomictangerine}{rgb}{1.0, 0.6, 0.4}

\usepackage{lstautodedent}
% lstlisting
\lstset{
    language=C,
    basicstyle=\footnotesize\ttfamily,
    keywordstyle=\color{awesome},
    showspaces=false,
    showstringspaces=false,
    commentstyle=\color{azure}\emph,
	autodedent
    %frame=single,
    %rulecolor=\color{comments},
    %rulesepcolor=\color{comments},
    %backgroundcolor = \color{lightgray}
}

\lstset{inputpath=./ParallelComputationMathExamples/}

\usepackage[
    type={CC},
    modifier={by-nc-nd},
    version={4.0},
]{doclicense} 

\usepackage{ifxetex}

\ifxetex
\usepackage{fontspec}
\setmainfont{Raleway}
\fi

\ifluatex
\usepackage{fontspec}
\setmainfont{Raleway}
\fi

\usepackage{xfrac}

\usepackage{tikz}
\usepackage{pgfplots}


\newcommand{\courseurl}[0]{https://www.cct.lsu.edu/\string~pdiehl/teaching/2021/4997/}
\newcommand{\coursetimeline}[0]{https://www.cct.lsu.edu/~pdiehl/teaching/2021/4997/timeline.pdf}
\newcommand{\coursesyllabus}[0]{https://www.cct.lsu.edu/~pdiehl/teaching/2021/4997/syllabus.pdf}
\newcommand{\coursename}[0]{}
\newcommand{\coursemailinglist}[0]{https://mail.cct.lsu.edu/mailman/listinfo/par4997}
\newcommand{\coursesemester}{Fall 2023}
\newcommand{\courselocation}{TBA}
\newcommand{\coursedate}{Tuesday and Thursday, 09:00 to 10:20}
\newcommand{\courseserver}[0]{https://hpx-tutorial.cct.lsu.edu}
\newcommand{\coursetitle}[0]{Applications of high performance computing with C++ in biomedical science and engineering}




% frame slide
\title{\coursename}
\subtitle{Lecture 14: Serial partition-based 1D heat equation  }

\author{\tiny Patrick Diehl \orcid{0000-0003-3922-8419}}
%\institute {
%    \href{}{\tt \scriptsize \today}
%}
\date {
 \tiny \url{\courseurl}
\vspace{2cm}
\doclicenseThis  
  
}



\usepackage{ifxetex}

\ifxetex
\usepackage{fontspec}
\setmainfont{Raleway}
\fi

\ifluatex
\usepackage{fontspec}
\setmainfont{Raleway}
\fi


\begin{document} {
    \setbeamertemplate{footline}{}
    \frame {
        \titlepage
    }
}

\frame{

\tableofcontents

}

\AtBeginSection[]{
  \begin{frame}
  \vfill
  \centering
  \begin{beamercolorbox}[sep=8pt,center,shadow=true,rounded=true]{title}
    \usebeamerfont{title}\insertsectionhead\par%
  \end{beamercolorbox}
  \vfill
  \end{frame}
}

%%%%%%%%%%%%%%%%%%%%%%%%%%%%%%%%%%%%%%%%%%%%%%%%%%%%%%%%%%%%%%%%%%%%%%%%%%%%%%
\section{Reminder}
%%%%%%%%%%%%%%%%%%%%%%%%%%%%%%%%%%%%%%%%%%%%%%%%%%%%%%%%%%%%%%%%%%%%%%%%%%%%%%
\begin{frame}{Lecture 13}
\begin{block}{What you should know from last lecture}
\begin{itemize}
\item \lstinline|hpx::make_reday_future|
\item \lstinline|hpx::dataflow|
\item \lstinline|hpx::unwrapping|
\item \lstinline|hpx::shared_future|
\end{itemize}
\end{block}
\end{frame}

\section{Recap of the previous implementation}

\begin{frame}{Scaling results from previous example}
\begin{center}
\begin{tikzpicture}[scale=0.75]
\selectcolormodel{gray}
\begin{axis}[xlabel=Grid points,ylabel=Execution time, grid=both,title= Stencil 2,legend pos=north west]
\addplot table [x=Grid_Points, y=Execution_Time_sec, col sep=comma] {./data/stencil_2_1.dat};
\addlegendentry{1 CPU}
\addplot table [x=Grid_Points, y=Execution_Time_sec, col sep=comma] {./data/stencil_2_2.dat};
\addlegendentry{2 CPU}
\addplot table [x=Grid_Points, y=Execution_Time_sec, col sep=comma] {./data/stencil_2_4.dat};
\addlegendentry{4 CPU}
\addplot table [x=Grid_Points, y=Execution_Time_sec, col sep=comma] {./data/stencil_2_6.dat};
\addlegendentry{6 CPU}
\end{axis}
\end{tikzpicture}
\end{center}
Note that we need to control the grain size (the amount of work) to get better scalabilty. 
\end{frame}


%%%%%%%%%%%%%%%%%%%%%%%%%%%%%%%%%%%%%%%%%%%%%%%%%%%%%%%%%%%%%%%%%%%%%%%%%%%%%%
\section{Introducing partitions}
%%%%%%%%%%%%%%%%%%%%%%%%%%%%%%%%%%%%%%%%%%%%%%%%%%%%%%%%%%%%%%%%%%%%%%%%%%%%%%

\begin{frame}{Goal of this week's lectures}
We will reuse the serial implementation of Lecture 12 and

\begin{itemize}
\item introduces a partitioning of the 1D grid into groups of grid partitions
\item which are handled at the same time.
\end{itemize}
So we can control the the amount of work performed (grain size) to improve the scalabilty of our program. \\
\vspace{0.25cm}

Note that before, we had exact one discrete mesh point handle per \lstinline|hpx::future| and now we want to have multiple discrete mesh points. 
\end{frame}

\begin{frame}[fragile]{Generating partitions}

\begin{center}
\begin{tikzpicture}

\foreach \i in {1,...,9}{
	\node[color=black] at (\i,0) {\pgfuseplotmark{square*}};
}
\foreach \i in {1,...,9}{
	\node[below] at (\i,-0.1) {$x_\i$};	
}

\draw (1,-.6) -- (1,-1);
\draw (3.5,-.6) -- (3.5,-1);
\draw (6.5,-.6) -- (6.5,-1);
\draw (9,-.6) -- (9,-1);
\node[below] at (2.5,-1) {$n_1$};	
\node[below] at (5,-1) {$n_2$};	
\node[below] at (8,-1) {$n_3$};	
\end{tikzpicture}
\end{center}

\begin{block}{Struct holding the data for each partition}
\begin{lstlisting}
struct partition_data
{
private:
    std::vector<double> data_;
};
\end{lstlisting}
\end{block}

\end{frame}

\begin{frame}[fragile]{Data structure for the partitions I}

\begin{lstlisting}
struct partition_data
{
  partition_data(std::size_t size = 0)
      : data_(size)
    {}

  partition_data(std::size_t size, double int_value)
      : data_(size)
    {
        double base_value = 
              double(int_value * size);
        for (std::size_t i = 0; i != size; ++i)
            data_[i] = base_value + double(i);
    }
}
\end{lstlisting}


\end{frame}

\begin{frame}[fragile]{Data structure for the partitions II}

\begin{lstlisting}
struct partition_data
{
  double& operator[](std::size_t idx) { 
     return data_[idx]; 
  }
    
  double operator[](std::size_t idx) const { 
       return data_[idx]; 
    }

    std::size_t size() const 
    { 
       return data_.size(); 
    }
}
\end{lstlisting}


\end{frame}

\section{Swapping the partitions}


\begin{frame}{New swapping scheme}
\vspace{-0.5cm}
\begin{center}
\begin{tikzpicture}

\foreach \j in {0,...,1}{


\foreach \i in {1,...,4}{
	\node[color=black] at (\j*5+\i,1.5) {\pgfuseplotmark{square*}};
}

\foreach \i in {1,...,4}{
	\node[color=black] at (\j*5+\i,0.75) {\pgfuseplotmark{square*}};
}


\foreach \i in {1,...,4}{
	\node[color=black] at (\j*5+\i,0) {\pgfuseplotmark{square*}};
}
\foreach \i in {1,...,4}{
	\node[below] at (\j*5+\i,-0.1) {$x_\i$};	
}

\node[left] at  (\j*5+0.5,0) {\small U[0][\j]};
\node[left] at  (\j*5+0.5,0.75) {\small U[1][\j]};
\node[left] at  (\j*5+0.5,1.5) {\small U[0][\j]};

\draw (\j*5+2,0) -- (\j*5+3,0.75);
\draw (\j*5+3,0) -- (\j*5+3,0.75);
\draw (\j*5+4,0) -- (\j*5+3,0.75);

\draw (\j*5+2,0.75) -- (\j*5+3,1.5);
\draw (\j*5+3,0.75) -- (\j*5+3,1.5);
\draw (\j*5+4,0.75) -- (\j*5+3,1.5);

} 

\node[right] at  (9.5,0) {t=1};
\node[right] at  (9.5,0.75) {t=2};
\node[right] at  (9.5,1.5) {t=3};

 

\end{tikzpicture}
\end{center}
\end{frame}


\begin{frame}[fragile]{Adapted class stepper}

\begin{lstlisting}
class stepper
{
    // Our data for one time step
    typedef partition_data partition;
    typedef std::vector<partition> space;
    
    std::vector<space> U(2);
    for (space& s: U)
        // np is the number of partitions
        s.resize(np);
        
    // Initial conditions: f(0, i) = i
    for (std::size_t i = 0; i != np; ++i)
        U[0][i] = partition_data(nx, double(i));
        
    // Return the solution at time-step 'nt'.
    return U[nt % 2];
    
}
\end{lstlisting}

\end{frame}

%%%%%%%%%%%%%%%%%%%%%%%%%%%%%%%%%%%%%%%%%%%%%%%%%%%%%%%%%%%%%%%%%%%%%%%%%%%%%%
\section{New C++ and HPX features }
%%%%%%%%%%%%%%%%%%%%%%%%%%%%%%%%%%%%%%%%%%%%%%%%%%%%%%%%%%%%%%%%%%%%%%%%%%%%%%

\begin{frame}[fragile]{Moving objects with \lstinline|std::move|\footnote{\tiny\url{https://en.cppreference.com/w/cpp/utility/move}}} 
We use \lstinline|std::move| to indicate that the object may be moved to another object. This allows the efficient transfer of resources to another object.

\begin{block}{Example}
\begin{lstlisting}
std::string str = "Hello";
std::vector<std::string> v;

v.push_back(std::move(str));
\end{lstlisting}
\end{block}

\begin{block}{Be aware of undefined states, but valid states}
\begin{lstlisting}
str.clear{} //Ok, since clear has no preconditions
str.back() //Undefined behavior if size()==0
\end{lstlisting}
\end{block}


\end{frame}

\begin{frame}{Semaphore}

\begin{block}{Analogy}
Imagine a public library lending books with no late fee. The have X copies of the Hitchhiker's Guide to the Galaxy~\cite{adams2017hitchhiker} to borrow. Five people can borrow these copies and the library does not care to get them back in a feasible amount of time, since they avoided to introduce late fees. If one is waiting for a copy the copy will be assigned to this person, but if none is waiting the copy just goes back to the shelf until one asks for it. 
\end{block}
\vspace{0.5cm}
The concept of semaphores was introduced by E. Dijkstra~\cite{dijkstra1962over}. More details~\cite{downey2008little}.
\end{frame}

\begin{frame}{P and V operations on a semaphore objects}

\begin{block}{Variables}
\begin{itemize}
\item maximum count
\item current count
\end{itemize}
\end{block}

\begin{block}{Operations}
\begin{itemize}
\item Taking ownership with the \lstinline|wait| function, which decrements the semaphore. The so--called $P$ operation from Dijkstra's paper.
\item Releasing ownership with the \lstinline|signal| function, the increments the semaphore. The so--called $V$ operation from Dijkstra's paper.
\end{itemize}
\end{block}

\end{frame}

\begin{frame}{More details}

\begin{block}{$P$-Operation}
If the \lstinline|wait| function is called, the current count is decreased. If the count is $\leq$ zero then the decrement just happens and the function will return. If the count is zero the function will wait until one other thread called the  \lstinline|signal| function.
\end{block}

\begin{block}{$V$-Operation}
If the \lstinline|signal| function is called, the current count is increased. If the count was zero before you called \lstinline|signal| function and another thread was blocked in \lstinline|wait| then that thread will be executed. If multiple threads are waiting, only one will be executed and the reaming ones have to wait for another increment of the counter.
\end{block}

\end{frame}

\begin{frame}[fragile]{Semaphores in the C++ standard}

We looked into \lstinline|std::mutex| in Lecture 6, which is tied to one thread and only one thread can lock or unlock the mutex. Any thread can access the ownership on a semaphore. \\
\begin{center}
\textcolor{red}{The C++ standard does not define semaphores.} 
\end{center}

\begin{lstlisting}
// Generate a semaphore with maximal count nd
hpx::lcos::local::sliding_semaphore sem(nd);

// Release ownership for t
sem.signal(t);

// Obtain ownership for t
sem.wait(t);
\end{lstlisting}


\end{frame}

%%%%%%%%%%%%%%%%%%%%%%%%%%%%%%%%%%%%%%%%%%%%%%%%%%%%%%%%%%%%%%%%%%%%%%%%%%%%%%
\section{Summary}
%%%%%%%%%%%%%%%%%%%%%%%%%%%%%%%%%%%%%%%%%%%%%%%%%%%%%%%%%%%%%%%%%%%%%%%%%%%%%%
\begin{frame}{Summary}
\begin{block}{After this lecture, you should know}
\begin{itemize}
\item Using the partitions to control the grain size
\item \lstinline|std::move| for moving objects
\item Semaphores and \lstinline|hpx::lcos::local::sliding_semaphore|
\end{itemize}
\end{block}
\end{frame}

%%%%%%%%%%%%%%%%%%%%%%%%%%%%%%%%%%%%%%%%%%%%%%%%%%%%%%%%%%%%%%%%%%%%%%%%%%%%%%
\section{References}
%%%%%%%%%%%%%%%%%%%%%%%%%%%%%%%%%%%%%%%%%%%%%%%%%%%%%%%%%%%%%%%%%%%%%%%%%%%%%%

\begin{frame}[t, allowframebreaks]
\frametitle{References}
\bibliographystyle{plain}
\bibliography{bib}
\end{frame}

\end{document}