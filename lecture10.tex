%Template
%Copyright (C) 2019  Patrick Diehl
%
%This program is free software: you can redistribute it and/or modify
%it under the terms of the GNU General Public License as published by
%the Free Software Foundation, either version 3 of the License, or
%(at your option) any later version.

%This program is distributed in the hope that it will be useful,
%but WITHOUT ANY WARRANTY; without even the implied warranty of
%MERCHANTABILITY or FITNESS FOR A PARTICULAR PURPOSE.  See the
%GNU General Public License for more details.

%You should have received a copy of the GNU General Public License
%along with this program.  If not, see <http://www.gnu.org/licenses/>.
\documentclass[12pt,t]{beamer}

\beamertemplatenavigationsymbolsempty

% usepackage
%\usepackage{template/dbt}
\usepackage{listings}
\usepackage{tikz,pgfplots}
\pgfplotsset{compat=1.16} 

\definecolor{comments}{RGB}{81,81,81}
\definecolor{keywords}{RGB}{255,0,90}

% lstlisting
\lstset{
    language=C,
    basicstyle=\footnotesize\ttfamily,
    keywordstyle=\color{keywords},
    showspaces=false,
    showstringspaces=false,
    commentstyle=\color{blue}\emph
    %frame=single,
    %rulecolor=\color{comments},
    %rulesepcolor=\color{comments},
    %backgroundcolor = \color{lightgray}
}

\usetheme{default}

\usepackage[
    type={CC},
    modifier={by-nc-nd},
    version={4.0},
]{doclicense} 

\newcommand{\courseurl}[0]{https://www.cct.lsu.edu/\string~pdiehl/teaching/2021/4997/}
\newcommand{\coursetimeline}[0]{https://www.cct.lsu.edu/~pdiehl/teaching/2021/4997/timeline.pdf}
\newcommand{\coursesyllabus}[0]{https://www.cct.lsu.edu/~pdiehl/teaching/2021/4997/syllabus.pdf}
\newcommand{\coursename}[0]{}
\newcommand{\coursemailinglist}[0]{https://mail.cct.lsu.edu/mailman/listinfo/par4997}
\newcommand{\coursesemester}{Fall 2023}
\newcommand{\courselocation}{TBA}
\newcommand{\coursedate}{Tuesday and Thursday, 09:00 to 10:20}
\newcommand{\courseserver}[0]{https://hpx-tutorial.cct.lsu.edu}
\newcommand{\coursetitle}[0]{Applications of high performance computing with C++ in biomedical science and engineering}




% frame slide
\title{\coursename}
\subtitle{Lecture 9: Solvers, Conjugate gradient method, and BlazeIterative}

%\author{\href{}{}}
%\institute {
%    \href{}{\tt \scriptsize \today}
%}
\date {
 \tiny \url{\courseurl}
\vspace{2cm}
\doclicenseThis  
  
}



\usepackage{ifxetex}

\ifxetex
\usepackage{fontspec}
\setmainfont{Raleway}
\fi

\ifluatex
\usepackage{fontspec}
\setmainfont{Raleway}
\fi


\begin{document} {
    \setbeamertemplate{footline}{}
    \frame {
        \titlepage
    }
}

\frame{

\tableofcontents

}

\AtBeginSection[]{
  \begin{frame}
  \vfill
  \centering
  \begin{beamercolorbox}[sep=8pt,center,shadow=true,rounded=true]{title}
    \usebeamerfont{title}\insertsectionhead\par%
  \end{beamercolorbox}
  \vfill
  \end{frame}
}

%%%%%%%%%%%%%%%%%%%%%%%%%%%%%%%%%%%%%%%%%%%%%%%%%%%%%%%%%%%%%%%%%%%%%%%%%%%%%%
\section{Reminder}
%%%%%%%%%%%%%%%%%%%%%%%%%%%%%%%%%%%%%%%%%%%%%%%%%%%%%%%%%%%%%%%%%%%%%%%%%%%%%%
\begin{frame}{Lecture 9}
\begin{block}{What you should know from last lecture}
\begin{itemize}
\item Vectors and matrices
\item How to use Blaze for matrix and vector operations
\item How to compile a program using a external library
\end{itemize}
\end{block}
\end{frame}

%%%%%%%%%%%%%%%%%%%%%%%%%%%%%%%%%%%%%%%%%%%%%%%%%%%%%%%%%%%%%%%%%%%%%%%%%%%%%%
\section{Solving linear equation systems}
%%%%%%%%%%%%%%%%%%%%%%%%%%%%%%%%%%%%%%%%%%%%%%%%%%%%%%%%%%%%%%%%%%%%%%%%%%%%%%

%%%%%%%%%%%%%%%%%%%%%%%%%%%%%%%%%%%%%%%%%%%%%%%%%%%%%%%%%%%%%%%%%%%%%%%%%%%%%%
\section{Conjugate gradient method}
%%%%%%%%%%%%%%%%%%%%%%%%%%%%%%%%%%%%%%%%%%%%%%%%%%%%%%%%%%%%%%%%%%%%%%%%%%%%%%

\begin{frame}{Conjugate gradient method}

\begin{block}{Properties:}
\begin{itemize}
\item Most popular iterative method for solving large systems of linear equations
\item Developed by Hestenes and Stiefel in 1952~\cite{hestenes1952methods}
\item Solves linear equation systems $\mathbf{A} \mathbf{x} = \mathbf{b}$
\item Each iteration does one matrix-vector multiplication and some computation of inner products
\end{itemize}
\end{block}

\begin{block}{Matrix}
\begin{itemize}
\item Symmetry $\mathbf{A}^T = \mathbf{A}$
\item Positive-definite $\mathbf{x}^T \mathbf{A} \mathbf{x} > 0, \forall \mathbf{x}>0$
\end{itemize}
\end{block}

More details about iterative methods~\cite{briggs2000multigrid}.
\end{frame}


\begin{frame}{Illustration of the linear system}

\begin{center}
\begin{tikzpicture}
\begin{axis}[grid=both,ymin=-6,ymax=4,xmax=8,xmin=-4,
               minor tick num=1,axis lines = middle,xlabel=$x_1$,ylabel=$x_2$]
\addplot [domain=-4:6, samples=101]{-1.5*x+1} node[pos=0.425] (endofplotsquare) {};
\node [right] at (endofplotsquare) {$3x_1+2x_2=2$};
\addplot [domain=-4:6, samples=101]{-(1/3)*x-(8/6)} node[pos=0.675] (endofplotsquare2) {};
\node [right] at (endofplotsquare2) {$2x_1+6x_2=-8$};
\end{axis}
\end{tikzpicture}
\end{center}

\end{frame}



\begin{frame}{The quadratic form}
Let us define the problem as a matrix:\\
\begin{center}
$\mathbf{A} \mathbf{x}=\mathbf{b}$
\end{center}
with 
\begin{center}
$\mathbf{A} = \begin{pmatrix}
 3 & 2 \\ 2 & 6 
\end{pmatrix}, \mathbf{x}=\begin{pmatrix}
x_1 \\ x_2  
\end{pmatrix}, \text{ and }  \mathbf{b}=\begin{pmatrix}
2 \\ -8  
\end{pmatrix}\text{.}$
\end{center}
\vspace{0.25cm}


Instead of solving $\mathbf{A}\mathbf{x}=\mathbf{b}$, the quadratic form, which is a function of $\mathbf{x}$ can be 

\begin{center}
$ f(\mathbf{x}) = \frac{1}{2} \mathbf{x}^T A \mathbf{x} - b^T \mathbf{x} + c$
\end{center}
can be minimized to find the solution $\mathbf{x}$.
\vspace{0.25cm}

\end{frame}

\begin{frame}{Plot of the quadratic form $f(\mathbf{x})$}

\begin{center}
\begin{tikzpicture}
\begin{axis}[ymin=-6,ymax=4,xmax=8,xmin=-4,zmin=0,zmax=160,xlabel=$x_1$,ylabel=$x_2$,zlabel={$f(\mathbf{x})$}]
\addplot3[
    surf,
    colormap/cool,
    %shader=interp,
    %shader=flat,
    samples=60] 
    {x*(1.5*x+y)-2*x+y*(x+3*y)+8*y};
\end{axis}
\end{tikzpicture}
\end{center}
\vspace{-0.5cm}
Finding the minimal point of $\mathbf{x}$ corresponds to the solution of $\mathbf{A} \mathbf{x}=\mathbf{b}$.
\end{frame}



\begin{frame}{Contour plot of the quadratic form $f(\mathbf{x})$}

\begin{center}
\begin{tikzpicture}
\begin{axis}[ymin=-6,ymax=4,xmax=6,xmin=-4,xlabel=$x_1$,ylabel=$x_2$]
\addplot gnuplot[raw gnuplot,thick,mark=none]
    {
        unset surface;
        set cntrparam levels auto 10;
        set contour;
        set yrange [-6:4];
        splot [x=-4:6] x*(1.5*x+y)-2*x+y*(x+3*y)+8*y;
    };
\addplot[mark=*] coordinates {(2,-2)};
\end{axis}
\end{tikzpicture}
\end{center}
\vspace{-0.5cm}
\end{frame}


%%%%%%%%%%%%%%%%%%%%%%%%%%%%%%%%%%%%%%%%%%%%%%%%%%%%%%%%%%%%%%%%%%%%%%%%%%%%%%
\section{Blaze Iterative}
%%%%%%%%%%%%%%%%%%%%%%%%%%%%%%%%%%%%%%%%%%%%%%%%%%%%%%%%%%%%%%%%%%%%%%%%%%%%%%


\begin{frame}[fragile]{About BlazeIterative\footnote{\tiny\url{https://github.com/STEllAR-GROUP/BlazeIterative}}}
This is a set of iterative linear system solvers intended for use with the Blaze library, a high-performance C++ linear algebra library. The API is currently based on a tag-dispatch system to choose a particular algorithm.

\begin{block}{Usage}
\begin{lstlisting}[language=bash]
#Install
tar -xvf blaze_iterative.gz
cd blaze_iterative
cp -r ./blaze_iterative /home/patrick/

#Compile
g++ -I/home/diehlpk/blaze 
    -I/home/patrick/blaze_iterative BlazeTest.cpp
\end{lstlisting}
\end{block}


\end{frame}

\begin{frame}[fragile]{Conjugate gradient example}
\begin{lstlisting}
#include "BlazeIterative.hpp"

using namespace blaze;
using namespace blaze::iterative;

std::size_t N = 10;
DynamicMatrix<double,false> A(N,N, 0.0);
DynamicVector<double> b(N, 0.0);
DynamicVector<double> x1(N, 0.);

//Initialize the matrix

// Solve the system
ConjugateGradientTag tag;
auto x2 = solve(A,b,tag);

\end{lstlisting}

\end{frame}

\begin{frame}{Available algorithms}

\begin{block}{Solvers}
\begin{itemize}
\item Conjugate Gradient
\item Preconditioned CG
\item BiCGSTAB
\item Generalized minimal residual method (GMRES),
\end{itemize}
\end{block}


\begin{block}{Eigenvalues}
\begin{itemize}
\item Lanczos
\end{itemize}
\end{block}
More details about solvers~\cite{barrett1994templates}.
\end{frame}




%%%%%%%%%%%%%%%%%%%%%%%%%%%%%%%%%%%%%%%%%%%%%%%%%%%%%%%%%%%%%%%%%%%%%%%%%%%%%%
\section{Summary}
%%%%%%%%%%%%%%%%%%%%%%%%%%%%%%%%%%%%%%%%%%%%%%%%%%%%%%%%%%%%%%%%%%%%%%%%%%%%%%
\begin{frame}{Summary}
\begin{block}{After this lecture, you should know}
\begin{itemize}
\item Linear equation systems
\item Conjugate gradient method
\item BlazeIterative
\end{itemize}
\end{block}
\end{frame}

%%%%%%%%%%%%%%%%%%%%%%%%%%%%%%%%%%%%%%%%%%%%%%%%%%%%%%%%%%%%%%%%%%%%%%%%%%%%%%
\section{References}
%%%%%%%%%%%%%%%%%%%%%%%%%%%%%%%%%%%%%%%%%%%%%%%%%%%%%%%%%%%%%%%%%%%%%%%%%%%%%%

\begin{frame}[t, allowframebreaks]
\frametitle{References}
\bibliographystyle{plain}
\bibliography{bib}
\end{frame}


\end{document}