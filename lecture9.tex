%Template
%Copyright (C) 2019  Patrick Diehl
%
%This program is free software: you can redistribute it and/or modify
%it under the terms of the GNU General Public License as published by
%the Free Software Foundation, either version 3 of the License, or
%(at your option) any later version.

%This program is distributed in the hope that it will be useful,
%but WITHOUT ANY WARRANTY; without even the implied warranty of
%MERCHANTABILITY or FITNESS FOR A PARTICULAR PURPOSE.  See the
%GNU General Public License for more details.

%You should have received a copy of the GNU General Public License
%along with this program.  If not, see <http://www.gnu.org/licenses/>.
\documentclass[12pt,t]{beamer}

\beamertemplatenavigationsymbolsempty

% usepackage
%\usepackage{template/dbt}
\usepackage{listings}

\definecolor{comments}{RGB}{81,81,81}
\definecolor{keywords}{RGB}{255,0,90}

% lstlisting
\lstset{
    language=C,
    basicstyle=\footnotesize\ttfamily,
    keywordstyle=\color{keywords},
    showspaces=false,
    showstringspaces=false,
    commentstyle=\color{blue}\emph
    %frame=single,
    %rulecolor=\color{comments},
    %rulesepcolor=\color{comments},
    %backgroundcolor = \color{lightgray}
}

\usetheme{default}

\usepackage[
    type={CC},
    modifier={by-nc-nd},
    version={4.0},
]{doclicense} 

\newcommand{\courseurl}[0]{https://www.cct.lsu.edu/\string~pdiehl/teaching/2021/4997/}
\newcommand{\coursetimeline}[0]{https://www.cct.lsu.edu/~pdiehl/teaching/2021/4997/timeline.pdf}
\newcommand{\coursesyllabus}[0]{https://www.cct.lsu.edu/~pdiehl/teaching/2021/4997/syllabus.pdf}
\newcommand{\coursename}[0]{Math 4997-3}
\newcommand{\coursemailinglist}[0]{https://mail.cct.lsu.edu/mailman/listinfo/par4997}
\newcommand{\coursesemester}{Fall 2021}








% frame slide
\title{\coursename}
\subtitle{Lecture 9: Linear algebra with Blaze}

%\author{\href{}{}}
%\institute {
%    \href{}{\tt \scriptsize \today}
%}
\date {
 \tiny \url{\courseurl}
\vspace{2cm}
\doclicenseThis  
  
}



\usepackage{ifxetex}

\ifxetex
\usepackage{fontspec}
\setmainfont{Raleway}
\fi

\ifluatex
\usepackage{fontspec}
\setmainfont{Raleway}
\fi


\begin{document} {
    \setbeamertemplate{footline}{}
    \frame {
        \titlepage
    }
}

\frame{

\tableofcontents

}

\AtBeginSection[]{
  \begin{frame}
  \vfill
  \centering
  \begin{beamercolorbox}[sep=8pt,center,shadow=true,rounded=true]{title}
    \usebeamerfont{title}\insertsectionhead\par%
  \end{beamercolorbox}
  \vfill
  \end{frame}
}

%%%%%%%%%%%%%%%%%%%%%%%%%%%%%%%%%%%%%%%%%%%%%%%%%%%%%%%%%%%%%%%%%%%%%%%%%%%%%%
\section{Reminder}
%%%%%%%%%%%%%%%%%%%%%%%%%%%%%%%%%%%%%%%%%%%%%%%%%%%%%%%%%%%%%%%%%%%%%%%%%%%%%%
\begin{frame}{Lecture 8}
\begin{block}{What you should know from last lecture}
\begin{itemize}
\item Bond-based Peridynamics (Course project)
\end{itemize}
\end{block}
\end{frame}

%%%%%%%%%%%%%%%%%%%%%%%%%%%%%%%%%%%%%%%%%%%%%%%%%%%%%%%%%%%%%%%%%%%%%%%%%%%%%%
\section{Vectors and Matrices}
%%%%%%%%%%%%%%%%%%%%%%%%%%%%%%%%%%%%%%%%%%%%%%%%%%%%%%%%%%%%%%%%%%%%%%%%%%%%%%


\begin{frame}{Vector space~\cite{hefferonlinear,scheick1997linear}}
A vector space (or a linear space) is a collection of so-called vectors $\mathbf{v}\in \mathbb{R}^n$. Vectors can be added or scaled (multiplied by a scalar value).

\begin{center}
$\mathbf{v} = \lbrace v_1,v_2,\ldots,v_n \rbrace $ \\
$\mathbf{w} = \lbrace w_1,w_2,\ldots,w_n \rbrace $
\end{center}

\begin{block}{Addition}
\begin{center}
$\mathbf{v} + \mathbf{w} = \lbrace v_1+w_1,v_2+w_2,\ldots,v_n+w_n \rbrace $
\end{center}
\end{block}

\begin{block}{Scaling}
\begin{center}
$ 2 \mathbf{v} = \lbrace 2v_1,2v_2,\ldots,2v_n \rbrace $
\end{center}
\end{block}

\end{frame}

\begin{frame}{Vector II}

\begin{block}{Column vector}
\begin{center}
$\mathbf{v} = \left\lbrace \begin{matrix}
v_1 \\ \vdots \\ v_n 
\end{matrix} \right\rbrace$
\end{center}
\end{block}

\begin{block}{Row vector}
\begin{center}
$\mathbf{v}^T = \lbrace v_1,\ldots v_n\rbrace$
\end{center}
\end{block}

\end{frame}


\begin{frame}{Matrix}
A matrix $\mathbf{A}\in \mathbb{R}^{n,m}$ has $n$ rows and $m$ columns
\begin{center}
$
\mathbf{A} = \begin{pmatrix}
a_{1,1} & \ldots & a_{1,m} \\
\vdots & \ldots & \vdots \\
a_{n,1} & \ldots & a_{n,m} 
\end{pmatrix}
$
\end{center}

\begin{block}{Scaling}
\begin{center}
$ 2 \mathbf{A} = \begin{pmatrix}
2a_{1,1} & \ldots & 2a_{1,m} \\
\vdots & \ldots & \vdots \\
2a_{n,1} & \ldots & 2a_{n,m} 
\end{pmatrix} $
\end{center}
\end{block}

\end{frame}

\begin{frame}{Matrix II}

\begin{block}{Addition}
\begin{center}
$ \mathbf{A} + \mathbf{B}  = \begin{pmatrix}
a_{1,1} + b_{1,1}  & \ldots & a_{1,m} + b_{1,m} \\
\vdots & \ldots & \vdots \\
a_{n,1} + b_{n,1} & \ldots & a_{n,m} + b_{n,m} 
\end{pmatrix} $
\end{center}
\end{block}

\begin{block}{Matrix vector multiplication}
\begin{center}
$ \mathbf{A}  \mathbf{v}  = \left\lbrace\begin{matrix}
a_{1,1} * b_{1} + & \ldots & + a_{1,m} * b_{n} \\
\vdots & \ldots & \vdots \\
a_{n,1} * b_{1} + & \ldots & + a_{n,m} * b_{n} 
\end{matrix} \right\rbrace $
\end{center}
\end{block}

\end{frame}


%%%%%%%%%%%%%%%%%%%%%%%%%%%%%%%%%%%%%%%%%%%%%%%%%%%%%%%%%%%%%%%%%%%%%%%%%%%%%%
\section{Applications}
%%%%%%%%%%%%%%%%%%%%%%%%%%%%%%%%%%%%%%%%%%%%%%%%%%%%%%%%%%%%%%%%%%%%%%%%%%%%%%


\begin{frame}{Communication}
We have a group of people $P_1,\ldots,P_n$, if person $P_1$ has contact with person $P_2$, we can model this information by setting the matrix element $a_{1,2}=1$. By doing this for all people in our group, we will get some matrix \\

\begin{center}
$
M = \left\lbrace\begin{matrix}
0 & 1 & 0 & 1 \\
0 & 0 & 1 & 0 \\
1 & 0 & 0 & 1 \\
1 & 1 & 0 & 0
\end{matrix} \right\rbrace
$
\end{center}
This matrix will tell us that $P_1$ has contact with $P_2$ and $P_4$, $P_2$ with $P_3$ and so on. \\
Now we define
\begin{center}
$M^2 = m^2_{i,j}$,  
\end{center}
for example $m^2_{3,2}=a_{3,1}a_{1,2}+a_{3,2}a_{2,2}+a_{3,4}a_{4,2}$.
\end{frame}

\begin{frame}{Communication}
We now can compute 

\begin{center}
$
M^2 = \left\lbrace\begin{matrix}
1 &  1 & 1 &  0 \\
1 & 0 & 0  & 1 \\
1 & 2 & 0 & 1 \\
0 & 1 & 1 & 1
\end{matrix} \right\rbrace
$
\end{center}
and see that person $P_3$ can send some message to Person $P_2$ in two cycles. 


\end{frame}


%%%%%%%%%%%%%%%%%%%%%%%%%%%%%%%%%%%%%%%%%%%%%%%%%%%%%%%%%%%%%%%%%%%%%%%%%%%%%%
\section{Blaze}
%%%%%%%%%%%%%%%%%%%%%%%%%%%%%%%%%%%%%%%%%%%%%%%%%%%%%%%%%%%%%%%%%%%%%%%%%%%%%%

\begin{frame}{Blaze\footnote{\tiny\url{https://bitbucket.org/blaze-lib/blaze/src/master/}}}
Blaze is an open-source, high-performance C++ math library for dense and sparse arithmetic. With its state-of-the-art Smart Expression Template implementation Blaze combines the elegance and ease of use of a domain-specific language with HPC-grade performance, making it one of the most intuitive and fastest C++ math libraries available.
\vspace{1cm}

More details about the implementation details~\cite{doi:10.1137/110830125,6266939}.
\end{frame}



\begin{frame}[fragile]{Installation\footnote{\tiny\url{https://bitbucket.org/blaze-lib/blaze/wiki/Configuration\%20and\%20Installation}}}


\begin{block}{CMake}
\begin{lstlisting}[language=bash]
tar -xvf blaze-3.6.tar.gz
cd blaze-3.6
cmake -DCMAKE_INSTALL_PREFIX=/home/patrick/blaze
make install
\end{lstlisting}
\end{block}

\begin{block}{Manual}
\begin{lstlisting}[language=bash]
tar -xvf blaze-3.6.tar.gz
cd blaze-3.6
cp -r ./blaze /home/patrick/blaze
\end{lstlisting}
\end{block}


\end{frame}


\begin{frame}[fragile]{Compilation\footnote{\tiny\url{https://bitbucket.org/blaze-lib/blaze/wiki/Configuration\%20and\%20Installation}}}

\begin{block}{CMake}
\begin{lstlisting}[language=bash]
find_package( blaze )
if( blaze_FOUND )
   add_library( blaze_target INTERFACE )
   target_link_libraries( blaze_target 
                          INTERFACE blaze::blaze )
endif()
\end{lstlisting}
\end{block}

\begin{block}{Compiler}
\begin{lstlisting}
g++ -I/home/diehlpk/blaze BlazeTest.cpp
\end{lstlisting}
\end{block}

\end{frame}


\begin{frame}[fragile]{Parallelism\footnote{\tiny\url{https://bitbucket.org/blaze-lib/blaze/wiki/Shared\%20Memory\%20Parallelization}}}

\begin{block}{C++11 Thread Setup}
Add following arguments to the compiler
\begin{lstlisting}
-std=c++11 -DBLAZE_USE_CPP_THREADS
\end{lstlisting}
and set the number of threads
\begin{lstlisting}
export BLAZE_NUM_THREADS=4  // Unix systems
\end{lstlisting}
\end{block}

\begin{block}{HPX}
Add following arguments to the compiler
\begin{lstlisting}
-DBLAZE_USE_HPX_THREADS
\end{lstlisting}
and set the number of threads
\begin{lstlisting}
./a.out --hpx:threads=4
\end{lstlisting}
\end{block}

\end{frame}

%%%%%%%%%%%%%%%%%%%%%%%%%%%%%%%%%%%%%%%%%%%%%%%%%%%%%%%%%%%%%%%%%%%%%%%%%%%%%%
\section{Blaze's API}
%%%%%%%%%%%%%%%%%%%%%%%%%%%%%%%%%%%%%%%%%%%%%%%%%%%%%%%%%%%%%%%%%%%%%%%%%%%%%%

\begin{frame}[fragile]{Vector\footnote{\tiny\url{https://bitbucket.org/blaze-lib/blaze/wiki/Vectors}}}
\begin{lstlisting}
using blaze::DynamicVector;
using blaze::columnVector;
using blaze::rowVector;

// Setup of the 3-dimensional dense column vector
DynamicVector<int,columnVector> a{ 1, 2, 3 };

// Setup of the 3-dimensional dense row vector
DynamicVector<int,rowVector> b{ 4, 5, 6 };

// Instantiation of a 3-dimensional column vector
blaze::DynamicVector<int> c( 3UL );

// Set all elements of the vector c to 5
c = 5;
\end{lstlisting}
\end{frame}

\begin{frame}[fragile]{Vector operations\footnote{\tiny\url{https://bitbucket.org/blaze-lib/blaze/wiki/Vector\%20Operations}}}
\begin{lstlisting}
// Get the size of the vector
auto size = c.size();

// Access the i-th element
auto value = c[i];

// Loop over the vector
for( size_t i=0UL; i< c.size(); ++i )
   std::cout << c[i] << std::endl;
   
// Iterate over a vector
blaze::CompressedVector<int> d{ 0, 2, 0, 0};
for( CompressedVector<int>::Iterator it=d.begin(); 
  it!=d.end(); ++it ) 
    std::cout << it->value() << std::endl;

\end{lstlisting}
\end{frame}

\begin{frame}[fragile]{Vector operations II\footnote{\tiny\url{https://en.cppreference.com/w/cpp/numeric/complex}}}
\begin{lstlisting}
blaze::DynamicVector<double> a, b;

// Computes the sine of each element of the vector
b = sin( a );  
// Computes the base e exponential of each element
b = exp( a ); 
// Computes the exponential value of each element
b = pow( a, 1.2 );   
// Computes the absolute value of each element
b = abs( a ); 
// Complex numbers
using blaze::StaticVector;
using cplx = std::complex<double>;
StaticVector<cplx,1UL> a{ cplx(-2.0,-1.0)};
double b = imag( a ); //Get the imaginary part 
\end{lstlisting}
\end{frame}



\begin{frame}[fragile]{Dense Matrix\footnote{\tiny\url{https://bitbucket.org/blaze-lib/blaze/wiki/Matrix\%20Types}}}
\begin{lstlisting}
// Definition of a 3x4 matrix 
// Values are not initialized
blaze::DynamicMatrix<int> A( 3UL, 4UL );

// Definition of a 3x4 matrix
// with 0 rows and columns
blaze::StaticMatrix<int,3UL,4UL> A;

// Definition of column-major matrix
// with 0 rows and columns
blaze::DynamicMatrix<double,blaze::columnMajor> C;



\end{lstlisting}

\begin{block}{Remarks:}
\begin{itemize}
\item Default is row-major matrices:
\item Static Matrix are small and size known at compile time
\end{itemize}
\end{block}
\end{frame}


\begin{frame}[fragile]{Sparse matrix}
\begin{lstlisting}
// Definition of a 3x4 integral row-major matrix
blaze::CompressedMatrix<int> A( 3UL, 4UL );


// Definition of a 3x3 identity matrix
blaze::IdentityMatrix<int> A( 3UL );

// Definition of a 3x5 zero matrix
blaze::ZeroMatrix<int> A( 3UL, 5UL );
\end{lstlisting}

\begin{center}
\textcolor{blue}{Sparse matrices are used, if you have only few non-zero entries}.
\end{center}
\end{frame}

\begin{frame}[fragile]{Matrix operation\footnote{\tiny\url{https://bitbucket.org/blaze-lib/blaze/wiki/Arithmetic\%20Operations}}}
\begin{lstlisting}

// Access elements
M1(0,0) = 1;

// Total amount of elements
size( M2 );

// Number of rows
M2.rows(); 

// Number of columns
M2.columns(); 

// Computes the element-wise absolute value
abs( A );
\end{lstlisting}
\end{frame}

\begin{frame}[fragile]{Matrix operation II\footnote{\tiny\url{https://bitbucket.org/blaze-lib/blaze/wiki/Matrix\%20Operations\#!element-access}}}
\begin{lstlisting}
// Traversing the matrix 
blaze::CompressedMatrix<int> M1( 4UL, 4UL );

for( size_t i=0UL; i<M1.rows(); ++i ) {
   for( size_t j=0UL; j<M1.columns(); ++j ) {
      ... = M1(i,j);
   }
}

// Traversing the matrix by Iterator
for( size_t i=0UL; i<A.rows(); ++i ) {
   for( CompressedMatrix<int,rowMajor>::Iterator it=
      A.begin(i); it!=A.end(i); ++it ) {
      it->value() = ...;  
   }
}
\end{lstlisting}
\end{frame}

\begin{frame}[fragile]{Arithmetic operation\footnote{\tiny\url{https://bitbucket.org/blaze-lib/blaze/wiki/Arithmetic\%20Operations}}}
\begin{lstlisting}
blaze::StaticVector<int,3UL> v1{ 3, 2, 5, -4, 1, 6 };
// Addition
blaze::StaticVector<int,3UL> res = v1 + v1;
// Subtraction
blaze::StaticVector<int,3UL> res = v1 - 2 * v1;

blaze::DynamicMatrix<float,rowMajor> M1( 7UL, 3UL );
// Addition
blaze::DynamicMatrix<float,rowMajor> res = M1 + M1;
// Subtraction
blaze::DynamicMatrix<float,rowMajor> res = 2*M1 - M1;
\end{lstlisting}
\end{frame}



\begin{frame}[fragile]{Matrix decomposition\footnote{\tiny\url{https://bitbucket.org/blaze-lib/blaze/wiki/Matrix\%20Operations\#!matrix-decomposition}}}
\begin{lstlisting}
blaze::DynamicMatrix<double,blaze::rowMajor> A;
// ... Resizing and initialization

blaze::DynamicMatrix<double,blaze::rowMajor> L, U, P;

// LU decomposition of a row-major matrix
lu( A, L, U, P );  

assert( A == L * U * P );
\end{lstlisting}

\begin{block}{Decompositions}
\begin{itemize}
\item Cholesky 
\item QR/RQ
\item QL/LQ
\end{itemize}
\end{block}

\end{frame}


\begin{frame}[fragile]{Eigen values\footnote{\tiny\url{https://bitbucket.org/blaze-lib/blaze/wiki/Matrix\%20Operations\#!eigenvalueseigenvectors}}$^,$\footnote{\tiny\url{https://bitbucket.org/blaze-lib/blaze/wiki/Symmetric\%20Matrices}}}
\begin{lstlisting}

// The symmetric matrix A
SymmetricMatrix< DynamicMatrix<double,rowMajor>> 
    A( 5UL, 5UL );  
// ... Initialization

// The vector for the real eigenvalues
DynamicVector<double,columnVector> w( 5UL );    
// The matrix for the left eigenvectors   
DynamicMatrix<double,rowMajor>     V( 5UL, 5UL );  

eigen( A, w, V );
\end{lstlisting}

\begin{center}
Adapters may be more efficient and less memory consuming.
\end{center}
\end{frame}


%%%%%%%%%%%%%%%%%%%%%%%%%%%%%%%%%%%%%%%%%%%%%%%%%%%%%%%%%%%%%%%%%%%%%%%%%%%%%%
\section{Summary}
%%%%%%%%%%%%%%%%%%%%%%%%%%%%%%%%%%%%%%%%%%%%%%%%%%%%%%%%%%%%%%%%%%%%%%%%%%%%%%
\begin{frame}{Summary}
\begin{block}{After this lecture, you should know}
\begin{itemize}
\item Vectors and matrices
\item How to use Blaze for matrix and vector operations
\item How to compile a program using a external library
\end{itemize}
\end{block}
\end{frame}


%%%%%%%%%%%%%%%%%%%%%%%%%%%%%%%%%%%%%%%%%%%%%%%%%%%%%%%%%%%%%%%%%%%%%%%%%%%%%%
\section{References}
%%%%%%%%%%%%%%%%%%%%%%%%%%%%%%%%%%%%%%%%%%%%%%%%%%%%%%%%%%%%%%%%%%%%%%%%%%%%%%

\begin{frame}[t, allowframebreaks]
\frametitle{References}
\bibliographystyle{plain}
\bibliography{bib}
\end{frame}


\end{document}