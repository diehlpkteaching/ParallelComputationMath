\documentclass[12pt,t]{beamer}

% usepackage
%\usepackage{template/dbt}
\usepackage{listings}

\definecolor{comments}{RGB}{81,81,81}
\definecolor{keywords}{RGB}{255,0,90}

% lstlisting
\lstset{
    language=C,
    basicstyle=\footnotesize\ttfamily,
    keywordstyle=\color{keywords},
    showspaces=false,
    showstringspaces=false,
    commentstyle=\color{blue}\emph,
    frame=single,
    rulecolor=\color{comments},
    rulesepcolor=\color{comments},
    %backgroundcolor = \color{lightgray}
}

\usetheme{default}

\usepackage[
    type={CC},
    modifier={by-nc-nd},
    version={4.0},
]{doclicense} 

\newcommand{\courseurl}[0]{https://www.cct.lsu.edu/\string~pdiehl/teaching/2021/4997/}
\newcommand{\coursetimeline}[0]{https://www.cct.lsu.edu/~pdiehl/teaching/2021/4997/timeline.pdf}
\newcommand{\coursesyllabus}[0]{https://www.cct.lsu.edu/~pdiehl/teaching/2021/4997/syllabus.pdf}
\newcommand{\coursename}[0]{}
\newcommand{\coursemailinglist}[0]{https://mail.cct.lsu.edu/mailman/listinfo/par4997}
\newcommand{\coursesemester}{Fall 2023}
\newcommand{\courselocation}{TBA}
\newcommand{\coursedate}{Tuesday and Thursday, 09:00 to 10:20}
\newcommand{\courseserver}[0]{https://hpx-tutorial.cct.lsu.edu}
\newcommand{\coursetitle}[0]{Applications of high performance computing with C++ in biomedical science and engineering}




% frame slide
\title{\coursename}
\subtitle{Lecture 1: Introduction and Getting started}

\author{\href{}{}}
\institute {
    \href{}{\tt \scriptsize \today}
}
\date {
 \tiny \url{https://www.cct.lsu.edu/~pdiehl/teaching/2019/4977/}
\vspace{2cm}
\doclicenseThis  
  
}




\usepackage{fontspec}
\setmainfont{Raleway}

\begin{document} {
    \setbeamertemplate{footline}{}
    \frame {
        \titlepage
    }
}

\frame{

\tableofcontents

}

\AtBeginSection[]{
  \begin{frame}
  \vfill
  \centering
  \begin{beamercolorbox}[sep=8pt,center,shadow=true,rounded=true]{title}
    \usebeamerfont{title}\insertsectionhead\par%
  \end{beamercolorbox}
  \vfill
  \end{frame}
}


%%%%%%%%%%%%%%%%%%%%%%%%%%%%%%%%%%%%%%%%%%%%%%%%%%%%%%%%%%%%%%%%%%%%%%%%%%%%%%
\section{Administration/Organization}
%%%%%%%%%%%%%%%%%%%%%%%%%%%%%%%%%%%%%%%%%%%%%%%%%%%%%%%%%%%%%%%%%%%%%%%%%%%%%%

\begin{frame}{Important dates}

\begin{block}{Lectures}
Tuesday and Thursday, 09:00 to 10:20, 130 LCKT
\end{block}

\begin{block}{Grading}
\begin{itemize}
\item Homework 30\%
\item Project 20\%
\item Midterm exam 20\%
\item  Final exam 30\%
\end{itemize}

\end{block}

\begin{block}{Exams}
\begin{itemize}
\item Midterm: 15.10 during the lecture
\item Final: 12.10 from 12:30 to 2:30
\end{itemize}
\end{block}
\centering
More: \href{\coursesyllabus}{Syllabus} and \href{\coursetimeline}{Timeline}.
\end{frame}


\begin{frame}{Reading}

\begin{block}{Course's books}
\begin{itemize}
\item Andrew, Koenig. Accelerated C++: practical programming by example. Pearson Education India, 2000.
\item Stroustrup, Bjarne. Programming: principles and practice using C++. Pearson Education, 2014.
\end{itemize}
\end{block}

\begin{block}{Assistance C++ basics }
\begin{itemize}
\item Stroustrup, Bjarne. A Tour of C++. Addison-Wesley Professional, 2018.
\end{itemize}
\end{block}
\end{frame}

\begin{frame}{Submitting home work}

\begin{block}{Theory exercises}
At the beginning of the lecture in printed form
\end{block}

\begin{block}{Programming exercises}
We will use Github Class room for submission of the programming exercises and the course project. 

\end{block}
\end{frame}



%%%%%%%%%%%%%%%%%%%%%%%%%%%%%%%%%%%%%%%%%%%%%%%%%%%%%%%%%%%%%%%%%%%%%%%%%%%%%%
\section{Getting started}
%%%%%%%%%%%%%%%%%%%%%%%%%%%%%%%%%%%%%%%%%%%%%%%%%%%%%%%%%%%%%%%%%%%%%%%%%%%%%%

\begin{frame}{A small C++ program}

\lstinputlisting{code/lecture1-1.cpp}

\begin{block}{Compile}
\lstinputlisting[language=bash,firstline=2, lastline=2]{code/lecture1-1.sh}
\end{block}


\begin{block}{Run}
\lstinputlisting[language=bash,firstline=3, lastline=3]{code/lecture1-1.sh}
\end{block}


\end{frame}


\begin{frame}{Structure of a C++ program}

\lstinputlisting{code/lecture1-1.cpp}

\only<1>{
\begin{block}{Comments}
\begin{itemize}
\item A one line comment starts with $//$ 
\item A comment over multiple lines starts with $/*$  and ends with $*/$
\item Comments are important to understand the program, especially if the code is shared
\end{itemize}
\end{block}
}
\only<2>{
\begin{block}{Include directives}
\begin{itemize}
\item Is needed to include functionality of the C++ standard library, e.g. IO, which is not part of the core language
\item To include functionality of external libraries or structure your own code
\end{itemize}
\end{block}
}
\only<3>{
\begin{block}{Main function}
\begin{itemize}
\item Every C++ needs a function called main returning an integer value
\item Return zero means success and any other value indicates failure
\item When we execute any C++ program the main function is invoked and all instructions are executed
\end{itemize}
\end{block}
}
\only<4>{
\begin{block}{\lstinline|return| statement }
\begin{itemize}
\item The value of the return statement is passed to the program, which called the function
\item One function can have multiple return statements
\end{itemize}
\end{block}
}
\end{frame}

\section{Working with strings}


\begin{frame}{Reading strings}
\only<1>{
\lstinputlisting{code/lecture1-2.cpp}
}
\only<2>{
\lstinputlisting[firstline=3, lastline=3]{code/lecture1-2.cpp}
\lstinputlisting[firstline=9, lastline=9]{code/lecture1-2.cpp}
\begin{block}{Variables: Definition}
\begin{itemize}
\item Variables have a name (name) and a type (\lstinline|std:string|)
\item We need to include the string type, since it is not in the core language
\item We just defined the variable and currently it is a empty or null string
\end{itemize}
\end{block}
}
\only<3>{
\lstinputlisting[firstline=10, lastline=10]{code/lecture1-2.cpp}
\begin{block}{Variables: Initialization}
\begin{itemize}
\item Now we initialize the string by reading from \lstinline|std:cin| and assigning the value to it
\item The \lstinline|<<| operator writes a string to \lstinline|std:cout|
\item The \lstinline|>>| operator reads a string to \lstinline|std:cin|
\end{itemize}
\end{block}
Variables can be defined in three different ways:
\begin{itemize}
\item \lstinline|std::string = "Peter Pan";|
\item \lstinline|std::string; // empty string|
\item \lstinline|std::string stars(10,'*')| // string of three stars
\end{itemize}
\vspace{0.25cm}
More details: \tiny\url{https://en.cppreference.com/w/cpp/string/basic_string}
}


\end{frame}

\end{document}