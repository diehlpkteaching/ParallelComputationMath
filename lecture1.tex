%Lecture 1
%Copyright (C) 2019  Patrick Diehl
%
%This program is free software: you can redistribute it and/or modify
%it under the terms of the GNU General Public License as published by
%the Free Software Foundation, either version 3 of the License, or
%(at your option) any later version.

%This program is distributed in the hope that it will be useful,
%but WITHOUT ANY WARRANTY; without even the implied warranty of
%MERCHANTABILITY or FITNESS FOR A PARTICULAR PURPOSE.  See the
%GNU General Public License for more details.

%You should have received a copy of the GNU General Public License
%along with this program.  If not, see <http://www.gnu.org/licenses/>.
\documentclass[12pt]{beamer}

\beamertemplatenavigationsymbolsempty
% usepackage
\usepackage{template/dbt}
%\usepackage{listings}

\definecolor{comments}{RGB}{81,81,81}
\definecolor{keywords}{RGB}{255,0,90}

\usepackage{lstautodedent}
% lstlisting
\lstset{
    language=C,
    basicstyle=\footnotesize\ttfamily,
    keywordstyle=\color{keywords},
    showspaces=false,
    showstringspaces=false,
    commentstyle=\color{blue}\emph,
	autodedent
    %frame=single,
    %rulecolor=\color{comments},
    %rulesepcolor=\color{comments},
    %backgroundcolor = \color{lightgray}
}

%\usetheme{default}

\usepackage[
    type={CC},
    modifier={by-nc-nd},
    version={4.0},
]{doclicense} 

\newcommand{\courseurl}[0]{https://www.cct.lsu.edu/\string~pdiehl/teaching/2021/4997/}
\newcommand{\coursetimeline}[0]{https://www.cct.lsu.edu/~pdiehl/teaching/2021/4997/timeline.pdf}
\newcommand{\coursesyllabus}[0]{https://www.cct.lsu.edu/~pdiehl/teaching/2021/4997/syllabus.pdf}
\newcommand{\coursename}[0]{}
\newcommand{\coursemailinglist}[0]{https://mail.cct.lsu.edu/mailman/listinfo/par4997}
\newcommand{\coursesemester}{Fall 2023}
\newcommand{\courselocation}{TBA}
\newcommand{\coursedate}{Tuesday and Thursday, 09:00 to 10:20}
\newcommand{\courseserver}[0]{https://hpx-tutorial.cct.lsu.edu}
\newcommand{\coursetitle}[0]{Applications of high performance computing with C++ in biomedical science and engineering}




% frame slide
\title{\coursename}
\subtitle{Lecture 1: Introduction and Getting started}

%\author{\href{}{}}
%\institute {
%    \href{}{\tt \scriptsize \today}
%}
\date {
 \tiny \url{\courseurl}
\vspace{2cm}
\doclicenseThis  
  
}



\usepackage{ifxetex}

\ifxetex
\usepackage{fontspec}
\setmainfont{Raleway}
\fi

\ifluatex
\usepackage{fontspec}
\setmainfont{Raleway}
\fi

\begin{document} {
    \setbeamertemplate{footline}{}
    \frame {
        \titlepage
    }
}

\frame{

\tableofcontents

}

\AtBeginSection[]{
  \begin{frame}
  \vfill
  \centering
  \begin{beamercolorbox}[sep=8pt,center,shadow=true,rounded=true]{title}
    \usebeamerfont{title}\insertsectionhead\par%
  \end{beamercolorbox}
  \vfill
  \end{frame}
}


%%%%%%%%%%%%%%%%%%%%%%%%%%%%%%%%%%%%%%%%%%%%%%%%%%%%%%%%%%%%%%%%%%%%%%%%%%%%%%
\section{Administration/Organization}
%%%%%%%%%%%%%%%%%%%%%%%%%%%%%%%%%%%%%%%%%%%%%%%%%%%%%%%%%%%%%%%%%%%%%%%%%%%%%%

\begin{frame}{Important dates}

\begin{block}{Lectures}
Tuesday and Thursday, 09:00 to 10:20, 130 LCKT
\end{block}

\begin{block}{Grading}
\begin{itemize}
\item Homework 30\%
\item Project 20\%
\item Midterm exam 20\%
\item  Final exam 30\%
\end{itemize}

\end{block}

\begin{block}{Exams}
\begin{itemize}
\item Midterm exam: 13.10 during lecture
\item Final exams: 10.12 from 12:30 to 2:30
\end{itemize}
\end{block}
\centering
More: \href{\coursesyllabus}{Syllabus} and \href{\coursetimeline}{Timeline}.
\end{frame}


\begin{frame}{Reading}

\begin{block}{Course's books}
\begin{itemize}
\item Andrew, Koenig. Accelerated C++: practical programming by example. Pearson Education India, 2000.
\item Stroustrup, Bjarne. Programming: principles and practice using C++. Pearson Education, 2014.
\end{itemize}
\end{block}

\begin{block}{Assistance C++ basics }
\begin{itemize}
\item Stroustrup, Bjarne. A Tour of C++. Addison-Wesley Professional, 2018.
\item O'Dwyer, Arthur. Mastering the C++17 STL. Packt Publishing Ltd; 2017.
\end{itemize}
\end{block}
\end{frame}

\begin{frame}{Submitting home work}

\begin{block}{Theory exercises}
At the beginning of the lecture in printed form
\end{block}

\begin{block}{Programming exercises}
\begin{itemize}
\item Github Classroom\footnote{\tiny\url{https://www.diehlpk.de/blog/githubclassroom/}} for submission of the programming exercises and the course project.
\item Juypter Server\footnote{\tiny\url{https://tutorial.cct.lsu.edu/hpx}} to work in your browser on the exercises and course project\footnote{\tiny\url{https://www.diehlpk.de/blog/jupyter-notebooks/}}.
\end{itemize}
\end{block}
Note that we use these tools the first time for this course. We anticipate to do a short survey at the end of the semester.
\end{frame}

\begin{frame}{}

\end{frame}



%%%%%%%%%%%%%%%%%%%%%%%%%%%%%%%%%%%%%%%%%%%%%%%%%%%%%%%%%%%%%%%%%%%%%%%%%%%%%%
\section{Getting started}
%%%%%%%%%%%%%%%%%%%%%%%%%%%%%%%%%%%%%%%%%%%%%%%%%%%%%%%%%%%%%%%%%%%%%%%%%%%%%%

\begin{frame}{A small C++ program}

\lstinputlisting{code/01/lecture1-1.cpp}

\begin{block}{Compile}
\lstinputlisting[language=bash,firstline=2, lastline=2]{code/01/lecture1-1.sh}
\end{block}


\begin{block}{Run}
\lstinputlisting[language=bash,firstline=3, lastline=3]{code/01/lecture1-1.sh}
\end{block}


\end{frame}


\begin{frame}{Structure of a C++ program}

\lstinputlisting{code/01/lecture1-1.cpp}

\only<1>{
\begin{block}{Comments}
\begin{itemize}
\item A one line comment starts with $//$ 
\item A comment over multiple lines starts with $/*$  and ends with $*/$
\item Comments are important to understand the program, especially if the code is shared
\end{itemize}
\end{block}
}
\only<2>{
\begin{block}{Include directives}
\begin{itemize}
\item Is needed to include functionality of the C++ standard library, e.g. IO, which is not part of the core language
\item To include functionality of external libraries or structure your own code
\end{itemize}
\end{block}
}
\only<3>{
\begin{block}{Main function}
\begin{itemize}
\item Every C++ needs a function called main returning an integer value
\item Return zero means success and any other value indicates failure
\item When we execute any C++ program the main function is invoked and all instructions are executed
\end{itemize}
\end{block}
}
\only<4>{
\begin{block}{\lstinline|return| statement }
\begin{itemize}
\item The value of the return statement is passed to the program, which called the function
\item One function can have multiple return statements
\end{itemize}
\end{block}
}
\end{frame}

%%%%%%%%%%%%%%%%%%%%%%%%%%%%%%%%%%%%%%%%%%%%%%%%%%%%%%%%%%
\section{Working with strings}
%%%%%%%%%%%%%%%%%%%%%%%%%%%%%%%%%%%%%%%%%%%%%%%%%%%%%%%%%%

\begin{frame}{Reading strings}
\only<1>{
\lstinputlisting{code/01/lecture1-2.cpp}
}
\only<2>{
\lstinputlisting[firstline=3, lastline=3]{code/01/lecture1-2.cpp}
\lstinputlisting[firstline=9, lastline=9]{code/01/lecture1-2.cpp}
\begin{block}{Variables: Definition}
\begin{itemize}
\item Variables have a name (name) and a type (\lstinline|std::string|)
\item We need to include the string type, since it is not in the core language
\item We just defined the variable and currently it is a empty or null string
\end{itemize}
\end{block}
}
\only<3>{
\lstinputlisting[firstline=10, lastline=10]{code/01/lecture1-2.cpp}
\begin{block}{Variables: Initialization}
\begin{itemize}
\item Now we initialize the string by reading from \lstinline|std::cin| and assigning the value to it
\item The \lstinline|<<| operator writes a string to \lstinline|std::cout|
\item The \lstinline|>>| operator reads a string to \lstinline|std::cin|
\end{itemize}
\end{block}
Variables can be defined in three different ways:
\begin{itemize}
\item \lstinline|std::string = "Peter Pan";|
\item \lstinline|std::string; // empty string|
\item \lstinline|std::string stars(3,'*')| // string of three stars
\end{itemize}
\vspace{0.25cm}
More details: \tiny\url{https://en.cppreference.com/w/cpp/string/basic_string}
}


\end{frame}

\begin{frame}[fragile]{More functionality of strings }
\lstinputlisting[firstline=12, lastline=12]{code/01/lecture1-3.cpp}
\begin{block}{Concatenation}
\lstinline|+| operator combines string
\end{block}
\begin{block}{Defining constants}
\lstinline|const| operator to make the promise that we will not change the value later
\end{block}
\begin{lstlisting}
const size_t length = greeting .size ();
\end{lstlisting}
\begin{block}{Getting the size}
\lstinline|.size()| operator to get the string's size
\end{block}
\end{frame}


%%%%%%%%%%%%%%%%%%%%%%%%%%%%%%%%%%%%%%%%%%%%%%%%%%%%%%%%%%
\section{Looping and counting}
%%%%%%%%%%%%%%%%%%%%%%%%%%%%%%%%%%%%%%%%%%%%%%%%%%%%%%%%%%


\begin{frame}[fragile]{Using loops and counting}

\only<1>{

\begin{block}{Compute the sum of $1,\ldots,n$}
\begin{align*}
\sum\limits_{i=1}^n i
\end{align*}
\end{block}

}

\begin{block}{Using the loop statement\footnote{\tiny\url{https://en.cppreference.com/w/cpp/language/for}}}

\begin{lstlisting}
size_t result = 0;
for(size_t i = 1; r != 5; i++){
	result = result + i
}
\end{lstlisting}


\only<2>{

\begin{block}{Condition}
\begin{itemize}
\item The variable i is only available inside the loop's body
\item The loop will execute the statements in the curly braces until i is equal to 5
\item The value of i is incremented after all statements are executed
\item \lstinline|i++| is equivalent to \lstinline|i = i+1|
\end{itemize}
\end{block}

}

\end{block}

\end{frame}



\begin{frame}[fragile]{The while statement\footnote{\tiny\url{https://en.cppreference.com/w/cpp/language/while}}}
\begin{lstlisting}
size_t result = 0;
size_t i = 1;
while (i != 5 ) {

	result += i;
	i++;
}
\end{lstlisting}
\only<1>{
\begin{block}{Condition}
\begin{itemize}
\item \lstinline|i != 5| the statement within the curly braces will be repeated five times
\item \lstinline|!=| is the inequality operator and once \lstinline|i| is equal to \lstinline|5| the loop stops
\end{itemize}

\end{block}

}
\end{frame}




\begin{frame}[fragile]{Conditionals\footnote{\tiny\url{https://en.cppreference.com/w/cpp/language/if}}}

\only<1>{
\begin{block}{Compute the sum of $f(i)$ for $i=1,\ldots,n$}
\begin{align*}
\sum\limits_{i=1}^n f(i) \text{ with } f(i) = 
\begin{cases}
i, \text{ if } i is even \\
i^2. else
\end{cases}
\end{align*}
\end{block}
}

\begin{lstlisting}
size_t result = 0;
for(size_t i = 1; r != 5; i++){
	if(	i % 0 == 0)
		result = result + i
	else
		result = result + i * i
}

\end{lstlisting}

\only<2>{
\begin{block}{if statement}
\begin{itemize}
\item If the condition is \lstinline|true| the statements in the \lstinline|if| branch are executed
\item If the condition is \lstinline|false| the statements in the \lstinline|else| branch are executed
\end{itemize}
\end{block}
\begin{block}{Logical operator}
\begin{itemize}
\item \lstinline|\%| Modulo operator for integers
\end{itemize}
\end{block}
}
\end{frame}

\begin{frame}{Operators\footnote{\tiny\url{https://en.cppreference.com/w/cpp/language/operator_precedence}}}
\begin{block}{Logical operators}
\begin{itemize}
\item \lstinline|&&| Logical and
\item \lstinline|||| Logial or
\item \lstinline|!x| Logical negation 
\end{itemize}
\end{block}
\begin{block}{Comparison operators}
\begin{itemize}
\item \lstinline|==|  Compares to equal
\item \lstinline|!=| Compares to unequal
\item \lstinline|<| Compares to be less
\item \lstinline|>| Compares to be higher
\item \lstinline|<=| Compares to be less or equal
\item \lstinline|>=| Compares to be higher or equal
\end{itemize}
\end{block}
\end{frame}

\begin{frame}{Built-in types\footnote{\tiny\url{https://en.cppreference.com/w/cpp/language/types}}}
\begin{block}{Integer types}
\begin{itemize}
\item \lstinline|bool| Representation of truth values: \lstinline|true| or \lstinline|false| 
\item \lstinline|unsigned| Integral type for non-negative values only 
\item \lstinline|short| Integral type that must hold at least 32 bits
\item \lstinline|long| Integral type that must hold at least 64 bits
\item \lstinline|size_t| Unsigned Integral type
\end{itemize}
\end{block}

\begin{block}{Floating points}
\begin{itemize}
\item \lstinline|float| Single precision floating point type
\item \lstinline|double| Double precision floating point type
\item \lstinline|long double| Extended precision floating point type
\end{itemize}
\end{block}

\end{frame}

%%%%%%%%%%%%%%%%%%%%%%%%%%%%%%%%%%%%%%%%%%%%%%%%%%%%%%%%%%%%%%%%%%%%%%%%%%%%%%
\section{Summary}
%%%%%%%%%%%%%%%%%%%%%%%%%%%%%%%%%%%%%%%%%%%%%%%%%%%%%%%%%%%%%%%%%%%%%%%%%%%%%%

\begin{frame}{Summary}
\begin{block}{After this lecture, you should know}
\begin{itemize}
\item Structure of a C++ program
\item Handling strings
\item Loops and counting
\item Conditionals
\item Operators
\item Built-in types
\end{itemize}

\end{block}


\end{frame}

\end{document}