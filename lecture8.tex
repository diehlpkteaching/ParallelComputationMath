%Template
%Copyright (C) 2019  Patrick Diehl
%
%This program is free software: you can redistribute it and/or modify
%it under the terms of the GNU General Public License as published by
%the Free Software Foundation, either version 3 of the License, or
%(at your option) any later version.

%This program is distributed in the hope that it will be useful,
%but WITHOUT ANY WARRANTY; without even the implied warranty of
%MERCHANTABILITY or FITNESS FOR A PARTICULAR PURPOSE.  See the
%GNU General Public License for more details.

%You should have received a copy of the GNU General Public License
%along with this program.  If not, see <http://www.gnu.org/licenses/>.
\documentclass[12pt,t]{beamer}

\beamertemplatenavigationsymbolsempty

% usepackage
%\usepackage{template/dbt}
\usepackage{listings}

\usepackage{tikz}
\usetikzlibrary{positioning,arrows}
\tikzset{
 arrow/.style={-latex, shorten >=1ex, shorten <=1ex}
 }
\usepackage{mathtools}
\usepackage{pgfplots}
\pgfplotsset{compat=1.12}

\usepackage{amsmath}
\definecolor{comments}{RGB}{81,81,81}
\definecolor{keywords}{RGB}{255,0,90}

% lstlisting
\lstset{
    language=C,
    basicstyle=\footnotesize\ttfamily,
    keywordstyle=\color{keywords},
    showspaces=false,
    showstringspaces=false,
    commentstyle=\color{blue}\emph
    %frame=single,
    %rulecolor=\color{comments},
    %rulesepcolor=\color{comments},
    %backgroundcolor = \color{lightgray}
}

\usetheme{default}

\usepackage[
    type={CC},
    modifier={by-nc-nd},
    version={4.0},
]{doclicense} 

\newcommand{\courseurl}[0]{https://www.cct.lsu.edu/\string~pdiehl/teaching/2021/4997/}
\newcommand{\coursetimeline}[0]{https://www.cct.lsu.edu/~pdiehl/teaching/2021/4997/timeline.pdf}
\newcommand{\coursesyllabus}[0]{https://www.cct.lsu.edu/~pdiehl/teaching/2021/4997/syllabus.pdf}
\newcommand{\coursename}[0]{}
\newcommand{\coursemailinglist}[0]{https://mail.cct.lsu.edu/mailman/listinfo/par4997}
\newcommand{\coursesemester}{Fall 2023}
\newcommand{\courselocation}{TBA}
\newcommand{\coursedate}{Tuesday and Thursday, 09:00 to 10:20}
\newcommand{\courseserver}[0]{https://hpx-tutorial.cct.lsu.edu}
\newcommand{\coursetitle}[0]{Applications of high performance computing with C++ in biomedical science and engineering}




% frame slide
\title{\coursename}
\subtitle{Lecture 8: Introduction to bond-based peridynamics}

%\author{\href{}{}}
%\institute {
%    \href{}{\tt \scriptsize \today}
%}
\date {
 \tiny \url{\courseurl}
\vspace{2cm}
\doclicenseThis  
  
}



\usepackage{ifxetex}

\ifxetex
\usepackage{fontspec}
\setmainfont{Raleway}
\fi

\ifluatex
\usepackage{fontspec}
\setmainfont{Raleway}
\fi


\newcommand{\R}{\mathbb{R}}
\newcommand{\ftime}{[0,T]}
\newcommand{\args}{(t,x(t,X')-x(t,X),X'-X)}
\newcommand{\argsd}{(t,x(t,X_j)-x(t,X_i),X_j-X_i)}
\newcommand{\x}{\Vert X'-X \Vert}
\newcommand{\xd}{\Vert X_j-X_i \Vert}
\newcommand{\J}{J\left(\frac{\x}{\epsilon}\right)}
\newcommand{\dx}{{\Delta x}}
\newcommand{\ratio}{\sfrac{\delta}{\dx}}

%Sparse grid stuff
\newcommand{\Model}{{\mathcal{M}}}
\newcommand{\vl}{{\vec{l}}}
\newcommand{\vi}{{\vec{i}}}
\newcommand{\vli}{{\vl, \vi}}
\newcommand{\vx}{{\mathbf{z}}}
\newcommand{\unitd}{{[0, 1]^d}}
\newcommand{\Real}{{\mathbb{R}}}
\newcommand{\Order}{{\mathcal{O}}}
\newcommand{\Il}{{\mathcal{I}_\vl}}
\newcommand{\level}{{\ell}}
\newcommand{\cV}{{\mathcal{V}}}
\newcommand{\cW}{{\mathcal{W}}}



\begin{document} {
    \setbeamertemplate{footline}{}
    \frame {
        \titlepage
    }
}

\frame{

\tableofcontents

}

\AtBeginSection[]{
  \begin{frame}
  \vfill
  \centering
  \begin{beamercolorbox}[sep=8pt,center,shadow=true,rounded=true]{title}
    \usebeamerfont{title}\insertsectionhead\par%
  \end{beamercolorbox}
  \vfill
  \end{frame}
}

%%%%%%%%%%%%%%%%%%%%%%%%%%%%%%%%%%%%%%%%%%%%%%%%%%%%%%%%%%%%%%%%%%%%%%%%%%%%%%
\section{Reminder}
%%%%%%%%%%%%%%%%%%%%%%%%%%%%%%%%%%%%%%%%%%%%%%%%%%%%%%%%%%%%%%%%%%%%%%%%%%%%%%
\begin{frame}{Lecture 8}
\begin{block}{What you should know from last lecture}
\begin{itemize}
\item Lambda functions
\item Asynchronous programming
\end{itemize}
\end{block}
\end{frame}


%%%%%%%%%%%%%%%%%%%%%%%%%%%%%%%%%%%%%%%%%%%%%%%%%%%%%%%%%%%%%%%%%%%%%%%%%%%%%%
\section{Classical continuum mechanics}
%%%%%%%%%%%%%%%%%%%%%%%%%%%%%%%%%%%%%%%%%%%%%%%%%%%%%%%%%%%%%%%%%%%%%%%%%%%%%%

\begin{frame}{}

\begin{figure}[!htbp]
\begin{tikzpicture}
\draw[fill=blue, opacity=0.5](1,1) circle (1);
 \node at (1,2.35) {{\small $\Omega_0$}};
 \draw[fill=blue,](1,1) circle (0.075);
 \node at (1,0.725) {{\small $X$}};
 \node at (6,1) {\includegraphics[scale=1.25]{images/deformation.pdf}};
 \node at (6,2.35) {{\small $\Omega(t)$}};
 \draw[fill=blue,](6,0.725) circle (0.075);
 \node at (6,0.45) {{\small $x(t,X)$}};
 \node at (3.75,2.45) {{\small $\phi:\Omega_0\rightarrow \R^3$}};
 \draw [arrow, bend angle=45, bend left] (1.95,1.25) to (5.8,1);
\end{tikzpicture}
\caption[The continuum in the reference configuration $\Omega_0$ and after the deformation $\phi : \Omega_0 \rightarrow \R^3$ in the current configuration $\Omega(t)$ at time $t$.]{The continuum in the reference configuration $\Omega_0$ and after the deformation $\phi : \Omega_0 \rightarrow \R^3$ with $\det(\text{grad}\;\phi) > 0$ in the current configuration $\Omega(t)$ at time $t$.}
\label{fig::chapter2:01}
\end{figure}

\begin{block}{Prerequisites}
\begin{itemize}
\item  A \index{material point}{material point} in the continuum is identified with its position $X \in \R^3$ in the so-called reference configuration $\Omega_0 \subset \R^3$.
\item The \index{configuration! reference}{reference configuration} $\Omega_0$ is refers to the shape of the continuum at rest with no internal forces. 
\end{itemize}

\end{block}

\end{frame}

\begin{frame}{}
\begin{block}{Prerequisites}
\begin{itemize}
\item  The deformation $\phi:[0,T]\times\R^3\rightarrow\R^3$ of a material point $X$ in the reference configuration $\Omega_0$ to the so-called \index{configuration! current}{current configuration} $\Omega(t)$ is given by
\begin{align*}
\phi(t,X) := id(X) + u(t,X) = x(t,X)
\end{align*}
\item where $u:[0,T]\times\R^3\rightarrow\R^3$ refers to the \index{displacement}{displacement}
\begin{align*}
u(t,X):= x(t,X) - X\,\text{.}
\end{align*}
\item The stretch $s:[0,T]\times\R^3\times\R^3\rightarrow\R^3$ between the material point $X$ and the material point $X'$ after the deformation $\phi$ in the configuration $\Omega(t)$ is defined by
\begin{align*}
s(t,X,X') := \phi(t,X') - \phi(t,X) \,\text{.}
\end{align*}
\end{itemize}

\end{block}

\end{frame}

\begin{frame}{Notice}
We just covered the prerequisites of classical continuum mechanics which are necessary to introduce the peridyanmic theory. For more details, we refer to \\
\begin{itemize}
\item Liu, I-Shih. Continuum mechanics. Springer Science \& Business Media, 2013. 
\item Gurtin, Morton E. An introduction to continuum mechanics. Vol. 158. Academic press, 1982.
\end{itemize}
\end{frame}

%%%%%%%%%%%%%%%%%%%%%%%%%%%%%%%%%%%%%%%%%%%%%%%%%%%%%%%%%%%%%%%%%%%%%%%%%%%%%%
\section{Peridyanmics}
%%%%%%%%%%%%%%%%%%%%%%%%%%%%%%%%%%%%%%%%%%%%%%%%%%%%%%%%%%%%%%%%%%%%%%%%%%%%%%

\begin{frame}{What is peridynamics}

\begin{itemize}
\item A non-local generalization of continuum mechanics 
\item Has a focus on discontinuous displacement as they arise in fracture mechanics.
\item Models crack and fractures on a mesoscopic scale using Newton's second law (force equals mass times acceleration)
\begin{align*}
F = m \cdot a = m \cdot \ddot X
\end{align*}
\end{itemize}
\small
\begin{itemize}
\item Silling, Stewart A. "Reformulation of elasticity theory for discontinuities and long-range forces." Journal of the Mechanics and Physics of Solids 48.1 (2000): 175-209.
\item Silling, Stewart A., and Ebrahim Askari. "A meshfree method based on the peridynamic model of solid mechanics." Computers \& structures 83.17-18 (2005): 1526-1535.
\end{itemize}
\end{frame}

\begin{frame}{Principle I}
\begin{figure}[!htbp]
\begin{tikzpicture}
\draw[fill=blue, opacity=0.5](0,0) circle (1);
 \node at (0,2.35) {{\small $\Omega_0$}};
 \draw[fill=blue](0,0) circle (2pt);
 \node at (0,-0.25) {{\small $X$}};
 \draw (0,0) ellipse (3cm and 2cm);
\draw [->] (0,0) to (0.95,0.225);
\node at (1.125,0.25) {{\tiny $\delta$}};
\node at (1.1,0.8) {{\tiny $B_\delta(X)$}};
\draw (0,0.5) circle (2pt);
\draw (0,-0.5) circle (2pt);
\draw (0.5,0.0) circle (2pt);
\draw (-0.5,0.0) circle (2pt);
\end{tikzpicture}
\caption{The continuum in the reference configuration $\Omega_0$ and the interaction zone $B_\delta(X)$ for material point $X$ with the horizon $\delta$.}
\label{fig::chapter2:02}
\end{figure}

\end{frame}

\begin{frame}{Principle II}

\begin{block}{Acceleration $a:\ftime\times\R^3\rightarrow\R^3$}
of a material point at position $X$ at time $t$ is given by
\begin{align*}
\rho(X)a(t,X)&:= \\
&\int\limits_{B_\delta(X)} f\left(t,x(t,X')-x(t,X), X'-X\right)dX' + b(t,X)\,\text{,}
\label{eq::chapter2:01}
\end{align*}
\end{block}
where $f:[0,T]\times\R^3\times\R^3\rightarrow\R^3$ denotes a \index{pair-wise force function}{pair-wise force function}, $\rho(X)$ is the mass density and $b:[0,T]\times\mathbb{R}^3\rightarrow\mathbb{R}^3$ the external force. 
\end{frame}


\begin{frame}{Important fundamental assumptions}
\begin{enumerate}
\item The medium is continuous (equal to a continuous mass density field exists)
\item Internal forces are contact forces (equal to that material points only interact if they are separated by zero distance.
\item Conservation laws of mechanics apply (conservation of mass, linear momentum, and angular momentum).
\end{enumerate}

\begin{block}{Conservation of linear momentum}
\vspace*{-0.5cm}
\begin{align*}
&f(t,-(x(t,X')-x(t,X)),-(X'-X))= \\ &-f(t,x(t,X')-x(t,X), X'-X)
\end{align*}

\end{block}
\vspace*{-0.5cm}
\begin{block}{Conservation of angular momentum}
\vspace*{-0.5cm}
\begin{align*}
(x(t,X')-x(t,X)+X'-X) \times f\left(t,x(t,X')-x(t,X), X'-X\right) = 0
\end{align*}
\end{block}

\end{frame}

%%%%%%%%%%%%%%%%%%%%%%%%%%%%%%%%%%%%%%%%%%%%%%%%%%%%%%%%%%%%%%%%%%%%%%%%%%%%%%
\section{Discretization}
%%%%%%%%%%%%%%%%%%%%%%%%%%%%%%%%%%%%%%%%%%%%%%%%%%%%%%%%%%%%%%%%%%%%%%%%%%%%%%

\begin{frame}{EMU nodal discretization (EMU ND)}

\begin{block}{Assumptions}
\begin{itemize}
\item All material points $X$ are placed at the nodes $\mathbf{X}:=\lbrace X_i \in \mathbb{R}^3\vert i=1,\ldots,n\rbrace$ of a regular grid in the reference configuration $\Omega_0$.
\item The discrete \index{nodal spacing}{nodal spacing} $\dx$ between $X_i$ and $X_j$ is defined as $\dx = \Vert X_j - X_i \Vert$.
\item  The discrete interaction zone $B_\delta(X_i)$ of $X_i$ is given by $B_\delta(X_i):=\lbrace X_j \vert \,\vert\vert X_j-X_i\vert\vert \leq \delta\rbrace$.
\item For all material points at the nodes $\mathbf{X}:=\lbrace X_i \in \mathbb{R}^3\vert i=1,\ldots,n\rbrace$ a surrounding volume $\mathbf{V}:=\lbrace\ \mathbf{V}_i \in \mathbb{R}\vert i=1,\ldots,n\rbrace$ is assumed.
\item These volumes are non overlapping $\mathbf{V}_i \cap \mathbf{V}_j = \emptyset$ and recover the volume of the volume of the reference configuration $\sum_{i=1}^n \mathbf{V}_i = \mathbf{V}_{\Omega_0}$.
\end{itemize}
\end{block}
\end{frame}

\begin{frame}{Discrete equation of motion}
\begin{center}
\begin{tikzpicture}
\shade[ball color = blue] (2,2) circle(1pt);
\shade[ball color = green] (1,1.5) circle(1pt);
\shade[ball color = green] (1,2.0) circle(1pt);
\shade[ball color = green] (1,2.5) circle(1pt);
\shade[ball color = green] (3,1.5) circle(1pt);
\shade[ball color = green] (3,2.0) circle(1pt);
\shade[ball color = green] (3,2.5) circle(1pt);
\shade[ball color = green] (1.5,1) circle(1pt);
\shade[ball color = green] (2.0,1) circle(1pt);
\shade[ball color = green] (2.5,1.) circle(1pt);
\shade[ball color = green] (1.5,3) circle(1pt);
\shade[ball color = green] (2.0,3) circle(1pt);
\shade[ball color = green] (2.5,3) circle(1pt);
 \foreach \x in {0.5,3.5}
 \foreach \y in {0.5,1,1.5,2,2.5,3,3.5}
 \shade[ball color = gray](\x,\y) circle (1pt);
 \foreach \x in {1,1.5,2,2.5,3}
 \foreach \y in {0.5,3.5}
 \shade[ball color = gray](\x,\y) circle (1pt);
 \foreach \x in {1.5,2,2.5}
 \foreach \y in {1.5,2.5}
 \shade[ball color = gray](\x,\y) circle (1pt);
 \foreach \x in {1,3}
 \foreach \y in {1,3}
 \shade[ball color = gray](\x,\y) circle (1pt);
 \shade[ball color = gray](1.5,2) circle (1pt);
 \shade[ball color = gray](2.5,2) circle (1pt);
 \foreach \x in {0.5,1,1.5,2,2.5,3,3.5,4.0}
 \draw(\x-0.25,0.25) -- (\x-0.25,4-0.25);
 \foreach \x in {0.5,1,1.5,2,2.5,3,3.5,4.0}
 \draw(0.25,\x-0.25) -- (4-0.25,\x-0.25);
 \draw(2,2) circle(30pt);
\node at (2.05,1.85) {\tiny{$X_i$}};
\end{tikzpicture}
\end{center}

\begin{align*}
\rho(X_i)a(t,X_i)&=\sum\limits_{X_j\in B_\delta(X_i)} \\ &f\left(t,x(t,X_j)-x(t,X_i), X_j-X_i\right)d\mathbf{V}_j + b(t,X_i)
\end{align*}

\end{frame}

\begin{frame}{}
Note that we computed the acceleration of a material point $a(t,X)$ and we need to compute the displacement $u(t,X)$ by a
\begin{block}{Central difference scheme}
\begin{align*}
u&(t+1,X) = \\  2 u(t,X) &- u(t-1,X) + \Delta t^2 \left(\sum\limits_{X_j\in B_\delta(X_i)} f(t,X_i,X_j)+b(t,X)\right)
\end{align*}
\end{block}
to compute the actual displacement $x(t,X):= id(X)+ u(t,X)$.

\end{frame}

%%%%%%%%%%%%%%%%%%%%%%%%%%%%%%%%%%%%%%%%%%%%%%%%%%%%%%%%%%%%%%%%%%%%%%%%%%%%%%
\section{Material models}
%%%%%%%%%%%%%%%%%%%%%%%%%%%%%%%%%%%%%%%%%%%%%%%%%%%%%%%%%%%%%%%%%%%%%%%%%%%%%%

\begin{frame}{Prototype Microelastic Brittle (PMB) model}
In this model the assumption is made that the pair-wise force $f$ only depends on the relative normalized bond stretch $s:[0,T]\times\mathbb{R}^3\times\mathbb{R}^3\rightarrow\mathbb{R}$
\begin{align*}
s(t,x(t,X')&-x(t,X),X'-X):= \\ &\frac{\vert\vert x(t,X')-x(t,X))\vert\vert - \vert\vert X'-X\vert\vert}{\vert\vert X'-X\vert\vert}\,\text{.} 
\end{align*}
where 
\begin{itemize}
\item X'-X is the vector between the material points in the reference configuration,
\item x(t,X')-x(t,X) is the vector between the material point in the current configuration.
\end{itemize}
\end{frame}


\begin{frame}{Pair-wise bond force $f$}

\begin{align*}
f(t,x(t,X')&-x(t,X),X'-X):= \notag\\ & \textcolor{blue}{c} \, s(t,x(t,X')-x(t,X),X'-X)\frac{x(t,X')-x(t,X)}{\Vert x(t,X')-x(t,X)\Vert}
\end{align*}
with a material dependent \index{stiffness constant}{stiffness constant} $\textcolor{blue}{c}$.

\begin{block}{More details:}
\small
\begin{itemize}
\item  Silling, Stewart A., and Ebrahim Askari. "A meshfree method based on the peridynamic model of solid mechanics." Computers \& structures 83.17-18 (2005): 1526-1535.
\item Parks, Michael L., et al. "Implementing peridynamics within a molecular dynamics code." Computer Physics Communications 179.11 (2008): 777-783.
\end{itemize}

\end{block}
\end{frame}

\begin{frame}{}
\begin{figure}[H]
\begin{tikzpicture}
 \begin{axis}[xlabel=$s$, axis lines=middle,ticks=none ,grid=major,ymax=0.75, ylabel=$f$]
 \addplot[domain=0:1,thick] {0.5 * x};
 \draw[gray,thick] (0.4,0.2) -- (0.6,0.2);
 \draw[gray,thick] (0.6,0.2) -- (0.6,0.3);
 \node at (0.525,0.225) {{\small \textcolor{blue}{$c$}}};
 \end{axis}
\end{tikzpicture}
\caption[Sketch of the pair-wise linear valued force function $f$ with the stiffness constant $c$ as slope.]{Sketch of the pair-wise linear valued force function $f$ with the stiffness constant \textcolor{blue}{$c$} as slope.}
\label{fig::force::sketch}
\end{figure}
\begin{center}
\textcolor{blue}{Note that there is no notion of failure/damage in the current material model.}
\end{center}
\end{frame}

\begin{frame}{Introducing failure}
Introduce a scalar valued history dependent function $\mu:[0,T]\times\mathbb{R}^3\times\mathbb{R}^3\rightarrow\mathbb{N}$ to the computation of the pair-wise force
\begin{align*}
&f(t,x(t,X')-x(t,X),X'-X):= \notag\\ & \textcolor{blue}{c} s(t,x(t,X')-x(t,X),X'-X) \\&\mu\args \frac{x(t,X')-x(t,X)}{\Vert x(t,X')-x(t,X)\Vert}\,\text{.} 
\end{align*}
with
\begin{align}
\mu(t,x(t,X')&-x(t,X),X'-X):= \\&
 \left\{
 \begin{aligned}
 & 1 \quad s(t,x(t,X')-x(t,X),X'-X) < \textcolor{blue}{s_{c}} \\
 %& \qquad -\alpha s_\text{smin}(t') \forall 0 \leq t' \leq t
%\\[1ex]
 & 0 \quad \text{otherwise}
\end{aligned}
 \right.
\end{align}
\end{frame}

\begin{frame}{}
\begin{figure}[H]
\begin{tikzpicture}
 \begin{axis}[xlabel=$s$, axis lines=middle ,grid=major,ymax=0.75, ylabel=$f$,xtick={1}, xticklabels={\textcolor{blue}{$s_c$}},yticklabels={,,}]
 \addplot[domain=0:1,thick] {0.5 * x};
 \addplot[domain=1:1.5,thick] {0 * x};
 \draw[gray,thick] (0.4,0.2) -- (0.6,0.2);
 \draw[gray,thick] (0.6,0.2) -- (0.6,0.3);
 \draw[gray,thick,dashed] (1,0) -- (1,0.5);
 \node at (0.525,0.225) {{\small \textcolor{blue}{$c$}}};
 \end{axis}
\end{tikzpicture}
\caption[Sketch of the pair-wise linear valued force function $f$ with the stiffness constant $c$ as slope and the critical bond stretch $s_c$.]{Sketch of the pair-wise linear valued force function $f$ with the stiffness constant \textcolor{blue}{$c$} as slope and the critical bond stretch \textcolor{blue}{$s_c$}.}
\label{fig::force::sketch::critical}
\end{figure}
\end{frame}

\begin{frame}{Definition of damage}
With the scalar valued history dependent function $\mu$ the notion of damage $d(t,X):[0,T]\times\R^3\rightarrow\R$ can be introduced via
\begin{align*}
d(t,X):= 1- \frac{\displaystyle\int\limits_{B_\delta(X)}\mu\args dX'}{\displaystyle\int\limits_{B_\delta(X)}dX'}\,\text{.}
\label{eq::damage:nodal}
\end{align*}
To express damage in words, it is the ratio of the active (non-broken) bonds and the amount of bonds in the reference configuration within the neighborhood.
\end{frame}



\begin{frame}{Relation to classical continuum mechanics}
\begin{block}{Stiffness constant}
\begin{align*}
c = \frac{18K}{\pi\delta}
\end{align*}
\end{block}
\vspace{-0.25cm}
\begin{block}{Critical bond stretch}
\begin{align*}
s_c = \frac{5}{12} \sqrt{\frac{K_{Ic}}{K^2\delta}}
\end{align*}
\end{block}
\vspace{-0.25cm}
With
\begin{itemize}
\item $K$ is the bulk modulus
\item $K_Ic$ is the critical stress intensity factor
\end{itemize}
\end{frame}



\begin{frame}{Notice}
We just covered the basics of peridynamics which are necessary to implement peridyanmics for the course project. Fore more details we refer to 
\begin{itemize}
\item Bobaru, Florin, et al., eds. Handbook of peridynamic modeling. CRC press, 2016.
\item Madenci E, Oterkus E. Peridynamic Theory. InPeridynamic Theory and Its Applications 2014 (pp. 19-43). Springer, New York, NY.
\end{itemize}

\end{frame}

%%%%%%%%%%%%%%%%%%%%%%%%%%%%%%%%%%%%%%%%%%%%%%%%%%%%%%%%%%%%%%%%%%%%%%%%%%%%%%
\section{Implementation}
%%%%%%%%%%%%%%%%%%%%%%%%%%%%%%%%%%%%%%%%%%%%%%%%%%%%%%%%%%%%%%%%%%%%%%%%%%%%%%

\begin{frame}{Algorithm}

\begin{enumerate}
\item Read the input files
\item Compute the neighborhoods $B_\delta$
\item While $t_n \leq T$
\begin{enumerate}
\item Update the boundary conditions 
\item Compute the pair-wise forces $f$
\item Compute the acceleration $a$
\item Approximate the displacement
\item Compute the new positions
\item Output the simulation data
\item Update the time step $t_n = t_n + 1$
\item Update the time $t=\Delta t * t_n$
\end{enumerate} 
\end{enumerate}

\end{frame}


%%%%%%%%%%%%%%%%%%%%%%%%%%%%%%%%%%%%%%%%%%%%%%%%%%%%%%%%%%%%%%%%%%%%%%%%%%%%%%
\section{Summary}
%%%%%%%%%%%%%%%%%%%%%%%%%%%%%%%%%%%%%%%%%%%%%%%%%%%%%%%%%%%%%%%%%%%%%%%%%%%%%%
\begin{frame}{Summary}
\begin{block}{After this lecture, you should know}
\begin{itemize}
\item Concept of peridyanmics
\item Discretization of peridynamics
\item Material models 
\end{itemize}
\end{block}

\begin{center}
\textcolor{red}{Note that this lecture is not relevant for the exams, but you should understand the content to implement the course project.}
\end{center}
\end{frame}



\begin{frame}{Disclaimer}
Some of the material, \emph{e.g.}\ figures, plots, equations, and sentences, were adapted from P. Diehl, Modeling and Simulation of cracks and fractures with peridynamics in brittle materials, Doktorarbeit, University of Bonn, 2017.\\



\end{frame}

\end{document}