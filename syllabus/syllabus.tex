\documentclass[11pt,letterpaper]{article}
\usepackage[utf8]{inputenc}
\usepackage[margin=0.75in]{geometry}
\usepackage{hyperref}

\usepackage{fancyhdr}
\pagestyle{fancy}
\fancyhf{}
\rhead{Fall 2019}
\lhead{Syllabus, 4977}
\rfoot{Page \thepage}

\begin{document}



\section*{Course summary}


\subsection*{Prerequisites}
There   are   no other required   prerequisites,   except   for   willingness   to   learn   andsome   interest   in programming 

\subsection*{Lectures}


\subsection*{Important dates}

\begin{itemize}
\item Midterm exam:
\item Final exams:

\end{itemize}

\subsection*{Reading}

We will use material from the following two books in the course:

\begin{itemize}
\item Andrew, Koenig. Accelerated C++: practical programming by example. Pearson Education India, 2000.
\item Stroustrup, Bjarne. Programming: principles and practice using C++. Pearson Education, 2014.
\end{itemize}

For advanced reading, following books are recommended:

\begin{itemize}
\item Chacon, Scott, and Ben Straub. Pro git. Apress, 2014. \href{https://github.com/progit/progit2/releases/download/2.1.146/progit.pdf}{Ebook}
\item Stroustrup, Bjarne. A Tour of C++. Addison-Wesley Professional, 2018.
\item Scott, Craig. Professional CMake: A Practical Guide, 2018.
\end{itemize}

\subsubsection*{Other resources}

\begin{itemize}
\item Mailing list: \url{https://mail.cct.lsu.edu/mailman/listinfo/par4997}
\item Web page:
\end{itemize}

Please don’t hesitate to ask questions related to the course by sending me emails: \href{mailto:patrickdiehl@lsu.edu}{Patrick Diehl}. 


\subsection*{Projects, Homework, and Quizzes}


\subsection*{Grading}

\begin{itemize}
\item Quizzes and homework 30\%
\item Project 20\%
\item Midterm exam 20\%
\item Final exam 30\%
\end{itemize}
Overall, in the end of the semester 90\% of all points or more will give you an ‘A’, 80\% or more a ‘B’, 70\% or more a ‘C’, and 60\% or more results in a ‘D’. Below that you’ll fail the course, but I’m sure that will not happen to anyone

\subsection*{Topics}

The following list indicates roughly how much time we will spend on each topic:

\begin{table}[h!]
\centering
\begin{tabular}{cl}
\hline
Portion & Topic \\
\hline
1/3 & Introduction to the C++ standard library\\
1/3 & Implementation of numerical algorithms\\
1/3 & Parallel computation using the C++ standard library for parallelism and concurrency \\
\hline
\end{tabular}
\end{table}

There  is  a  full  lecture  calendar  available  on  the  course  webpage  outlining  the  topics  in  more  detail. There  is  also some flexibility in shortening some of these topics and adding other advanced topics.


\section*{Office hours}

After the lectures, I will be around for discussion. During the first lecture, we will arrange a office hour which will be announced on the course web page.


\section*{Course Policy}

\subsection*{Grading}
It is course policy that whoever graded something will be responsible for handling grading disputes. I will grade the midterm exam and the final exam. Thegrader will grade the homeworkand the project. Grades become final one week after homeworkor exam is handed back. This should leave ample time to resolve grading disputes.

\subsection*{Homework Standards}

All homework and the project will be submitted using Gihtub Classroom. There will be instructions on the first exercise sheet. All work submitted must carry the student's name and must have sufficient comments and be well organized. A work that can not be open or read easily will get less credits.  A  reasonable standard  of  English  expression  and  grammar  is  also  required.  The  same  requirements  apply  to  exams.Additional requirements   may   apply   for   any   of   the   separate   assignments   and   will   be   outlined   in   the   corresponding descriptions.
 
\section{Cheating}

Cheating is a very serious offense and will not be tolerated. Supplying others with homework solutions or materials is forbidden. The  policy  is  that  the supplier and the receiver of information will both be dealt with in accordance with and as outlined by the LSU Code of Student Conduct\footnote{\url(http://saa.lsu.edu/Code\%20of\%20Student\%20Conduct\%20August\%2009.pdf)}

\section*{Programming standards}

For this course the C++17 standard is used and obviously your submitted code should compile and run using this standard. Due to the complexity of the program, no credit can be given for a program that does not compile. If a program only partly runs, only partial credit will be given.

\subsection*{Laptops}

My experience has shown that taking notes on electronic devices is quite challenging and might distract others. Therefore, I like to restrict the usage of electronic devices during this lecture. 


\end{document}