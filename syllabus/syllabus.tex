%Syllabus
%Copyright (C) 2019  Patrick Diehl
%
%This program is free software: you can redistribute it and/or modify
%it under the terms of the GNU General Public License as published by
%the Free Software Foundation, either version 3 of the License, or
%(at your option) any later version.

%This program is distributed in the hope that it will be useful,
%but WITHOUT ANY WARRANTY; without even the implied warranty of
%MERCHANTABILITY or FITNESS FOR A PARTICULAR PURPOSE.  See the
%GNU General Public License for more details.

%You should have received a copy of the GNU General Public License
%along with this program.  If not, see <http://www.gnu.org/licenses/>.
\documentclass[11pt,letterpaper]{article}
\usepackage[utf8]{inputenc}
\usepackage[margin=1in]{geometry}
\usepackage[hidelinks]{hyperref}
\usepackage{lastpage}


\newcommand{\courseurl}[0]{https://www.cct.lsu.edu/\string~pdiehl/teaching/2021/4997/}
\newcommand{\coursetimeline}[0]{https://www.cct.lsu.edu/~pdiehl/teaching/2021/4997/timeline.pdf}
\newcommand{\coursesyllabus}[0]{https://www.cct.lsu.edu/~pdiehl/teaching/2021/4997/syllabus.pdf}
\newcommand{\coursename}[0]{Math 4997-3}
\newcommand{\coursemailinglist}[0]{https://mail.cct.lsu.edu/mailman/listinfo/par4997}
\newcommand{\coursesemester}{Fall 2021}








\usepackage{fancyhdr}
\pagestyle{fancy}
\fancyhf{}
\rhead{\coursesemester}
\lhead{Syllabus, \coursename~Parallel computational mathematics}
\rfoot{Page \thepage~of \pageref{LastPage}}
\lfoot{Last updated: \today}

\usepackage{ifxetex}

\ifxetex
\usepackage{fontspec}
\setmainfont{Raleway}
\fi

\usepackage[
    type={CC},
    modifier={by-nc-nd},
    version={4.0},
]{doclicense}

\usepackage{xfrac}

\begin{document}


%%%%%%%%%%%%%%%%%%%%%%%%%%%%%%%%%%%%%%%%%%%%%%%%%%%%%%%%%%%%%%%%%%%%%%%%%%%%%%
\section*{Course summary}
%%%%%%%%%%%%%%%%%%%%%%%%%%%%%%%%%%%%%%%%%%%%%%%%%%%%%%%%%%%%%%%%%%%%%%%%%%%%%%
This course will focus on the parallel implementation of computational mathematics problems using modern accelerated C++. The aim of this course is to learn how to quickly write useful efficient C++ programs. The students will not learn low-level C/C++ instead they will learn how to use high-level data structures, iterators, generic strings, and streams (including interactive and file I/O) of the C++ ISO Standard library. In addition, highly-optimized linear algebra libraries are introduced since the course teaches to solve problems, instead of explaining low-level C++ and computer science algorithms, like sorting algorithms, which are provided in the C++ standard library.\\

\noindent
The first part, provides a brief overview of the containers, strings, streams, input/output, and the numeric library of the C++ standard library. For linear algebra, we will look into Blaze which is an open-source, high-performance C++ math library for dense and sparse arithmetic.

The second part will solve computational mathematics problems based-on the the previous introduced features of the C++ standard library.

The third part will focus on the parallel features provided by the C++ standard library. Here, the implemented computational problems in the second part of the course will be parallized using the C++ standard library for parallelism and concurrency.

Since programming skills can only be improved by doing, there will be weekly programming exercises and a small project. After this course students have a basic overview of the C++ standard library to solve efficiently computational mathematics problems without using low-level C/C++.
%%%%%%%%%%%%%%%%%%%%%%%%%%%%%%%%%%%%%%%%%%%%%%%%%%%%%%%%%%%%%%%%%%%%%%%%%%%%%%
\subsection*{Prerequisites}
%%%%%%%%%%%%%%%%%%%%%%%%%%%%%%%%%%%%%%%%%%%%%%%%%%%%%%%%%%%%%%%%%%%%%%%%%%%%%%
None, but some basic knowledge about C++ is beneficial. 

%%%%%%%%%%%%%%%%%%%%%%%%%%%%%%%%%%%%%%%%%%%%%%%%%%%%%%%%%%%%%%%%%%%%%%%%%%%%%%
\subsection*{Lectures}
%%%%%%%%%%%%%%%%%%%%%%%%%%%%%%%%%%%%%%%%%%%%%%%%%%%%%%%%%%%%%%%%%%%%%%%%%%%%%%
Tuesday and Thursday, 09:00 to 10:20, TBA
%%%%%%%%%%%%%%%%%%%%%%%%%%%%%%%%%%%%%%%%%%%%%%%%%%%%%%%%%%%%%%%%%%%%%%%%%%%%%%
\subsection*{Important dates}
%%%%%%%%%%%%%%%%%%%%%%%%%%%%%%%%%%%%%%%%%%%%%%%%%%%%%%%%%%%%%%%%%%%%%%%%%%%%%%
\begin{itemize}
\item Midterm exam: 15.10 (Take home exam that will be returned at 17.10 before the lecture)
\item Final exams: 10.12 from 12:30 to 2:30
\end{itemize}
%%%%%%%%%%%%%%%%%%%%%%%%%%%%%%%%%%%%%%%%%%%%%%%%%%%%%%%%%%%%%%%%%%%%%%%%%%%%%%
\subsection*{Reading}
%%%%%%%%%%%%%%%%%%%%%%%%%%%%%%%%%%%%%%%%%%%%%%%%%%%%%%%%%%%%%%%%%%%%%%%%%%%%%%

We will use material from the following two books in the course:

\begin{itemize}
\item Andrew, Koenig. Accelerated C++: practical programming by example. Pearson Education India, 2000.
\item Stroustrup, Bjarne. Programming: principles and practice using C++. Pearson Education, 2014.
\end{itemize}

For advanced reading, following books are recommended:

\begin{itemize}
\item Stroustrup, Bjarne. A Tour of C++. Addison-Wesley Professional, 2018.
\item O'Dwyer, Arthur. Mastering the C++17 STL. Packt Publishing Ltd; 2017.
\item Chacon, Scott, and Ben Straub. Pro git. Apress, 2014. \href{https://github.com/progit/progit2/releases/download/2.1.146/progit.pdf}{Ebook}
\item Scott, Craig. Professional CMake: A Practical Guide, 2018.
\end{itemize}

%%%%%%%%%%%%%%%%%%%%%%%%%%%%%%%%%%%%%%%%%%%%%%%%%%%%%%%%%%%%%%%%%%%%%%%%%%%%%%
\subsubsection*{Other resources}
%%%%%%%%%%%%%%%%%%%%%%%%%%%%%%%%%%%%%%%%%%%%%%%%%%%%%%%%%%%%%%%%%%%%%%%%%%%%%%

\begin{itemize}
\item Mailing list: \url{\coursemailinglist}
\item Web page: \url{\courseurl}
\end{itemize}

Please don’t hesitate to ask questions related to the course by sending me emails: \href{mailto:patrickdiehl@lsu.edu}{Patrick Diehl}. 

%%%%%%%%%%%%%%%%%%%%%%%%%%%%%%%%%%%%%%%%%%%%%%%%%%%%%%%%%%%%%%%%%%%%%%%%%%%%%%
\subsection*{Projects, Homework, and Quizzes}
%%%%%%%%%%%%%%%%%%%%%%%%%%%%%%%%%%%%%%%%%%%%%%%%%%%%%%%%%%%%%%%%%%%%%%%%%%%%%%

%%%%%%%%%%%%%%%%%%%%%%%%%%%%%%%%%%%%%%%%%%%%%%%%%%%%%%%%%%%%%%%%%%%%%%%%%%%%%%
\subsection*{Grading}
%%%%%%%%%%%%%%%%%%%%%%%%%%%%%%%%%%%%%%%%%%%%%%%%%%%%%%%%%%%%%%%%%%%%%%%%%%%%%%
\begin{itemize}
\item Quizzes and homework 30\%
\item Project 20\%
\item Midterm exam 20\%
\item Final exam 30\%
\end{itemize}
Overall, in the end of the semester 90\% of all points or more will give you an ‘A’, 80\% or more a ‘B’, 70\% or more a ‘C’, and 60\% or more results in a ‘D’. Below that you’ll fail the course, but I’m sure that will not happen to anyone.

%%%%%%%%%%%%%%%%%%%%%%%%%%%%%%%%%%%%%%%%%%%%%%%%%%%%%%%%%%%%%%%%%%%%%%%%%%%%%%
\subsection*{C-I Certification}
%%%%%%%%%%%%%%%%%%%%%%%%%%%%%%%%%%%%%%%%%%%%%%%%%%%%%%%%%%%%%%%%%%%%%%%%%%%%%%
This is a certified Communication-Intensive (C-I) course which meets all of the requirements set forth by LSU’s Communication across the Curriculum program, including
\begin{itemize}
\item instruction and assignments emphasizing informal and formal \textit{Written} and \textit{Technological};
\item teaching of discipline-specific communication techniques;
\item use of feedback loops for learning;
\item 40\% of the course grade rooted in communication-based work; and
\item practice of ethical and professional work standards.
\end{itemize}
Students interested in pursuing the LSU Communicator Certificate and/or the LSU Distinguished Communicator Medal may use this C-I course for credit. For more information about these student recognition programs, visit \url{www.cxc.lsu.edu}.

%%%%%%%%%%%%%%%%%%%%%%%%%%%%%%%%%%%%%%%%%%%%%%%%%%%%%%%%%%%%%%%%%%%%%%%%%%%%%%
\subsection*{Topics}
%%%%%%%%%%%%%%%%%%%%%%%%%%%%%%%%%%%%%%%%%%%%%%%%%%%%%%%%%%%%%%%%%%%%%%%%%%%%%%

Table~\ref{tab:outline} indicates roughly how much time we will spend on each topic. There  is  a  full  lecture  calendar  available  on  the  course  web page  outlining  the  topics  in  more  detail. There  is  also some flexibility in shortening some of these topics and adding other advanced topics.

\begin{table}[b]
\centering
\begin{tabular}{cl}
\hline
Portion & Topic \\
\hline
\sfrac{1}{3} & Accelerated/Modern C++\\
\sfrac{1}{3} & Modern C++ implementation of computational mathematics problems \\
\sfrac{1}{3} & Parallel computation using the C++ standard library for parallelism and concurrency \\
\hline
\end{tabular}
\caption{Brief outline of the topics of the course}
\label{tab:outline}
\end{table}

%%%%%%%%%%%%%%%%%%%%%%%%%%%%%%%%%%%%%%%%%%%%%%%%%%%%%%%%%%%%%%%%%%%%%%%%%%%%%%
\section*{Office hours}
%%%%%%%%%%%%%%%%%%%%%%%%%%%%%%%%%%%%%%%%%%%%%%%%%%%%%%%%%%%%%%%%%%%%%%%%%%%%%%
After the lectures, I will be around for discussion. During the first lecture, we will arrange a office hour which will be announced on the course web page.

%%%%%%%%%%%%%%%%%%%%%%%%%%%%%%%%%%%%%%%%%%%%%%%%%%%%%%%%%%%%%%%%%%%%%%%%%%%%%%
\section*{Course Policy}

%%%%%%%%%%%%%%%%%%%%%%%%%%%%%%%%%%%%%%%%%%%%%%%%%%%%%%%%%%%%%%%%%%%%%%%%%%%%%%
%%%%%%%%%%%%%%%%%%%%%%%%%%%%%%%%%%%%%%%%%%%%%%%%%%%%%%%%%%%%%%%%%%%%%%%%%%%%%%
\subsection*{Grading}
%%%%%%%%%%%%%%%%%%%%%%%%%%%%%%%%%%%%%%%%%%%%%%%%%%%%%%%%%%%%%%%%%%%%%%%%%%%%%%
It is course policy that whoever graded something will be responsible for handling grading disputes. I will grade the midterm exam and the final exam. The grader will grade the homework and the project. Grades become final one week after homework or exam is handed back. This should leave ample time to resolve grading disputes.

%%%%%%%%%%%%%%%%%%%%%%%%%%%%%%%%%%%%%%%%%%%%%%%%%%%%%%%%%%%%%%%%%%%%%%%%%%%%%%
\subsection*{Homework Standards}
%%%%%%%%%%%%%%%%%%%%%%%%%%%%%%%%%%%%%%%%%%%%%%%%%%%%%%%%%%%%%%%%%%%%%%%%%%%%%%
All homework and the project will be submitted using Gihtub Classroom. There will be instructions on the first exercise sheet. All work submitted must carry the student's name and must have sufficient comments and be well organized. A work that can not be open or read easily will get less credits.  A  reasonable standard  of  English  expression  and  grammar  is  also  required.  The  same  requirements  apply  to  exams. Additional requirements   may   apply   for   any   of   the   separate   assignments   and   will   be   outlined   in   the   corresponding descriptions.
 
%%%%%%%%%%%%%%%%%%%%%%%%%%%%%%%%%%%%%%%%%%%%%%%%%%%%%%%%%%%%%%%%%%%%%%%%%%%%%%
\section*{Cheating}
%%%%%%%%%%%%%%%%%%%%%%%%%%%%%%%%%%%%%%%%%%%%%%%%%%%%%%%%%%%%%%%%%%%%%%%%%%%%%%
Cheating is a very serious offense and will not be tolerated. Supplying others with homework solutions or materials is forbidden. The  policy  is  that  the supplier and the receiver of information will both be dealt with in accordance with and as outlined by the LSU Code of Student Conduct\footnote{\url{https://www.lsu.edu/saa/students/codeofconduct.php}}.

%%%%%%%%%%%%%%%%%%%%%%%%%%%%%%%%%%%%%%%%%%%%%%%%%%%%%%%%%%%%%%%%%%%%%%%%%%%%%%
\section*{Programming standards}
%%%%%%%%%%%%%%%%%%%%%%%%%%%%%%%%%%%%%%%%%%%%%%%%%%%%%%%%%%%%%%%%%%%%%%%%%%%%%%
For this course the C++17 standard is used and obviously your submitted code should compile and run using this standard. Due to the complexity of the program, no credit can be given for a program that does not compile. If a program only partly runs, only partial credit will be given.

%%%%%%%%%%%%%%%%%%%%%%%%%%%%%%%%%%%%%%%%%%%%%%%%%%%%%%%%%%%%%%%%%%%%%%%%%%%%%%
\subsection*{Laptops}
%%%%%%%%%%%%%%%%%%%%%%%%%%%%%%%%%%%%%%%%%%%%%%%%%%%%%%%%%%%%%%%%%%%%%%%%%%%%%%
My experience has shown that taking notes on electronic devices is quite challenging and might distract others. Therefore, I like to restrict the usage of electronic devices during the lectures, if I feel that the lecture will be disturbed. During live coding session, I encourage you to use your device and play with the code.


\doclicenseThis

\end{document}
