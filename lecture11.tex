%Template
%Copyright (C) 2019  Patrick Diehl
%
%This program is free software: you can redistribute it and/or modify
%it under the terms of the GNU General Public License as published by
%the Free Software Foundation, either version 3 of the License, or
%(at your option) any later version.

%This program is distributed in the hope that it will be useful,
%but WITHOUT ANY WARRANTY; without even the implied warranty of
%MERCHANTABILITY or FITNESS FOR A PARTICULAR PURPOSE.  See the
%GNU General Public License for more details.

%You should have received a copy of the GNU General Public License
%along with this program.  If not, see <http://www.gnu.org/licenses/>.
\documentclass[12pt,t]{beamer}

\beamertemplatenavigationsymbolsempty

% usepackage
%\usepackage{template/dbt}
\usepackage{listings}

\definecolor{comments}{RGB}{81,81,81}
\definecolor{keywords}{RGB}{255,0,90}

% lstlisting
\lstset{
    language=C,
    basicstyle=\footnotesize\ttfamily,
    keywordstyle=\color{keywords},
    showspaces=false,
    showstringspaces=false,
    commentstyle=\color{blue}\emph
    %frame=single,
    %rulecolor=\color{comments},
    %rulesepcolor=\color{comments},
    %backgroundcolor = \color{lightgray}
}

\usetheme{default}

\usepackage[
    type={CC},
    modifier={by-nc-nd},
    version={4.0},
]{doclicense} 

\newcommand{\courseurl}[0]{https://www.cct.lsu.edu/\string~pdiehl/teaching/2021/4997/}
\newcommand{\coursetimeline}[0]{https://www.cct.lsu.edu/~pdiehl/teaching/2021/4997/timeline.pdf}
\newcommand{\coursesyllabus}[0]{https://www.cct.lsu.edu/~pdiehl/teaching/2021/4997/syllabus.pdf}
\newcommand{\coursename}[0]{Math 4997-3}
\newcommand{\coursemailinglist}[0]{https://mail.cct.lsu.edu/mailman/listinfo/par4997}
\newcommand{\coursesemester}{Fall 2021}








% frame slide
\title{\coursename}
\subtitle{Lecture 11: Introduction to HPX}

%\author{\href{}{}}
%\institute {
%    \href{}{\tt \scriptsize \today}
%}
\date {
 \tiny \url{\courseurl}
\vspace{2cm}
\doclicenseThis  
  
}



\usepackage{ifxetex}

\ifxetex
\usepackage{fontspec}
\setmainfont{Raleway}
\fi

\ifluatex
\usepackage{fontspec}
\setmainfont{Raleway}
\fi


\begin{document} {
    \setbeamertemplate{footline}{}
    \frame {
        \titlepage
    }
}

\frame{

\tableofcontents

}

\AtBeginSection[]{
  \begin{frame}
  \vfill
  \centering
  \begin{beamercolorbox}[sep=8pt,center,shadow=true,rounded=true]{title}
    \usebeamerfont{title}\insertsectionhead\par%
  \end{beamercolorbox}
  \vfill
  \end{frame}
}

%%%%%%%%%%%%%%%%%%%%%%%%%%%%%%%%%%%%%%%%%%%%%%%%%%%%%%%%%%%%%%%%%%%%%%%%%%%%%%
\section{Reminder}
%%%%%%%%%%%%%%%%%%%%%%%%%%%%%%%%%%%%%%%%%%%%%%%%%%%%%%%%%%%%%%%%%%%%%%%%%%%%%%
\begin{frame}{Lecture X}
\begin{block}{What you should know from last lecture}
\begin{itemize}
\item 
\end{itemize}
\end{block}
\end{frame}


%%%%%%%%%%%%%%%%%%%%%%%%%%%%%%%%%%%%%%%%%%%%%%%%%%%%%%%%%%%%%%%%%%%%%%%%%%%%%%
\section{What is HPX}
%%%%%%%%%%%%%%%%%%%%%%%%%%%%%%%%%%%%%%%%%%%%%%%%%%%%%%%%%%%%%%%%%%%%%%%%%%%%%%


%%%%%%%%%%%%%%%%%%%%%%%%%%%%%%%%%%%%%%%%%%%%%%%%%%%%%%%%%%%%%%%%%%%%%%%%%%%%%%
\section{Compilation and running}
%%%%%%%%%%%%%%%%%%%%%%%%%%%%%%%%%%%%%%%%%%%%%%%%%%%%%%%%%%%%%%%%%%%%%%%%%%%%%%

\begin{frame}[fragile]{Compilation and running}

\begin{block}{CMake}
\begin{lstlisting}[language=bash]
cmake_minimum_required(VERSION 3.3.2)
project(my_hpx_project CXX)
find_package(HPX REQUIRED)
add_hpx_executable(my_hpx_program
    SOURCES main.cpp
)
\end{lstlisting}
\end{block}

\begin{block}{Running}
\begin{lstlisting}[language=bash]
cmake .
make
./my_hpx_program --hpx:threads=4
\end{lstlisting}
\end{block}
\end{frame}


%%%%%%%%%%%%%%%%%%%%%%%%%%%%%%%%%%%%%%%%%%%%%%%%%%%%%%%%%%%%%%%%%%%%%%%%%%%%%%
\section{Hello World}
%%%%%%%%%%%%%%%%%%%%%%%%%%%%%%%%%%%%%%%%%%%%%%%%%%%%%%%%%%%%%%%%%%%%%%%%%%%%%%

\begin{frame}[fragile]{A small HPX program}

\begin{block}{C++}
\begin{lstlisting}
int main()
{
    hpx::cout << "Hello World!\n" << hpx::flush;
    return 0;
}
\end{lstlisting}
\end{block}

\begin{block}{HPX}
\begin{lstlisting}
#include <hpx/hpx_main.hpp>
#include <iostream>

int main()
{
    std::cout << "Hello World!\n" << std::endl;
    return 0;
}
\end{lstlisting}
\end{block}

\end{frame}


\begin{frame}[fragile]{Hello world using \lstinline{hpx::init}}

\begin{lstlisting}
#include <hpx/hpx_init.hpp>
#include <iostream>

int hpx_main(int, char**)
{
    // Say hello to the world!
    std::cout << "Hello World!\n" << std::endl;
    return hpx::finalize();
}

int main(int argc, char* argv[])
{
    return hpx::init(argc, argv);
}
\end{lstlisting}
\begin{center}
Note that here we  initialize the HPX runtime explicitly.
\end{center}
\end{frame}

%%%%%%%%%%%%%%%%%%%%%%%%%%%%%%%%%%%%%%%%%%%%%%%%%%%%%%%%%%%%%%%%%%%%%%%%%%%%%%
\section{Asynchronous programming}
%%%%%%%%%%%%%%%%%%%%%%%%%%%%%%%%%%%%%%%%%%%%%%%%%%%%%%%%%%%%%%%%%%%%%%%%%%%%%%


\begin{frame}[fragile]{Futurization\footnote{\tiny\href{https://stellar-group.github.io/hpx/docs/sphinx/latest/html/examples/fibonacci_local.html?highlight=async}{Example: hpx::async}}}

\begin{lstlisting}
#include <hpx/hpx_init.hpp>
#include <hpx/incldue/lcos.hpp>

int square(int a)
{ 
    return a*a; 
}

int main()
{
    hpx::future<int> f1 = hpx::async(square,10); 
    
    hpx::cout << f1.get() << hpx::flush;
    
    return EXIT_SUCCESS;
}

\end{lstlisting}
\vspace{-0.25cm}
\begin{center}
Note that we just replaced \lstinline|std| by the namespace \lstinline|hpx|
\end{center}
\end{frame}


\begin{frame}[fragile]{Advanced synchronization\footnote{\tiny\href{https://stellar-group.github.io/hpx/docs/sphinx/latest/html/api.html?highlight=when_all\#_CPPv4IDpEN3hpx8when_allE6futureI5tupleIDp6futureI1TEEEDpRR1T}{Documentation: hpx::when\_all}}}

\begin{lstlisting}

void result(std::vector<hpx::future<int> f)
{
    hpx::cout << f[0].get() 
        + f[1].get() << hpx::flush;
}

std::vector<hpx::future<int> futures;

futures.push_back(hpx::async(square,10);
futures.push_back(hpx::async(square,100);

hpx::when_all(futures).then([](auto&& f){
 auto = f.get();
 std::cout << futures[0].get() 
 	<< " and " << futures[1].get();
});


\end{lstlisting}

\end{frame}

\begin{frame}[fragile]{Synchronization\footnote{\tiny\href{https://stellar-group.github.io/hpx/docs/sphinx/latest/html/terminology.html\#term-lco}{Documentation: LCO}}}

\begin{itemize}
\item \lstinline|when_all| \\
It \textit{AND}-composes all the given futures and returns a new future containing all the given futures.
\item \lstinline|when_any| \\
It \textit{OR}-composes all the given futures and returns a new future a new future containing all the given futures.
\item \lstinline|when_each| \\
It \textit{AND}-composes all the given futures and returns a new future containing all futures being ready.
\item \lstinline|when_some| \\
It \textit{AND}-composes all the given futures and returns a new future object representing the same list of futures after n of them finished.

\end{itemize}

\end{frame}


%%%%%%%%%%%%%%%%%%%%%%%%%%%%%%%%%%%%%%%%%%%%%%%%%%%%%%%%%%%%%%%%%%%%%%%%%%%%%%
\section{Parallel algorithms}
%%%%%%%%%%%%%%%%%%%%%%%%%%%%%%%%%%%%%%%%%%%%%%%%%%%%%%%%%%%%%%%%%%%%%%%%%%%%%%


\begin{frame}[fragile]{Example: Reduce}

\begin{block}{C++}
\begin{lstlisting}
#include<algorithm>
#include<execution>

std::reduce(std::execution::par,
	values.begin(),values.end(),0);
\end{lstlisting}
\end{block}

\begin{block}{HPX}
\begin{lstlisting}
#include<hpx/include/parallel_reduce.hpp>
#include<vector>

hpx::parallel::v1::reduce(
	hpx::parallel::execution::par,
		values.begin(),values.end(),0);

\end{lstlisting}
\end{block}

\end{frame}


\begin{frame}[fragile]{Example: Reduce with future}


\begin{lstlisting}
auto f =

hpx::parallel::v1::reduce(
	hpx::parallel::execution::par(
		hpx::parallel::execution::task),
	values.begin(),
	values.end(),0);

std::cout<< f.get();
\end{lstlisting}

\begin{itemize}
\item \lstinline|hpx::parallel::execution::par| Parallel execution
\item \lstinline|hpx::parallel::execution::par| Sequential execution
\item \lstinline|hpx::parallel::execution::task| Task-based execution
\end{itemize}

\end{frame}


\begin{frame}[fragile]{Execution parameters}


\begin{lstlisting}
#include<hpx/include/parallel_executor_parameters.hpp>

hpx::parallel::execution::static_chunk_size scs(10);
hpx::parallel::v1::reduce(
	hpx::parallel::execution::par.with(scs),
	values.begin(),
	values.end(),0);
\end{lstlisting}

\begin{itemize}
\item \lstinline|hpx::parallel::execution::static_chunk_size| \\
Loop iterations are divided into pieces of a given size and then assigned to threads.
\item \lstinline|hpx::parallel::execution::auto_chunk_size| \\
Pieces are determined based on the first 1\% of the total loop iterations. 
\item \lstinline|hpx::parallel::execution::dynamic_chunk_size| \\
Dynamically scheduled among the cores and if one core finished it gets dynamically assigned a new chunk.


\end{itemize}

\end{frame}


\begin{frame}[fragile]{Example: Range-based for loops}


\begin{lstlisting}
#include<vector>
#include<iostream>
#include<hpx/include/parallel_for_loop.hpp>

std::vector<double> values = {1,2,3,4,5,6,7,8,9};

hpx::parallel::for_loop(
	hpx::parallel::execution::par, 
	0, 
	values.size();
	[](boost::uint64_t i)
		{
		std::cout<< values[i] << std::endl;
		}
	);
\end{lstlisting}

\end{frame}




%%%%%%%%%%%%%%%%%%%%%%%%%%%%%%%%%%%%%%%%%%%%%%%%%%%%%%%%%%%%%%%%%%%%%%%%%%%%%%
\section{Summary}
%%%%%%%%%%%%%%%%%%%%%%%%%%%%%%%%%%%%%%%%%%%%%%%%%%%%%%%%%%%%%%%%%%%%%%%%%%%%%%
\begin{frame}{Summary}
\begin{block}{After this lecture, you should know}
\begin{itemize}
\item What is HPX
\item Asynchronous programming using HPX
\item Shared memory parallelism using HPX
\end{itemize}
\end{block}
\end{frame}


\end{document}