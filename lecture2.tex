%Lecture 2
%Copyright (C) 2019  Patrick Diehl
%
%This program is free software: you can redistribute it and/or modify
%it under the terms of the GNU General Public License as published by
%the Free Software Foundation, either version 3 of the License, or
%(at your option) any later version.

%This program is distributed in the hope that it will be useful,
%but WITHOUT ANY WARRANTY; without even the implied warranty of
%MERCHANTABILITY or FITNESS FOR A PARTICULAR PURPOSE.  See the
%GNU General Public License for more details.

%You should have received a copy of the GNU General Public License
%along with this program.  If not, see <http://www.gnu.org/licenses/>.
\documentclass[12pt]{beamer}

\beamertemplatenavigationsymbolsempty

% usepackage
%\usepackage{template/dbt}
\usepackage{listings}

\usepackage{xfrac}

\usepackage{tikz}

\definecolor{comments}{RGB}{81,81,81}
\definecolor{keywords}{RGB}{255,0,90}

% lstlisting
\lstset{
    language=C,
    basicstyle=\footnotesize\ttfamily,
    keywordstyle=\color{keywords},
    showspaces=false,
    showstringspaces=false,
    commentstyle=\color{blue}\emph
    %frame=single,
    %rulecolor=\color{comments},
    %rulesepcolor=\color{comments},
    %backgroundcolor = \color{lightgray}
}

\usetheme{default}

\usepackage[
    type={CC},
    modifier={by-nc-nd},
    version={4.0},
]{doclicense} 

\newcommand{\courseurl}[0]{https://www.cct.lsu.edu/\string~pdiehl/teaching/2021/4997/}
\newcommand{\coursetimeline}[0]{https://www.cct.lsu.edu/~pdiehl/teaching/2021/4997/timeline.pdf}
\newcommand{\coursesyllabus}[0]{https://www.cct.lsu.edu/~pdiehl/teaching/2021/4997/syllabus.pdf}
\newcommand{\coursename}[0]{Math 4997-3}
\newcommand{\coursemailinglist}[0]{https://mail.cct.lsu.edu/mailman/listinfo/par4997}
\newcommand{\coursesemester}{Fall 2021}








% frame slide
\title{\coursename}
\subtitle{Lecture 2: Monte Carlo, Vectors, and Algorithms}

%\author{\href{}{}}
%\institute {
%    \href{}{\tt \scriptsize \today}
%}
\date {
 \tiny \url{\courseurl}
\vspace{2cm}
\doclicenseThis  
  
}

\usepackage{ifxetex}

\ifxetex
\usepackage{fontspec}
\setmainfont{Raleway}
\fi

\ifluatex
\usepackage{fontspec}
\setmainfont{Raleway}
\fi


\begin{document} {
    \setbeamertemplate{footline}{}
    \frame {
        \titlepage
    }
}

\frame{

\tableofcontents

}

\AtBeginSection[]{
  \begin{frame}
  \vfill
  \centering
  \begin{beamercolorbox}[sep=8pt,center,shadow=true,rounded=true]{title}
    \usebeamerfont{title}\insertsectionhead\par%
  \end{beamercolorbox}
  \vfill
  \end{frame}
}


%%%%%%%%%%%%%%%%%%%%%%%%%%%%%%%%%%%%%%%%%%%%%%%%%%%%%%%%%%%%%%%%%%%%%%%%%%%%%%
\section{Reminder}
%%%%%%%%%%%%%%%%%%%%%%%%%%%%%%%%%%%%%%%%%%%%%%%%%%%%%%%%%%%%%%%%%%%%%%%%%%%%%%

\begin{frame}{Lecture 1}
\begin{block}{What you should know from last lecture}
\begin{itemize}
\item Structure of a C++ program
\item \lstinline|std::string|
\begin{itemize}
\item Reading strings \lstinline|std::cin|
\item Writing strings \lstinline|std:cout|
\end{itemize}
\item Looping and counting
\begin{itemize}
\item \lstinline|while|
\item \lstinline|for|
\end{itemize}
\item Conditionals \lstinline|if|
\item Operators
\item Built-in types
\end{itemize}
\end{block}
\end{frame}

%%%%%%%%%%%%%%%%%%%%%%%%%%%%%%%%%%%%%%%%%%%%%%%%%%%%%%%%%%%%%%%%%%%%%%%%%%%%%%
\section{Monte Carlo methods}
%%%%%%%%%%%%%%%%%%%%%%%%%%%%%%%%%%%%%%%%%%%%%%%%%%%%%%%%%%%%%%%%%%%%%%%%%%%%%%

\begin{frame}{Monte Carlo method}

\textbf{Monte Carlo methods} are computational algorithms which rely on repeated random sampling to obtain numerical results. More details~\cite{shonkwiler2009explorations}.

\begin{block}{Three problem classes~\cite{kroese2014monte}:}
\begin{enumerate}
\item Optimization
\item Numerical integration
\item Probability distributions
\end{enumerate}
\end{block}

\begin{block}{General pattern:}
\begin{enumerate}
\item Define the input parameters
\item Randomly chose input parameters   
\item Do deterministic computations on the inputs
\item Aggregate the results
\end{enumerate}
\end{block}
\end{frame}

\begin{frame}{Example: Estimate the value of $\pi$}

\begin{columns}

\begin{column}{0.3\textwidth}
\centering
\only<1>{
\begin{tikzpicture}
\draw (0,0) -- (2,0) -- (2,2) -- (0,2) -- cycle;
\end{tikzpicture}
}
\only<2>{
\begin{tikzpicture}
\draw (0,0) -- (2,0) -- (2,2) -- (0,2) -- cycle;
\draw (1,1) circle (1cm);
\end{tikzpicture}
}
\end{column}
\begin{column}{0.7\textwidth}
\begin{block}{Ingredients}
\begin{itemize}
\item Unit square $1\times 1$ 
\item circle with the radius of $r=\sfrac{1}{2}$
\item Area of the circle $A_c=\pi r^2 = \sfrac{\pi}{4} $ 
\item Area of the square $A_s = 1 \cdot 1 = 1$
\item Recall: $A_c=\sfrac{\pi}{4}\rightarrow \pi = 4A_c$
\item Hint:   $\sfrac{A_c}{A_s} = A_c$ because $A_s=1$
\end{itemize}
\end{block}
\end{column}
\end{columns}
%\vspace{1cm}
Now compute PI by using the two areas:
\begin{align*}
\pi \approx 4 \sfrac{A_c}{A_s}
\end{align*}
The areas can be approximated by using $N$ random samples $(x,y)$ and count the points inside the circle $N_c$ 
\begin{align*}
\pi \approx 4 \sfrac{N_c}{N}
\end{align*}
\end{frame}


\begin{frame}{Algorithm}
\begin{enumerate}
\item Generate random coordinates $(x,y)\in \mathbb{R}^2$
\item Check if $x^2+y^2 \leq 1$
\begin{itemize}
\item Update $N_c$ if $\leq 1$
\end{itemize}
\item Increment $n$ the interval  
\item If $n$ < $n_{\text{total}}$ go to 1
\item Calculate $\pi \approx 4 \sfrac{N_c}{n_\text{total}}$
\item Print result
\end{enumerate}
\vspace{0.25cm}
\textcolor{blue}{Interactive question:} \\
\vspace{0.125cm}
Which features of C++ do we need? \\
\vspace{0.25cm}
\only<2>{

\lstinline|for|, \lstinline|if|, \lstinline|std::cout|, and random numbers
}
\end{frame}

\begin{frame}{Random numbers\footnote{\tiny\url{https://en.cppreference.com/w/cpp/numeric/random/srand}}}

\lstinputlisting{code/02/lecture2-random.cpp}
More details: Section~3.2~\cite{knuth1997art2}
\end{frame}

\begin{frame}{Uniform distributed random numbers\footnote{\tiny\url{https://en.cppreference.com/w/cpp/numeric/random/uniform_int_distribution}} \footnote{\tiny\url{https://en.cppreference.com/w/cpp/numeric/random/uniform_real_distribution}}}

\lstinputlisting{code/02/lecture2-distrandom.cpp}

\end{frame}

%%%%%%%%%%%%%%%%%%%%%%%%%%%%%%%%%%%%%%%%%%%%%%%%%%%%%%%%%%%%%%%%%%%%%%%%%%%%%%
\section{Containers \& Algorithms}
%%%%%%%%%%%%%%%%%%%%%%%%%%%%%%%%%%%%%%%%%%%%%%%%%%%%%%%%%%%%%%%%%%%%%%%%%%%%%%

\begin{frame}{Computation of the average\footnote{\tiny\url{https://en.cppreference.com/w/cpp/io/manip/setprecision/}}}


\lstinputlisting{code/02/lecture2-average.cpp}

\end{frame}

\begin{frame}{Reading multiple values}

\begin{block}{Recall: Reading one value}
\lstinline| std::cin >> x|
\end{block}

\begin{block}{Reading multiple values}
\lstinputlisting[firstline=7, lastline=13]{code/02/lecture2-average.cpp}
\end{block}

Note that the \lstinline|while| loops stops by inserting \lstinline|\n|

\end{frame}

\begin{frame}{What about computing the median?}

\begin{block}{We have to:}
\begin{itemize}
\item We have to store all values without knowing the amount in advance
\item We have to sort all the values 
\item We have to select the middle value(s) efficiently
\end{itemize}

\end{block}
\pause
\begin{block}{What we will use:}
\begin{itemize}
\item Containers: \lstinline| std::vector | to store the elements
\item Algorithm, like sorting or accumulate
\item Iterators for accessing elements
\end{itemize}

\end{block}

\end{frame}

\begin{frame}{Advanced computing of the average}

\lstinputlisting{code/02/lecture2-averageContainers.cpp}

\end{frame}

\begin{frame}{Initialization of Vectors\footnote{\tiny{\url{https://en.cppreference.com/w/cpp/container/vector}}}}

\begin{block}{Initialization of an empty vector}
\begin{itemize}
\item \lstinline| std::vector<double> values;| 
\item \lstinline| values.empty() | will return \lstinline|true|
\item \lstinline| values.size() | will return zero
\end{itemize}
\end{block}

\begin{block}{Initialization}
\begin{itemize}
\item \lstinline| std::vector<double> v = \{1, 2.5\}; |
\item \lstinline| values.empty() | will return \lstinline|false|
\item \lstinline| values.size() | will return two
\end{itemize}
\end{block}

\textcolor{red}{Note that C++ starts counting by 0 and the length is always number of elements minus one!}

\end{frame}

\begin{frame}{Manipulating vectors}

\begin{block}{Inserting}
\begin{itemize}
\item \lstinline| values.push_back(3.5); | append 3.5 at the end
\item \lstinline| values[3] = 1.5; | replaces the third element by 1.5
\end{itemize}
\end{block}

\begin{block}{Accessing}
\begin{itemize}
\item \lstinline| values[i] | returns element $i$
\item \lstinline| values.first() | returns the first element
\item \lstinline| values.last() | return the last element
\end{itemize}
\end{block}

\begin{block}{Deleting}
\begin{itemize}
\item \lstinline|values.pop_back(); | deletes the last element
\item \lstinline|values.erase(values.start()+i)| deletes the $i$-th element 
\end{itemize}
\end{block}

\end{frame}


\begin{frame}{Algorithms}

\begin{block}{Accumulate\footnote{\tiny\url{https://en.cppreference.com/w/cpp/algorithm/accumulate}}}
\lstinline| double sum = std::accumulate(v.begin(), v.end(), 0.0f); |
\begin{itemize}
\item The first two arguments define the range of the vector
\item The third is the initial value of the sum 
\item The header \lstinline|\#include <numerics>| provides this functionality
\end{itemize}
\end{block}

\begin{block}{Sorting\footnote{\tiny\url{https://en.cppreference.com/w/cpp/algorithm/sort}}}
\lstinline|std::sort(s.begin(), s.end());|
\begin{itemize}
\item The header \lstinline|\#include <algorithm>| provides this functionality
\end{itemize}
\end{block}

\end{frame}

\begin{frame}[fragile]{Looping over a vector}

\begin{block}{Long form}
\begin{lstlisting}
for( size_t i = 0; i < value.size() ; i++)
{
	v[i] *= 2
}
\end{lstlisting}
\end{block}


\begin{block}{Short form}
\begin{lstlisting}
for (auto& v : values)
{
	v *= 2
}
\end{lstlisting}
\end{block}
\begin{itemize}
\item \lstinline|auto| is equivalent to \lstinline|double v : values|\footnote{\tiny\url{https://en.cppreference.com/w/cpp/language/auto}} 
\end{itemize}

\end{frame}

\begin{frame}{Recall: Advanced computing of the average}

\lstinputlisting{code/02/lecture2-averageContainers.cpp}

\end{frame}

\begin{frame}[fragile]{Computing the median}

\begin{block}{Code}
\begin{lstlisting}

typedef std::vector<double>::size_type vec_sz;
vec_sz size = values.size();
double median = 0;

if( size % 2 == 0 ) 
	median = 0.5*(values[mid] + values[mid-1]);
else
	median = values[mid];
\end{lstlisting}
\end{block}

\begin{block}{Features}
\begin{itemize}
\item \lstinline|typedef|\footnote{\tiny\url{https://en.cppreference.com/w/cpp/language/typedef}} is useful to not write large expression again
\item Conditional parameter\footnote{\tiny\url{https://en.cppreference.com/w/cpp/language/operator_other}} \lstinline|median = size % 2 == 0 ? true : false; |
\end{itemize}
\end{block}
\end{frame}

\begin{frame}{Example: Advanced computing of the average}

\lstinputlisting{code/02/lecture2-median.cpp}

\end{frame}

%%%%%%%%%%%%%%%%%%%%%%%%%%%%%%%%%%%%%%%%%%%%%%%%%%%%%%%%%%%%%%%%%%%%%%%%%%%%%%
\section{Functions}
%%%%%%%%%%%%%%%%%%%%%%%%%%%%%%%%%%%%%%%%%%%%%%%%%%%%%%%%%%%%%%%%%%%%%%%%%%%%%%

\begin{frame}[fragile]{Function definition\footnote{\tiny\url{https://en.cppreference.com/w/c/language/function_definition}}}

\begin{block}{Example: maximum}
\begin{lstlisting}
int max(int a, int b)
{
return return a>b?a:b;
}

\end{lstlisting}
\end{block}
\begin{block}{Each function has}
\begin{itemize}
\item a name \lstinline|max|
\item a return type \lstinline|int max()|
\item a return value \lstinline|return a>b?a:b;|
\item some function parameters \lstinline| max(int a, int b)| or none \lstinline| max(void) |
\end{itemize}
\end{block}
Function are defined between \lstinline|#include| and \lstinline|int main (void)|
\end{frame}

\begin{frame}[fragile]{Function definition for the median}

\begin{block}{Function definition}
\begin{lstlisting}
double median(std::vector<double> values)
{
std::sort(values.begin(),values.end());
vec_sz mid = values.size() / 2;
double median = values.size() % 2 == 0 ? 
    0.5*(values[mid]+values[mid-1]) : values[mid];
    
return median;
}
\end{lstlisting}
\end{block}

\begin{block}{Example}

\lstinline|double median = median(values)|

\end{block}

\end{frame}

%%%%%%%%%%%%%%%%%%%%%%%%%%%%%%%%%%%%%%%%%%%%%%%%%%%%%%%%%%%%%%%%%%%%%%%%%%%%%%
\section{Summary}
%%%%%%%%%%%%%%%%%%%%%%%%%%%%%%%%%%%%%%%%%%%%%%%%%%%%%%%%%%%%%%%%%%%%%%%%%%%%%%

\begin{frame}{Summary}
\begin{block}{After this lecture, you should know}
\begin{itemize}
\item Monte Carlo Methods
\item Random numbers
\item Containers like \lstinline|std::vector|
\item Functions
\end{itemize}
\end{block}


\end{frame}

%%%%%%%%%%%%%%%%%%%%%%%%%%%%%%%%%%%%%%%%%%%%%%%%%%%%%%%%%%%%%%%%%%%%%%%%%%%%%%
\section{References}
%%%%%%%%%%%%%%%%%%%%%%%%%%%%%%%%%%%%%%%%%%%%%%%%%%%%%%%%%%%%%%%%%%%%%%%%%%%%%%

\begin{frame}[t, allowframebreaks]
\frametitle{References}
\bibliographystyle{plain}
\bibliography{bib}
\end{frame}

\end{document}