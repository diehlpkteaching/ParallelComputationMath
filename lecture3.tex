\documentclass[12pt,t]{beamer}

\beamertemplatenavigationsymbolsempty
% usepackage
%\usepackage{template/dbt}
\usepackage{listings}

\definecolor{comments}{RGB}{81,81,81}
\definecolor{keywords}{RGB}{255,0,90}

% lstlisting
\lstset{
    language=C,
    basicstyle=\footnotesize\ttfamily,
    keywordstyle=\color{keywords},
    showspaces=false,
    showstringspaces=false,
    commentstyle=\color{blue}\emph,
    showtabs=false,
    showspaces=false,
    showstringspaces=false,
    %frame=single,
    %rulecolor=\color{comments},
    %rulesepcolor=\color{comments},
    %backgroundcolor = \color{lightgray}
}

\usetheme{default}

\usepackage[
    type={CC},
    modifier={by-nc-nd},
    version={4.0},
]{doclicense} 

\newcommand{\courseurl}[0]{https://www.cct.lsu.edu/\string~pdiehl/teaching/2021/4997/}
\newcommand{\coursetimeline}[0]{https://www.cct.lsu.edu/~pdiehl/teaching/2021/4997/timeline.pdf}
\newcommand{\coursesyllabus}[0]{https://www.cct.lsu.edu/~pdiehl/teaching/2021/4997/syllabus.pdf}
\newcommand{\coursename}[0]{Math 4997-3}
\newcommand{\coursemailinglist}[0]{https://mail.cct.lsu.edu/mailman/listinfo/par4997}
\newcommand{\coursesemester}{Fall 2021}








% frame slide
\title{\coursename}
\subtitle{Lecture 3: Iterators, Lists, and using library algorithms}

%\author{\href{}{}}
%\institute {
%    \href{}{\tt \scriptsize \today}
%}
\date {
 \tiny \url{https://www.cct.lsu.edu/~pdiehl/teaching/2019/4977/}
\vspace{2cm}
\doclicenseThis  
  
}



\usepackage{ifxetex}

\ifxetex
\usepackage{fontspec}
\setmainfont{Raleway}
\fi

\begin{document} {
    \setbeamertemplate{footline}{}
    \frame {
        \titlepage
    }
}

\frame{

\tableofcontents

}

\AtBeginSection[]{
  \begin{frame}
  \vfill
  \centering
  \begin{beamercolorbox}[sep=8pt,center,shadow=true,rounded=true]{title}
    \usebeamerfont{title}\insertsectionhead\par%
  \end{beamercolorbox}
  \vfill
  \end{frame}
}

%%%%%%%%%%%%%%%%%%%%%%%%%%%%%%%%%%%%%%%%%%%%%%%%%%%%%%%%%%%%%%%%%%%%%%%%%%%%%%
\section{Reminder}
%%%%%%%%%%%%%%%%%%%%%%%%%%%%%%%%%%%%%%%%%%%%%%%%%%%%%%%%%%%%%%%%%%%%%%%%%%%%%%

\begin{frame}{Lecture 2}
\begin{block}{What you should know from last lecture}
\begin{itemize}
\item Monte Carlo Methods
\item Random numbers
\item Containers like \lstinline|std::vector|
\item Functions
\end{itemize}
\end{block}
\end{frame}

%%%%%%%%%%%%%%%%%%%%%%%%%%%%%%%%%%%%%%%%%%%%%%%%%%%%%%%%%%%%%%%%%%%%%%%%%%%%%%
\section{Iterators \lstinline|\#include<iterator>|}
%%%%%%%%%%%%%%%%%%%%%%%%%%%%%%%%%%%%%%%%%%%%%%%%%%%%%%%%%%%%%%%%%%%%%%%%%%%%%%

\begin{frame}{Iterators: \lstinline|\#include<iterator>|}
When we know that we access the elements of the vector sequentially, we can let the compiler know that we are doing this by using iterators.
\vspace{1cm}
\begin{block}{Iterators are values that}
\begin{itemize}
\item identifies a container and an element in the container
\item let us access the value stored in that element
\item provides operations for moving between elements
\item are needed for the algorithms of the standard library
\end{itemize}
\end{block}


\end{frame}


\begin{frame}[fragile]{Iterating over vectors}

\begin{block}{Easiest}
\begin{lstlisting}[language=C]
std::vector<int> values;
for(size_t i = 0 ; i < values.size(); i++)
  std::cout << values[i] << std::endl;
\end{lstlisting}
\end{block}

\begin{block}{Using the \lstinline|size_type|\footnote{\tiny\url{https://en.cppreference.com/w/cpp/types/size_t}}}
\begin{lstlisting}[language=C]
std::vector<int> values;
std::vector<int>::size_type i = 0;
for(; i < values.size(); i++)
   std::cout << values[i] << std::endl;
\end{lstlisting}
\end{block}

\end{frame}


\begin{frame}[fragile]{Advanced iterating over vectors}

\begin{block}{Example}
\begin{lstlisting}
for(
 std::vector<int>::const_iterator iter =
 values.begin();
 iter != values.end(); 
 ++iter
)
{
   std::cout << *iter << std::endl;
}
\end{lstlisting}
\end{block}
\begin{block}{Features}
\begin{itemize}
\item \lstinline|const_iterator| allows read-only access
\item \lstinline|++iter| increments the iterator to the next element
\item Dereference the iterator \lstinline|*iter| to access the value
\end{itemize}
\end{block}


\end{frame}

\begin{frame}[fragile]{Erasing elements with iterators gets easier }

\begin{block}{Using the basic way }
\begin{lstlisting}
std::vector<int> values = {1,2,3};
values.erase(values.begin()+i)
\end{lstlisting}
\end{block}

\begin{block}{Using iterators}
\begin{lstlisting}
values.erase(iter)
\end{lstlisting}
\end{block}
Note that with an iterator there is no need to compute the position anymore!

\begin{block}{Useful feature}
\lstinline| iter = values.erase(iter)|
\end{block}
Returns the iterator pointing to the element after the erasure.
\end{frame}


%%%%%%%%%%%%%%%%%%%%%%%%%%%%%%%%%%%%%%%%%%%%%%%%%%%%%%%%%%%%%%%%%%%%%%%%%%%%%%
\section{Lists \lstinline|\#include<list>|}
%%%%%%%%%%%%%%%%%%%%%%%%%%%%%%%%%%%%%%%%%%%%%%%%%%%%%%%%%%%%%%%%%%%%%%%%%%%%%%

\begin{frame}{Lists vs Vectors}

\begin{block}{Vectors \lstinline|\#include<vector>|}
\begin{itemize}
\item Are sufficient for small amount of elements (around 7000)
\item Is optimized to access elements arbitrary
\item Performs well adding one element by time to its end
\end{itemize}
\end{block}

\begin{block}{Lists \lstinline|\#include<list>|  }
\begin{itemize}
\item Are slower for small amount of elements
\item Are optimized to insert and delete elements anywhere
\end{itemize}
\end{block}
\begin{block}{Complexity}
\begin{itemize}
\item Inserting/Removing: Vector $\mathcal{O}(n^2)$ vs List $\mathcal{O}(n)$~\cite{Wirth:1978:ADS:540029,knuth1997art}
\end{itemize}
\end{block}
\end{frame}

\begin{frame}{Example lists\footnote{\tiny\url{https://en.cppreference.com/w/cpp/container/list}}}

\lstinputlisting{code/lecture3-averageList.cpp}

\end{frame}

%%%%%%%%%%%%%%%%%%%%%%%%%%%%%%%%%%%%%%%%%%%%%%%%%%%%%%%%%%%%%%%%%%%%%%%%%%%%%%
\section{Library algorithms \lstinline|\#include<algorithm>|}
%%%%%%%%%%%%%%%%%%%%%%%%%%%%%%%%%%%%%%%%%%%%%%%%%%%%%%%%%%%%%%%%%%%%%%%%%%%%%%


\begin{frame}[fragile]{Sorting\footnote{\tiny\url{https://en.cppreference.com/w/cpp/algorithm/sort}}}

\begin{block}{Sorting using $<$}
\begin{lstlisting}
 std::sort(s.begin(), s.end());
\end{lstlisting}
\end{block}

\begin{block}{Sorting using $>$\footnote{\tiny\url{https://en.cppreference.com/w/cpp/utility/functional/greater}}}
\begin{lstlisting}
 std::sort(s.begin(), s.end(), std::greater<int>());
\end{lstlisting}
\end{block}

\begin{block}{Advanced sorting using a lambda expression\footnote{\tiny\url{https://en.cppreference.com/w/cpp/language/lambda}} }
\begin{lstlisting}
    std::sort(s.begin(), s.end(), [](int a, int b) {
        return a > b; 
\end{lstlisting}
\end{block}
We will look into lambda expression later in more detail

\end{frame}

\begin{frame}[fragile]{Accumulate}

\begin{block}{Sum\footnote{\tiny\url{https://en.cppreference.com/w/cpp/algorithm/accumulate}}}
\begin{lstlisting}
int sum = std::accumulate(v.begin(), v.end(), 0);
\end{lstlisting}
\end{block}

\begin{block}{Multiplication\footnote{\tiny\url{https://en.cppreference.com/w/cpp/utility/functional/multiplies}}}
\begin{lstlisting}
int product = 
	std::accumulate(v.begin(), v.end(), 0, 
	std::multiplies<int>());
\end{lstlisting}
\end{block}
Note that zero is the initial value of the accumulate

\begin{block}{Various}
\begin{itemize}
\item \lstinline|std::inner_product| 
\item \lstinline|std::partial_sum|
\end{itemize}
\end{block}

\end{frame}

\begin{frame}[fragile]{Removing elements}

\begin{block}{Remove\footnote{\tiny\url{https://en.cppreference.com/w/cpp/algorithm/remove}}}
\begin{lstlisting}
std::list<int> l = { 1,100,2,3,10,1,11,-1,12 };
l.remove(1); // remove both elements equal to 1
\end{lstlisting}
\end{block}

\begin{block}{Remove\_if\footnote{\tiny\url{http://www.cplusplus.com/reference/algorithm/remove_if/}}}
\begin{lstlisting}
//Define function
bool IsOdd (int i) {
  return ((i%2)==1);
}
//Check for the first odd number
l.remove_if(IsOdd); // remove all odd numbers
\end{lstlisting}
\end{block}


\end{frame}

\begin{frame}[fragile]{Searching for existence of elements}
\begin{block}{Find\footnote{\tiny\url{https://en.cppreference.com/w/cpp/algorithm/find}}}
\begin{lstlisting}
int n1 = 3;
std::vector<int> v{0, 1, 2, 3, 4}; 
auto result1 = 
	std::find(std::begin(v), std::end(v), n1);
\end{lstlisting}
\end{block}

\begin{block}{Find\_if\footnote{\tiny\url{http://www.cplusplus.com/reference/algorithm/find_if/}}}
\begin{lstlisting}
//Define function
bool IsOdd (int i) {
  return ((i%2)==1);
}
//Check for the first odd number
std::vector<int>::iterator it = 
	std::find(std::begin(v), std::end(v), IsOdd);
\end{lstlisting}
\end{block}

\end{frame}


\begin{frame}[fragile]{Search\footnote{\tiny\url{https://en.cppreference.com/w/cpp/algorithm/search}}}
Find a substring within a string

\begin{block}{Check for the substring }
\begin{lstlisting}
std::string name = "Math 4997-3";
std::string exp = "4997";
std::search(name.begin(), name.end(), 
	exp.begin(), exp.end()) != name.end();
\end{lstlisting}
\end{block}

\begin{block}{Get the leading position\footnote{\tiny\url{https://en.cppreference.com/w/cpp/utility/functional/boyer_moore_searcher}}}
\begin{lstlisting}
auto it = std::search(name.begin(), name.end(),
	std::boyer_moore_searcher(
	exp.begin(), exp.end()));
std::cout << it - name.begin() << std::endl;
\end{lstlisting}
\end{block}

\end{frame}

\begin{frame}[fragile]{Copy}
\begin{center}
Copy the content of vector a to vector b
\end{center}

\begin{block}{Without library algorithm}
\begin{lstlisting}
typedef std::vector<int>::const_iterator it vit; 
for(vit it = a.begin(); it != a.end(); ++it)
{
	b.push_back(*it);
}
\end{lstlisting}
\end{block}

\begin{block}{With library algorithm\footnote{\tiny\url{https://en.cppreference.com/w/cpp/algorithm/copy}}}
\begin{lstlisting}
std::copy(a.begin(),a.end(),b)
\end{lstlisting}
\end{block}

\end{frame}

\begin{frame}[fragile]{Insert}
\begin{center}
Append the content of vector a to vector b
\end{center}

\begin{block}{Without library algorithm}
\begin{lstlisting}
typedef std::vector<int>::const_iterator it vit; 
for(vit it = a.begin(); it != a.end(); ++it)
{
	b.push_back(*it);
}
\end{lstlisting}
\end{block}

\begin{block}{With library algorithm}
\begin{lstlisting}
b.insert(b.end(),a.begin(),a.end());
\end{lstlisting}
\end{block}

\end{frame}

\begin{frame}[fragile]{Filling vectors}

\begin{block}{Fill vector with one value\footnote{\tiny\url{https://en.cppreference.com/w/cpp/algorithm/fill}}}
\begin{lstlisting}
std::vector<int> v{0, 1, 2, 3, 4, 5, 6, 7, 8, 9};
//Replace all elements by -1
std::fill(v.begin(), v.end(), -1);
\end{lstlisting}
\end{block}

\begin{block}{Replacing the first 5 elements\footnote{\tiny\url{https://en.cppreference.com/w/cpp/algorithm/fill_n}}}
\begin{lstlisting}
//Replace the first 5 elements by -1
std::fill_n(v1.begin(), 5, -1);
\end{lstlisting}
\end{block}

\begin{block}{Replacing the last 5 elements\footnote{\tiny\url{https://en.cppreference.com/w/cpp/iterator/back_inserter}}}
\begin{lstlisting}
//Replace the first 5 elements by -1
std::fill_n(std::back_inserter(v), 5, -1);
\end{lstlisting}
\end{block}

\end{frame}


\begin{frame}[fragile]{Transform\footnote{\tiny\url{https://en.cppreference.com/w/cpp/algorithm/transform}}}

\begin{block}{Convert to upper case letters}
\begin{lstlisting}
std::string s("hello");
    std::transform(s.begin(), s.end(), s.begin(),
    [](unsigned char c) -> 
    unsigned char { return std::toupper(c); });
\end{lstlisting}
\end{block}


\end{frame}

\begin{frame}{Partition}

\lstinputlisting[firstline=3, lastline=20]{code/lecture3-partition.cpp}

\end{frame}

%%%%%%%%%%%%%%%%%%%%%%%%%%%%%%%%%%%%%%%%%%%%%%%%%%%%%%%%%%%%%%%%%%%%%%%%%%%%%%
\section{Numeric limits \lstinline|\#include<limits>|}
%%%%%%%%%%%%%%%%%%%%%%%%%%%%%%%%%%%%%%%%%%%%%%%%%%%%%%%%%%%%%%%%%%%%%%%%%%%%%%

\begin{frame}{Limits}

\lstinputlisting{code/lecture3-limits.cpp}

\begin{itemize}
\item \lstinline|::min| returns the smallest finite value of the given type 
\item \lstinline|::max| returns the largest finite value of the given type 
\end{itemize}
More details about IEEE 754~\cite{goldberg1991every,4610935}
\end{frame}


\begin{frame}{Limits\footnote{\tiny\url{https://en.cppreference.com/w/cpp/types/numeric_limits}}}

\lstinputlisting{code/lecture3-limits2.cpp}

\begin{itemize}
\item \lstinline|::round_error| returns the maximum rounding error of the given floating-point type  
\item \lstinline|::epsilon| returns the difference between 1.0 and the next representable value of the given floating-point type 
\end{itemize}

\end{frame}

%%%%%%%%%%%%%%%%%%%%%%%%%%%%%%%%%%%%%%%%%%%%%%%%%%%%%%%%%%%%%%%%%%%%%%%%%%%%%%
\section{IO}
%%%%%%%%%%%%%%%%%%%%%%%%%%%%%%%%%%%%%%%%%%%%%%%%%%%%%%%%%%%%%%%%%%%%%%%%%%%%%%

\begin{frame}[fragile]{Writing files\footnote{\tiny\url{https://en.cppreference.com/w/cpp/io/basic_ofstream}}}

\begin{lstlisting}
// basic file operations
#include <iostream>
#include <fstream>

int main () {
  std::ofstream myfile;
  myfile.open ("example.txt", std::ios::out);
  myfile << "Writing this to a file.\n";
  myfile.close();
  return 0;
}
\end{lstlisting}

\begin{block}{Mode\footnote{\tiny\url{https://en.cppreference.com/w/cpp/io/ios_base/openmode}}}
\begin{itemize}
\item \lstinline|out| Open for writing (Default)
\item \lstinline|app| Append to the end 
\end{itemize}

\end{block}

\end{frame}

\begin{frame}[fragile]{Reading files\footnote{\tiny\url{https://en.cppreference.com/w/cpp/io/basic_ifstream}}~\footnote{\tiny\url{https://en.cppreference.com/w/cpp/io/basic_fstream}}}

\begin{lstlisting}
#include <iostream>
#include <fstream>
#include <string>

int main () {
  string line;
  ifstream myfile ("example.txt");
  if (myfile.is_open())
  {
    while ( getline (myfile,line) )
    {
      cout << line << '\n';
    }
    myfile.close();
  }
  return 0;
}
\end{lstlisting}


\end{frame}

%%%%%%%%%%%%%%%%%%%%%%%%%%%%%%%%%%%%%%%%%%%%%%%%%%%%%%%%%%%%%%%%%%%%%%%%%%%%%%
\section{Summary}
%%%%%%%%%%%%%%%%%%%%%%%%%%%%%%%%%%%%%%%%%%%%%%%%%%%%%%%%%%%%%%%%%%%%%%%%%%%%%%

\begin{frame}{Summary}
\begin{block}{After this lecture, you should know}
\begin{itemize}
\item Iterators
\item Lists
\item Library algorithms
\item Numerical limits
\item Reading and Writing files
\end{itemize}
\end{block}


\end{frame}


%%%%%%%%%%%%%%%%%%%%%%%%%%%%%%%%%%%%%%%%%%%%%%%%%%%%%%%%%%%%%%%%%%%%%%%%%%%%%%
\section{References}
%%%%%%%%%%%%%%%%%%%%%%%%%%%%%%%%%%%%%%%%%%%%%%%%%%%%%%%%%%%%%%%%%%%%%%%%%%%%%%

\begin{frame}[t, allowframebreaks]
\frametitle{References}
\bibliographystyle{plain}
\bibliography{bib}
\end{frame}


\end{document}